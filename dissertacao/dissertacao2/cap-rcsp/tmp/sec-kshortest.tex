\section{Ranqueamento de caminhos}
\label{sec:ksp}

O problema de ranqueamento do caminhos, mais conhecido como o problema 
dos $k$ menores caminhos ($\textsc{KSP}$ -- \emph{$k$ shortest path}), 
consiste me determinar o $k$-�simo menor caminho em um grafo. Este 
problema foi estudado primeiramente por \citet{hoffman:xx}. Desde ent�o, 
o problema tem sido massivamente estudado e v�rias solu��es foram 
propostas.

Muitos desses m�todos tem complexidade de tempo polinomial para um $k$ 
fixo. \citet{eppstein:xx} descreveu um algoritmo como complexidade $O(m 
+ n \lg{n} + k)$ que resolve o problema quando ciclos s�o permitidos.  
Podemos usar o $\textsc{KSP}$ para resolver o \rcsp.  Neste caso, 
ignoramos o $k$ e enumeramos os caminhos em ordem n�o decrescente de 
custo at� encontrar um caminho vi�vel, tal abordagem resulta em um 
algoritmo de complexidade exponencial para o \rcsp.

O uso do \textsc{KSP} n�o � recomendado para ser usado diretamente para 
resolver o \rcsp. Porem, ele tem sido usado com sucesso como 
sub-problema para m�todos mais sofisticados coma a relaxa��o lagrangiana 
proposta por \citet{zang:80}. Uma extensa bibliografia para o problema 
dos $k$ menores caminhos pode ser encontrado em \citet{eppstein:xx};

%Path ranking, often known as the K shortest path problem, seeks the K 
%(unconstrained)
%shortest paths in a graph. This problem was first studied by Hoffman and 
%Pavley [67]. Since
%then, several methods have been proposed (see for example [37, 73, 78, 
%91, 114]). Most of
%the methods have polynomial time complexity for fixed K. Eppstein [37] 
%describes an
%algorithm with complexity O(|A| + |V | log |V | + K) that solves the 
%problem when cycles are
%allowed. When used to solve the RCSPP, however, K is ignored and paths 
%are enumerated
%in non-decreasing order of cost until the first weight feasible path is 
%found, resulting in
%an exponential time algorithm. While this discourages the use of path 
%ranking to solve
%the RCSPP directly, it has been used successfully as a subroutine in 
%more complicated
%approaches [61, 92]. An extensive bibliography for the K shortest path 
%problem can be
%found on the website of Eppstein [38].

