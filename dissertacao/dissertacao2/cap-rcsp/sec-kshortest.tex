\section{Ranqueamento de caminhos}
\label{sec:ksp}

O problema de ranqueamento de caminhos, mais conhecido como o problema 
dos $k$ menores caminhos ($\textsc{KSP}$ -- \emph{$k$ shortest path}), 
consiste me determinar o $k$-�simo menor caminho em um grafo. Este 
problema foi estudado primeiramente por \citet{hoffman:59}. Desde ent�o, 
o problema tem sido massivamente estudado e v�rias solu��es foram 
propostas.

Muitos desses m�todos tem complexidade de tempo polinomial para um $k$ 
fixo. \citet{eppstein:94} descreveu um algoritmo com complexidade de 
tempo $O(m + n \lg{n} + k)$ que resolve o problema quando ciclos s�o 
permitidos.  Podemos usar o $\textsc{KSP}$ para resolver o \rcsp.  Neste 
caso, ignoramos o $k$ e enumeramos os caminhos em ordem n�o decrescente 
de custo at� encontrar um caminho vi�vel, tal abordagem resulta em um 
algoritmo de complexidade exponencial para o \rcsp.

O \textsc{KSP} n�o � recomendado para ser usado diretamente para 
resolver o \rcsp. Porem, ele tem sido usado com sucesso como 
sub-problema para m�todos mais sofisticados como a relaxa��o lagrangiana 
proposta por \citet{zang:80}. Uma extensa bibliografia para o problema 
dos $k$ menores caminhos pode ser encontrada em \citet{eppstein:12}.

