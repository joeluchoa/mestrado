%% ------------------------------------------------------------------ %%
\section{Defini��o do problema}
\label{sec:definicao}

\begin{problema}{\rcsp($G, \scor, \tcor, k, r, \lambda, c$)} Como par�metros do 
  problema s�o dados
\begin{itemize}
\item\ um grafo dirigido $G=(V,A)$,
\item\ um v�rtice origem $\scor \in V$ e um v�rtice destino $\tcor \in V$, $\scor 
\neq \tcor$,
\item\ um n�mero $k \in \mathbb{N}$ de recursos dispon�veis 
$\{1,\dots,k\}$,
\item\ o consumo de recursos $r^i_{uv} \in \mathbb{N}_0$ de cada arco de 
  $G$ sobre os  $k$ recursos dispon�veis, $i = 1, \dots, k$, $uv \in A$,
\item\ o limite $\lambda^i \in \mathbb{N}_0$ que dispomos de cada recurso, $i 
= 1, \dots, k$,
\item\ o custo $c_{uv} \in \mathbb{N}_0$, para cada arco, $uv \in A$.
\end{itemize}

O consumo de um recurso $i$, $i = 1, \dots, k$ em um $st$-caminho $P$ � 
$r^i(P) = \sum_{uv \in P}{r^i_{uv}}$. Um $st$-caminho $P$ � limitado 
pelos recursos $1, \dots, k$ se este consome n�o mais que o limite
dispon�vel de cada recurso, ou seja, se $ r^i(P) \leq \lambda^i$, $i = 1, 
\dots, k$.
O custo de um $st$-caminho $P$ � $c(P) = \sum_{uv \in P}{c_{uv}}$.
O problema \rcsp\ consiste em encontrar o caminho limitado pelos 
recursos de
menor custo.

\end{problema}

Usaremos no decorrer deste trabalho $n = |V|$ e $m = |A|$. Quando
estivermos tratando de um contexto onde existe apenas um recurso 
(\srcsp), ou seja, $k = 1$, usaremos apenas $\lambda$ para representar 
$\lambda^1$ e apenas $r_{uv}$ para representar $r^1_{uv}$. 

