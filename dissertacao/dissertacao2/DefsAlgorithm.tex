\usepackage{ifthen}
\newcommand{\eqdef}{:=}
\newcommand{\recebe}{\leftarrow}
\newcommand{\Enqto}{\textbf{Enquanto~}}
\newcommand{\enqto}{\textbf{enquanto~}}
\newcommand{\Repita}{\textbf{Repita~}}
\newcommand{\Ate}{\textbf{Até~}}
\newcommand{\Para}{\textbf{Para~}}
\newcommand{\faca}{\textbf{faça~}}
\newcommand{\Se}{\textbf{Se~}}
\newcommand{\Senao}{\textbf{Senão~}}
\newcommand{\entao}{\textbf{então~}}
\newcommand{\Devolva}{\textbf{Devolva~}}
\newcommand{\comentario}[1]{/* \textit{#1} */}
\newcommand{\entrada}[1]{\x\textsc{Entrada: }#1}
\newcommand{\saida}[1]{\x\textsc{Saída: }#1}

\newcommand{\lb}{\linebreak}

% PSEUDOCODE for ALGORITHMS 
\newenvironment{pseudoc}
         {\begin{list}%
                {}%
                {%
                 \sffamily%
                 \setlength{\partopsep}{1ex}%
                 \setlength{\topsep}{1ex}%
                 \setlength{\leftmargin}{\parindent}%
                 \setlength{\parsep}{0.5ex}%
                }%
          \item}%
         {\end{list}%
         }
\newenvironment{pseudocode}%
         {\begin{pseudoc}%
          \textbf{\textrm{Algoritmo}}\hspace{0.7ex}\ignorespaces%
         }%
         {\end{pseudoc}}
\newenvironment{pseudocodem}%
         {\begin{pseudoc}%
          \textbf{\textrm{Método}}\hspace{0.7ex}\ignorespaces%
         }%
         {\end{pseudoc}}
\newenvironment{pseudocodeproc}%
         {\begin{pseudoc}%
          \textbf{\textrm{Procedimento}}\hspace{0.7ex}\ignorespaces%
         }%
         {\end{pseudoc}}
\newenvironment{pseudocex}% dentro de exercícios
         {\begin{small}%
          \begin{list}%
                {}%
                {%
                 \sffamily%
                 \setlength{\partopsep}{0.5ex}%
                 \setlength{\topsep}{1ex}%
                 \setlength{\leftmargin}{8ex}%
                 \setlength{\parsep}{0.2ex}%
                }%
          \item}%
         {\end{list}%
          \end{small}%
         }
\newenvironment{pseudocodeex}%
         {\begin{pseudocex}%
          \textbf{\textrm{Algoritmo}}\hspace{0.7ex}\ignorespaces%
         }%
         {\end{pseudocex}}

\newcommand{\phentao}{\phantom{então}} 
\newcommand{\phsenao}{\phantom{senão}} 
\newcommand{\x}{\hspace*{4ex}} % era 3ex
\newcommand{\xx}{\hspace*{8ex}}
\newcommand{\xxx}{\hspace*{12ex}}
\newcommand{\xxxx}{\hspace*{16ex}}
\newcommand{\xxxxx}{\hspace*{20ex}}
\newcommand{\xxxxxx}{\hspace*{24ex}}
\newlength{\tsts}
\settowidth{\tsts}{9}
\newcommand{\ts}{\hspace{\tsts}}
\newcommand{\RM}{\textrm}
% Example:
% \begin{pseudocode}
% \max $(A, p, r)$
%    
% \RM{\ts1}\x se $p = r$
% 
% \RM{\ts2}\xx então devolva $A[p]$
%    
% \RM{\ts3}\xx senão $m\leftarrow \textsc{Máximo}\,(A,p+1,r)$
% 
% \RM{\ts9}\xx\phsenao se $m < A[p]$ \quad $\rhd$ comentário
% 
% \RM{10}\xxx\phsenao então $m\leftarrow A[p]$
%  
% \RM{11}\xx\phsenao devolva $m$
% \end{pseudocode}s

