% Arquivo LaTeX de exemplo de dissertação/tese a ser apresentados à CPG do IME-USP
% 
% Versão 4: Sun Feb 20 18:25:33 BRT 2011
%
% Criação: Jesús P. Mena-Chalco
% Revisão: Fabio Kon e Paulo Feofiloff
%  
% Obs: Leia previamente o texto do arquivo README.txt

\documentclass[12pt,twoside,a4paper]{book}

% Pacotes 
\usepackage[T1]{fontenc}
\usepackage[brazil]{babel}
\usepackage[utf8]{inputenc}
\usepackage{amsthm}
\usepackage{amsfonts}
\usepackage{graphicx}
\usepackage{setspace}                   % espa�amento flex�vel
\usepackage{indentfirst}                % indenta��o do primeiro par�grafo
\usepackage[nottoc]{tocbibind}          % acrescentamos a bibliografia/indice/conteudo no Table of Contents
%\usepackage{courier}                    % usa o Adobe Courier no lugar de Computer Modern Typewriter
\usepackage{type1cm}                    % fontes realmente escal�veis
\usepackage{listings}                   % para formatar c�digo-fonte (ex. em Java)
\usepackage{titletoc}
%\usepackage[bf,small,compact]{titlesec} % cabe�alhos dos t�tulos: menores e compactos
\usepackage[fixlanguage]{babelbib}
\usepackage[font=small,format=plain,labelfont=bf,up,textfont=it,up]{caption}
\usepackage[usenames,svgnames,dvipsnames]{xcolor}

\usepackage[a4paper,top=2.54cm,bottom=2.0cm,left=2.0cm,right=2.54cm]{geometry} % margens
%\usepackage[pdftex,plainpages=false,pdfpagelabels,pagebackref,colorlinks=true,citecolor=black,linkcolor=black,urlcolor=black,filecolor=black,bookmarksopen=true]{hyperref} % links em preto
\usepackage[plainpages=false,pdfpagelabels,pagebackref,colorlinks=true,citecolor=DarkGreen,linkcolor=NavyBlue,urlcolor=DarkRed,filecolor=green,bookmarksopen=true]{hyperref} % links coloridos
\usepackage[all]{hypcap}                % soluciona o problema com o hyperref e capitulos
\fontsize{60}{62}\usefont{OT1}{cmr}{m}{n}{\selectfont}

% Util
\usepackage{enumerate}
\usepackage{verbatim}
%\usepackage[svgnames]{xcolor}

% Figures
\usepackage{float}
\usepackage{subfig}

%\usepackage{ upgreek }

\hyphenpenalty=10000000
\tolerance=10000000
% to avoid annoying message "Underful [h/v]box"
\hbadness=100000
\vbadness=100000

\setlength\parskip{0.3cm}

%-------------------------------------------------------%
% Chapter marks and page headers
\usepackage{fancyhdr}
\usepackage[avantgarde]{quotchap2}


% ---------------------------------------------------------------------------- %
% Cabe�alhos similares ao TAOCP de Donald E. Knuth
\usepackage{fancyhdr}
\pagestyle{fancy}
\fancyhf{}
\renewcommand{\chaptermark}[1]{\markboth{\rm\textsc{#1}}{}}
\renewcommand{\sectionmark}[1]{\markright{\rm\textsc{#1}}{}}
\renewcommand{\headrulewidth}{0.4pt}
\renewcommand{\footrulewidth}{0pt}

\usepackage{titlesec}
\titleformat{\section}
{\normalfont\Large\scshape\bfseries}{\thesection}{1em}{}
\titleformat{\subsection}
{\normalfont\large\scshape\bfseries}{\thesubsection}{1em}{}
\titleformat{\subsubsection}
{\normalfont\normalsize\scshape\bfseries}{\thesubsubsection}{1em}{}
\titleformat{\paragraph}[runin]
{\normalfont\normalsize\scshape\bfseries}{\theparagraph}{1em}{}
\titleformat{\subparagraph}[runin]
{\normalfont\normalsize\scshape\bfseries}{\thesubparagraph}{1em}{}
\titlespacing*{\chapter}
{0pt}{50pt}{40pt}
\titlespacing*{\section}
{0pt}{3.5ex plus 1ex minus .2ex}{2.3ex plus .2ex}
\titlespacing*{\subsection}
{0pt}{3.25ex plus 1ex minus .2ex}{1.5ex plus .2ex}
\titlespacing*{\subsubsection}{0pt}{3.25ex plus 1ex minus .2ex}{1.5ex plus .2ex}
\titlespacing*{\paragraph}
{0pt}{3.25ex plus 1ex minus .2ex}{1em}
\titlespacing*{\subparagraph} {\parindent}{3.25ex plus 1ex minus .2ex}{1em}


% ---------------------------------------------------------------------------- %

\definecolor{MyGrey}{rgb}{0.96,0.97,0.98}
\makeatletter\newenvironment{greybox}{%
   \begin{lrbox}{\@tempboxa}\begin{minipage}{\columnwidth}}{\end{minipage}\end{lrbox}%
   \colorbox{MyGrey}{\usebox{\@tempboxa}}
}\makeatother

%Todo: Arrumar para não imprimir a seção 0
\makeatletter\renewcommand{\thesection}{%
	\@arabic\c@chapter
    \ifnum \c@section=0 \relax \else .\@arabic\c@section \fi
}\makeatother
% ---------------------------------------------------------------------------- %
\graphicspath{{./figuras/}}              % caminho das figuras (recomendável)
\frenchspacing                          % arruma o espaço: id est (i.e.) e exempli gratia (e.g.) 
\urlstyle{same}                         % URL com o mesmo estilo do texto e não mono-spaced
%\makeindex                              % para o índice remissivo
\raggedbottom                           % para não permitir espaços extra no texto
\fontsize{60}{62}\usefont{OT1}{cmr}{m}{n}{\selectfont}
\cleardoublepage
\normalsize

\def\PP{{\mathbb P}}
\def\EE{{\mathbb E}}
\def\ZZ{{\mathbb Z}}
\def\NN{{\mathbb N}}
\def\QQ{{\mathbb Q}}

\def\CC{{\mathbb C}}
\def\DD{{\mathbb D}}
\def\GG{{\mathbb G}}
\def\HH{{\mathbb H}}
\def\II{{\mathbb I}}
\def\RR{{\mathbb R}}
\def\TT{{\mathbb T}}

\def\FF{{\mathcal F}}

\def\PTAS{\mathop{\text{\rm PTAS}}}
\def\APX{\mathop{\text{\rm APX}}}
\def\P{\mathop{\text{\rm P}}}
\def\NP{\mathop{\text{\rm NP}}}
\def\NPd{\mathop{\text{\rm NP-difícil}}}
\def\NPc{\mathop{\text{\rm NP-completo}}}

\newcommand{\e}{\textrm{~e~}}

%Conceitos gerais
\newcommand{\FaceExt}{f_{\infty}}

%XinHe
\newcommand{\FaceEsq}[1]{face_{esq}(#1)}
\newcommand{\FaceDir}[1]{face_{dir}(#1)}

\newcommand{\FaceAcima}[1]{face_{acima}(#1)}
\newcommand{\FaceAbaixo}[1]{face_{abaixo}(#1)}

\newcommand{\snRede}{\textit{S,N}\textrm{-rede~}}
\newcommand{\olRede}{\textit{O,L}\textrm{-rede~}}

\newcommand{\stgrafo}{\textit{s,t}\textrm{-grafo~}}
\newcommand{\stgrafos}{\textit{s,t}\textrm{-grafos~}}

\newcommand{\stgrafosN}{\textit{s,t}\textrm{-grafos}}

\newcommand{\FaceExtEsq}{f^{esq}_{\infty}}
\newcommand{\FaceExtDir}{f^{dir}_{\infty}}

\newcommand{\FaceExtAcima}{f^{acima}_{\infty}}
\newcommand{\FaceExtAbaixo}{f^{abaixo}_{\infty}}

%--------------------------------------

\newtheorem{problema}{Problema}
%\newtheorem{definicao}{Definição}[chapter]
%\newtheorem{teorema}{Teorema}[chapter]
%\newtheorem{conjectura}[teorema]{Conjectura}
%\newtheorem{proposicao}[teorema]{Proposição}
%\newtheorem{lema}[teorema]{Lema}

%\newcommand{\qed}{$\Box$}
\newenvironment{baretheorem}%
         {\begin{list}%
                {}%
                {%
                 %\addtolength{\partopsep}{0.5ex}%
                 \setlength{\partopsep}{1ex}% igual em prova
%                \addtolength{\topsep}{-0.5ex}%
                 \setlength{\topsep}{0.8ex}% igual prova
                 \setlength{\leftmargin}{\parindent}%
                 \setlength{\rightmargin}{\parindent}%
                 \setlength{\itemindent}{0ex}%
                 \setlength{\labelsep}{\itemindent}%
                 \setlength{\labelwidth}{0ex}%
                 \setlength{\parsep}{1ex}% parskip is 1.5ex
                 \setlength{\listparindent}{0ex}%
                }%
          \item%
         }%
         {\end{list}%
         }
\newcounter{theoremnumber}[chapter]
\renewcommand{\thetheoremnumber}{\thechapter.\arabic{theoremnumber}}
\newenvironment{taggednumberedtheorem}[2]%
               {%
                \refstepcounter{theoremnumber}%
                \begin{baretheorem}%
                \textbf{#1\hspace{0.7ex}\thetheoremnumber}%
                \ifthenelse{\equal{#2}{}}% Lamport p.195
                        {\hspace{1pt}:\hspace{1ex}\ignorespaces}%
                        {\hspace{1ex}\textup{(#2)}:\hspace{1ex}\ignorespaces}%
                \slshape%
               }%
               {\end{baretheorem}%
               }

\newcounter{definitionnumber}[chapter]
\renewcommand{\thedefinitionnumber}{\thechapter.\arabic{definitionnumber}}
\newenvironment{taggednumbereddefinition}[2]%
               {%
                \refstepcounter{definitionnumber}%
                \begin{baretheorem}%
                \textbf{#1\hspace{0.7ex}\thedefinitionnumber}%
                \ifthenelse{\equal{#2}{}}% Lamport p.195
                        {\hspace{1pt}:\hspace{1ex}\ignorespaces}%
                        {\hspace{1ex}\textup{(#2)}:\hspace{1ex}\ignorespaces}%
                \slshape%
               }%
               {\end{baretheorem}%
               }


\newenvironment{lema}[1]%
         {\begin{taggednumberedtheorem}{Lema}{#1}%
         }%
         {\end{taggednumberedtheorem}%
         }
\newenvironment{teorema}[1]%
         {\begin{taggednumberedtheorem}{Teorema}{#1}%
         }%
         {\end{taggednumberedtheorem}%
         }
\newenvironment{fato}[1]%
         {\begin{taggednumberedtheorem}{Fato}{#1}%
         }%
         {\end{taggednumberedtheorem}%
         }
\newenvironment{definicao}[1]%
         {\begin{taggednumbereddefinition}{Definição}{#1}%
         }%
         {\end{taggednumbereddefinition}%
         }
\newenvironment{corolario}[1]%
         {\begin{taggednumberedtheorem}{Corolário}{#1}%
         }%
         {\end{taggednumberedtheorem}%
         }

\newenvironment{bareproof}%
         {\begin{list}%
                {}%
                {%
                 \setlength{\partopsep}{1ex}% como em baretheorem
                 \setlength{\topsep}{0.8ex}% como em baretheorem
                 \setlength{\leftmargin}{0ex}%
                 \setlength{\labelwidth}{0ex}% 
                 \setlength{\labelsep}{1ex}%
                 \setlength{\itemindent}{\parindent}% 
                 \addtolength{\itemindent}{\labelsep}%
                 \setlength{\parsep}{1ex}% parskip is 1.5ex
                 \setlength{\listparindent}{\parindent}%
                }%
         }%
         {\end{list}}
\newenvironment{prova}[1]
         {\begin{bareproof}%
          \item[Demonstração:]%
            #1%
         }%
         {\hfill\qed\ignorespacesafterend
          \end{bareproof}%
         }

\newenvironment{exemplo}[1]%
         {\begin{taggednumberedtheorem}{Exemplo}{#1}%
         }%
         {\end{taggednumberedtheorem}%
         }

%\renewcommand{\thetheorem}{\thechapter.\arabic{theorem}}

\def\({\left(}
\def\){\right)}  
\def\<{\langle}
\def\>{\rangle}
\let\:=\colon

% MATHEMATICS
%%%%%%%%%%%%%
  \newcommand{\binomial}% to be used in displaystyle only
             [2]%
             {\big(\!\begin{array}{c}%
                 #1\\[-0.5ex]%
                 #2%
                 \end{array}%
               \!\big)}
  \newcommand{\floor}[1]{\lfloor #1 \rfloor}
  \newcommand{\ceil}[1]{\lceil #1 \rceil}
  \newcommand{\seq}[1]{( #1 )}
  \newcommand{\tam}[1]{\langle #1 \rangle}
% número total de dígitos
  \newcommand{\conjunto}[1]{\{\,#1\,\}}
  \newcommand{\conj}[1]{\{#1\}}
  \newcommand{\twodots}{\,.\,.\,} % in place of \ldots
  \newcommand{\card}[1]{{\mid}#1{\mid}} 
% $\card{X}$ é o mesmo que $|X|$
  \newcommand{\dcard}[1]{{\parallel}#1{\parallel}} 
% $\dcard{X}$ é o mesmo que $||X||$
%\renewcommand{\emptyset}{\mbox{\textup{\O}}} % slash O
%\renewcommand{\emptyset}{\mbox{\textit{\O}}} % slash O
  \newcommand{\eps}{\varepsilon}
% \newcommand{\qed}{\mbox{\hspace{0.2ex}\rule{0.7ex}{0.9ex}}}
%  \newcommand{\qed}{$\Box$}
% \newcommand{\Cdot}{\cdot}
% \newcommand{\Cdot}{}
  \newcommand{\Cdot}{\hspace{1pt}} % produto de vetores $w\Cdot x$
  %\newcommand{\Ddot}{\hspace{0.8pt}} % produto mat por vet e vet por matr
  \newcommand{\Edot}{\hspace{0.8pt}} % produto de matrizes $A\Edot B$
% \newcommand{\dotplus}{\stackrel{.}{+}}
% soma disjunta de grafo com conjunto de arestas $G\dotplus F$
  \newcommand{\dotplus}{+}
  \newcommand{\dotcup}{\stackrel{.}{\cup}}
%% união disjunta de dois conjuntos
%\newcommand{\wt}{\widetilde} % matriz transposta
%\newcommand{\wt}[1]{#1^{\!\top}} % matriz transposta
 \newcommand{\wt}[1]{#1^{\scriptscriptstyle\top}} % matriz transposta
%\newcommand{\tq}{\ :\ } % tal que (no primal-dual)
 \newcommand{\tq}{:} % tal que (no primal-dual)

\usepackage{enumitem}
%usage: \defProblema{label}{nome}{entrada}{saida}

%\newenvironment{mydescription}{%
%   \renewcommand*\descriptionlabel[1]{\hspace\labelsep\rm\textsc{#1}:~}%
%   \begin{description}%
%}{%
%   \end{description}%
%}

\newcommand\descriptionDefProblabel[1]{\hspace{\labelsep}\rm\textsc{#1}:~}
\newenvironment{descriptionDefProb}{%
   \let\descriptionlabel\descriptionDefProblabel
   \description
}{%
   \enddescription
}

\newcommand{\defProblem}[4]{%
	%\parindent=0cm%
	\begin{problema}%
		\label{#1}%
		\hspace{.1cm}%
		\textsc{#2}%
 		\begin{descriptionDefProb}[style=multiline,leftmargin=3.5cm, labelindent=1cm, topsep=-0.1ex, itemsep=0.1ex]%
				\item[Entrada] #3%
				\item[Saida]  #4%
		\end{descriptionDefProb}%
	\end{problema}%
}

\usepackage{ifthen}
\newcommand{\eqdef}{:=}
\newcommand{\recebe}{\leftarrow}
\newcommand{\Enqto}{\textbf{Enquanto~}}
\newcommand{\enqto}{\textbf{enquanto~}}
\newcommand{\Repita}{\textbf{Repita~}}
\newcommand{\Ate}{\textbf{Até~}}
\newcommand{\Para}{\textbf{Para~}}
\newcommand{\faca}{\textbf{faça~}}
\newcommand{\Se}{\textbf{Se~}}
\newcommand{\Senao}{\textbf{Senão~}}
\newcommand{\entao}{\textbf{então~}}
\newcommand{\Devolva}{\textbf{Devolva~}}
\newcommand{\comentario}[1]{/* \textit{#1} */}
\newcommand{\entrada}[1]{\x\textsc{Entrada: }#1}
\newcommand{\saida}[1]{\x\textsc{Saída: }#1}

\newcommand{\lb}{\linebreak}

% PSEUDOCODE for ALGORITHMS 
\newenvironment{pseudoc}
         {\begin{list}%
                {}%
                {%
                 \sffamily%
                 \setlength{\partopsep}{1ex}%
                 \setlength{\topsep}{1ex}%
                 \setlength{\leftmargin}{\parindent}%
                 \setlength{\parsep}{0.5ex}%
                }%
          \item}%
         {\end{list}%
         }
\newenvironment{pseudocode}%
         {\begin{pseudoc}%
          \textbf{\textrm{Algoritmo}}\hspace{0.7ex}\ignorespaces%
         }%
         {\end{pseudoc}}
\newenvironment{pseudocodem}%
         {\begin{pseudoc}%
          \textbf{\textrm{Método}}\hspace{0.7ex}\ignorespaces%
         }%
         {\end{pseudoc}}
\newenvironment{pseudocodeproc}%
         {\begin{pseudoc}%
          \textbf{\textrm{Procedimento}}\hspace{0.7ex}\ignorespaces%
         }%
         {\end{pseudoc}}
\newenvironment{pseudocex}% dentro de exercícios
         {\begin{small}%
          \begin{list}%
                {}%
                {%
                 \sffamily%
                 \setlength{\partopsep}{0.5ex}%
                 \setlength{\topsep}{1ex}%
                 \setlength{\leftmargin}{8ex}%
                 \setlength{\parsep}{0.2ex}%
                }%
          \item}%
         {\end{list}%
          \end{small}%
         }
\newenvironment{pseudocodeex}%
         {\begin{pseudocex}%
          \textbf{\textrm{Algoritmo}}\hspace{0.7ex}\ignorespaces%
         }%
         {\end{pseudocex}}

\newcommand{\phentao}{\phantom{então}} 
\newcommand{\phsenao}{\phantom{senão}} 
\newcommand{\x}{\hspace*{4ex}} % era 3ex
\newcommand{\xx}{\hspace*{8ex}}
\newcommand{\xxx}{\hspace*{12ex}}
\newcommand{\xxxx}{\hspace*{16ex}}
\newcommand{\xxxxx}{\hspace*{20ex}}
\newcommand{\xxxxxx}{\hspace*{24ex}}
\newlength{\tsts}
\settowidth{\tsts}{9}
\newcommand{\ts}{\hspace{\tsts}}
\newcommand{\RM}{\textrm}
% Example:
% \begin{pseudocode}
% \max $(A, p, r)$
%    
% \RM{\ts1}\x se $p = r$
% 
% \RM{\ts2}\xx então devolva $A[p]$
%    
% \RM{\ts3}\xx senão $m\leftarrow \textsc{Máximo}\,(A,p+1,r)$
% 
% \RM{\ts9}\xx\phsenao se $m < A[p]$ \quad $\rhd$ comentário
% 
% \RM{10}\xxx\phsenao então $m\leftarrow A[p]$
%  
% \RM{11}\xx\phsenao devolva $m$
% \end{pseudocode}s



\renewcommand{\thesubfigure}{\Alph{subfigure}}


%Necessário para forçar o conclusão estar fora das partes

% Redefinition of \cleardoublepage. Even pages are left completely blank.
\let\origdoublepage\cleardoublepage
\renewcommand{\cleardoublepage}{%
  \clearpage{\pagestyle{empty}\origdoublepage}}

% ---------------------------------------------------------------------------- %
% Alterando o sumário
%\renewcommand{\cftdotsep}{\cftnodots}

%\cftpagenumbersoff{part}
%\renewcommand{\cftpartpresnum}{\sffamily\large\bfseries\color{blue}\partname\ }
%\renewcommand{\cftpartaftersnumb}{\\}
%\cftsetindents{part}{0em}{0em}

%\renewcommand{\cftchapterleader}{\hskip1cm}
%\renewcommand{\cftchapterafterpnum}{\cftparfillskip}

%\renewcommand{\cftsectionleader}{\hskip1cm}
%\renewcommand{\cftsectionafterpnum}{\cftparfillskip}
% ---------------------------------------------------------------------------- %

% ---------------------------------------------------------------------------- %
% Corpo do texto
\begin{document}
\frontmatter 
% cabeçalho para as páginas das seções anteriores ao capítulo 1 (frontmatter)
\fancyhead[RO]{{\footnotesize\rightmark}\hspace{2em}\thepage}
\setcounter{tocdepth}{2}
\fancyhead[LE]{\thepage\hspace{2em}\footnotesize{\leftmark}}
\fancyhead[RE,LO]{}
\fancyhead[RO]{{\footnotesize\rightmark}\hspace{2em}\thepage}

\onehalfspacing  % espaçamento

% ---------------------------------------------------------------------------- %
% CAPA
% Nota: O título para as teses/dissertações do IME-USP devem caber em um 
% orifício de 10,7cm de largura x 6,0cm de altura que há na capa fornecida pela SPG.
\thispagestyle{empty}
\begin{center}
    \vspace*{2.3cm}
    \textbf{\Large{Caminhos mínimos com\\
                   recursos limitados}}\\
    
    \vspace*{1.2cm}
    \Large{Joel Silva Uchoa}
    
    \vskip 2cm
    \textsc{
    Dissertação apresentada\\[-0.25cm] 
    ao\\[-0.25cm]
    Instituto de Matemática e Estatística\\[-0.25cm]
    da\\[-0.25cm]
    Universidade de São Paulo\\[-0.25cm]
    para\\[-0.25cm]
    obtenção do título\\[-0.25cm]
    de\\[-0.25cm]
    Mestre}
    
    \vskip 1.5cm
    Programa: Ciência da Computação\\
    Orientador: Prof. Dr. Carlos Eduardo Ferreira\\

   	\vskip 1cm
    \normalsize{Durante o desenvolvimento deste trabalho o autor recebeu auxílio
    financeiro da CAPES}
    
    \vskip 0.5cm
    \normalsize{São Paulo, agosto de 2012}
\end{center}

% ---------------------------------------------------------------------------- %
% Página de rosto (SÓ PARA A VERSÃO DEPOSITADA - ANTES DA DEFESA)
% Resolução CoPGr 5890 (20/12/2010)
%
% Nota: O título para as teses/dissertações do IME-USP devem caber em um 
% orifício de 10,7cm de largura x 6,0cm de altura que há na capa fornecida pela SPG.

\newpage
\thispagestyle{empty}
   \begin{center}
        \vspace*{2.3 cm}
        \textbf{\Large{Caminhos mínimos com\\
                       recursos limitados}}\\
        \vspace*{2 cm}
    \end{center}

    \vskip 2cm

    \begin{flushright}
	Esta dissertação trata-se da versão original\\
  do aluno (Joel Silva Uchoa).
    \end{flushright}

\pagebreak


% ---------------------------------------------------------------------------- %
% Página de rosto (SÓ PARA A VERSÃO CORRIGIDA - APÓS DEFESA)
% Resolução CoPGr 5890 (20/12/2010)
%
% Nota: O título para as teses/dissertações do IME-USP devem caber em um 
% orifício de 10,7cm de largura x 6,0cm de altura que há na capa fornecida pela SPG.
%%%%\newpage
%%%%\thispagestyle{empty}
%%%%    \begin{center}
%%%%        \vspace*{2.3 cm}
%%%%        \textbf{\Large{Representação retangular de grafos planares}}\\
%%%%        \vspace*{2 cm}
%%%%    \end{center}

%%%%    \vskip 2cm

%%%%    \begin{flushright}
%%%%	Esta tese/dissertação contém as correções e	alterações\\
%%%%	sugeridas pela Comissão Julgadora durante a defesa\\
%%%%	realizada por (Guilherme Puglia Assunção) em 14/12/2010.\\
%%%%	O original encontra-se disponível no Instituto de\\ 
%%%%	Matemática e Estatística da Universidade de São Paulo.

%%%%    \vskip 2cm

%%%%    \end{flushright}
%%%%    \vskip 4.2cm

%%%%    \begin{quote}
%%%%    \noindent Comissão Julgadora:
    
%%%%    \begin{itemize}
%%%%		\item Profa. Dra. Nome Completo (orientadora) - IME-USP [sem ponto final]
%%%%		\item Prof. Dr. Nome Completo - IME-USP [sem ponto final]
%%%%		\item Prof. Dr. Nome Completo - IMPA [sem ponto final]
%%%%    \end{itemize}
      
%%%%    \end{quote}
%%%%\pagebreak


\pagenumbering{roman}     % começamos a numerar 

% ---------------------------------------------------------------------------- %
% Agradecimentos
\chapter*{Agradecimentos}

Primeiramente, agradeço a Deus pela oportunidade de desenvolver este trabalho
e de encontrar pessoas nesta jornada que me fazem crescer tanto intelectualmente
como moralmente.


% ---------------------------------------------------------------------------- %
% Resumo
\chapter*{Resumo}

\begin{center}
  \Large{Caminhos mínimos com recursos limitados}
\end{center}

O problema de escolher uma rota para se fazer uma viagem, tal que a rota 
minimize a distância do percurso é um problema fundamental em 
computação. Nesta forma básica, esse problema é o problema de caminho 
mínimo em grafos onde os arcos são possíveis trechos, valorados por seu 
comprimento. Algumas vezes um caminho mínimo desta forma é bom, outras 
vezes não. Existem ocasiões, onde tal caminho possui propriedades 
indesejáveis. Por exemplo, alguns trechos podem ter tráfego denso e nos 
fazer perder muito tempo na travessia, ou existem muitos pedágios com 
taxas que, acumuladas pelo caminho, vão exceder o dinheiro que temos 
disponível.  Isso nos leva a considerar um ou mais parâmetros adicionais 
para a escolha do caminho.  Os casos mais comuns de parâmetros a 
considerar envolvem o consumo de recursos em um orçamento que limita a 
quantidade disponível desses recursos. Um caminho mínimo com essas 
limitações adicionais é chamado de {\bf caminho mínimo com recursos 
limitados } ({\it resource constrained shortest path -- \textsc{RCSP}}). 

Este trabalho possui dois objetivos principais: apresentar um histórico 
bibliográfico do problema \textsc{RCSP}, tendo como foco algoritmos exatos para 
o caso onde possuímos um único recurso; e implementar e comparar os 
principais algoritmos conhecidos, observando-os em situações práticas.

\noindent \textbf{Palavras-chave:} Otimização combinatória, caminhos 
mínimos com restrições.

% ---------------------------------------------------------------------------- 
%  %
\chapter*{Abstract}

\begin{center}
  \Large{Resource constrained shortest path}
\end{center}



The problem of choosing a route to a trip, where we want
minimize the distance of the path is a major problem in
computing. In this basic form, this is the shortest path problem. But 
sometimes, besides the length we need to consider more  
parameters for selecting a good path. A common parameters
to consider is the consumption of resources in a limited budget. A 
shortest path with these additional constraints is called resource 
constrained shortest path - \textsc{RCSP}.

This paper has two main objectives: to present a literature review of 
the problem \textsc{RCSP}, focusing on exact algorithms for
the case where we have a single resource, and implement and compare
some algorithms, observing them in practical situations.

\noindent \textbf{Keywords:} Combinatorial Optimization, shortest path 
with constraints.

% ---------------------------------------------------------------------------- 
%  %
% Sumário
\pdfbookmark[-1]{\contentsname}{toc}
\tableofcontents    % imprime o sumário

% ---------------------------------------------------------------------------- 
%  %
%%%%\chapter{Lista de Abreviaturas}
%%%%\begin{tabular}{ll}
%         CFT         & Transformada contínua de Fourier 
%         (\emph{Continuous Fourier Transform})\\
%         DFT         & Transformada discreta de Fourier (\emph{Discrete 
%         Fourier Transform})\\
%        EIIP         & Potencial de interação elétron-íon 
%        (\emph{Electron-Ion Interaction Potentials})\\
%        STFT         & Tranformada de Fourier de tempo reduzido 
%        (\emph{Short-Time Fourier Transform})\\
%%%%	xxx & xxx (\emph{english(xxx)})\\
%%%%\end{tabular}

% ---------------------------------------------------------------------------- 
%  %
%%%%\chapter{Lista de Símbolos}
%%%%\begin{tabular}{ll}
%%%%        $\omega$    & Frequência angular\\
%%%%        $\psi$      & Função de análise \emph{wavelet}\\
%%%%        $\Psi$      & Transformada de Fourier de $\psi$\\
%%%%end{tabular}

% ---------------------------------------------------------------------------- 
%  %
% Listas de figuras e tabelas criadas automaticamente
%%%%\listoffigures            
%%%%\listoftables            

% ---------------------------------------------------------------------------- 
%  %
% Capítulos do trabalho
%\mainmatter

% cabeçalho para as páginas de todos os capítulos
%\fancyhead[RE,LO]{\thesection}

%\singlespacing\ch              % espaçamento simples
%\onehalfspacing            % espaçamento um e meio

%\cleardoublepage\pdfbookmark[-1]{Introduction}{introduction}
%\input cap-introducao

%\input cap-spp

%\input cap-rcsp/cap 

%\input cap-experimentos



%\cleardoublepage\pagestyle{empty}\mbox{}\cleardoublepage
%\addtocontents{toc}{\protect\addvspace{2.25em}}
%\cleardoublepage\pdfbookmark[-1]{Conclusions}{conclusions}
%\input cap-conclusoes

%\begin{comment}
% cabeçalho para os apêndices
\renewcommand{\chaptermark}[1]{\markboth{\MakeUppercase{\appendixname\ \thechapter}} {\MakeUppercase{#1}} }
%\fancyhead[RE,LO]{}
%\fancyhead[RE,LO]{\thesection}
%
%\appendix
%\input{./Chapters/Apendice/Ape1}
%\end{comment}

% ---------------------------------------------------------------------------- 
%  %
% Bibliografia
%\cleardoublepage\pdfbookmark[-1]{\bibname}{bibliography}
%\phantomsection\addcontentsline{toc}{chapter}{\bibname}
%\bibliographystyle{plainnat-ime} % citação bibliográfica textual
%\bibliography{bibliografia}  % associado ao arquivo: 'bibliografia.bib'

\end{document}
