% -------------------------------------------------------------------- %
\chapter*{Resumo}

Um problema bastante conhecido � o de escolher uma rota para se fazer 
uma viagem, tal que a rota minimize a dist�ncia do percurso. Nesta forma 
b�sica, esse problema � o problema de caminho m�nimo em grafos onde as 
arestas s�o poss�veis trechos, valorados por seu comprimento. Algumas 
vezes um caminho m�nimo desta forma � bom, outras vezes n�o. Existem 
ocasi�es, onde tal caminho possui propriedades indesej�veis. Por 
exemplo, alguns trechos podem ter tr�fego denso e nos fazer perder muito 
tempo na travessia, ou existem muitos ped�gios com taxas que, acumuladas 
pelo caminho, v�o exceder o dinheiro que temos dispon�vel.  Isso nos 
leva a considerar um ou mais par�metros adicionais para a escolha do 
caminho. Os casos mais comuns de par�metros a considerar envolvem o 
consumo de recursos em um or�amento que limita a quantidade dispon�vel 
desses recursos. Um caminho m�nimo com essas limita��es adicionais � 
chamado de {\bf caminho m�nimo com recursos limitados } ({\it resource 
constrained shortest path - \rcsp}). 

Este trabalho possui dois objetivos principais: apresentar um hist�rico 
bibliogr�fico do problema \rcsp, tendo como foco algoritmos exatos para 
o caso onde possu�mos um �nico recurso; e implementar e comparar os 
principais algoritmos conhecidos, observando-os em situa��es pr�ticas.

\noindent \textbf{Palavras-chave:} caminhos m�nimos com recursos 
limitados.
