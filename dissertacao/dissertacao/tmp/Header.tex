% Pacotes 
\usepackage[T1]{fontenc}
\usepackage[brazil]{babel}
\usepackage[utf8]{inputenc}
\usepackage{amsthm}
\usepackage{amsfonts}
\usepackage{graphicx}
\usepackage{setspace}                   % espa�amento flex�vel
\usepackage{indentfirst}                % indenta��o do primeiro par�grafo
\usepackage[nottoc]{tocbibind}          % acrescentamos a bibliografia/indice/conteudo no Table of Contents
%\usepackage{courier}                    % usa o Adobe Courier no lugar de Computer Modern Typewriter
\usepackage{type1cm}                    % fontes realmente escal�veis
\usepackage{listings}                   % para formatar c�digo-fonte (ex. em Java)
\usepackage{titletoc}
%\usepackage[bf,small,compact]{titlesec} % cabe�alhos dos t�tulos: menores e compactos
\usepackage[fixlanguage]{babelbib}
\usepackage[font=small,format=plain,labelfont=bf,up,textfont=it,up]{caption}
\usepackage[usenames,svgnames,dvipsnames]{xcolor}

\usepackage[a4paper,top=2.54cm,bottom=2.0cm,left=2.0cm,right=2.54cm]{geometry} % margens
%\usepackage[pdftex,plainpages=false,pdfpagelabels,pagebackref,colorlinks=true,citecolor=black,linkcolor=black,urlcolor=black,filecolor=black,bookmarksopen=true]{hyperref} % links em preto
\usepackage[plainpages=false,pdfpagelabels,pagebackref,colorlinks=true,citecolor=DarkGreen,linkcolor=NavyBlue,urlcolor=DarkRed,filecolor=green,bookmarksopen=true]{hyperref} % links coloridos
\usepackage[all]{hypcap}                % soluciona o problema com o hyperref e capitulos
\fontsize{60}{62}\usefont{OT1}{cmr}{m}{n}{\selectfont}

% Util
\usepackage{enumerate}
\usepackage{verbatim}
%\usepackage[svgnames]{xcolor}

% Figures
\usepackage{float}
\usepackage{subfig}

%\usepackage{ upgreek }

\hyphenpenalty=10000000
\tolerance=10000000
% to avoid annoying message "Underful [h/v]box"
\hbadness=100000
\vbadness=100000

\setlength\parskip{0.3cm}

%-------------------------------------------------------%
% Chapter marks and page headers
\usepackage{fancyhdr}
\usepackage[avantgarde]{quotchap2}


% ---------------------------------------------------------------------------- %
% Cabe�alhos similares ao TAOCP de Donald E. Knuth
\usepackage{fancyhdr}
\pagestyle{fancy}
\fancyhf{}
\renewcommand{\chaptermark}[1]{\markboth{\rm\textsc{#1}}{}}
\renewcommand{\sectionmark}[1]{\markright{\rm\textsc{#1}}{}}
\renewcommand{\headrulewidth}{0.4pt}
\renewcommand{\footrulewidth}{0pt}

\usepackage{titlesec}
\titleformat{\section}
{\normalfont\Large\scshape\bfseries}{\thesection}{1em}{}
\titleformat{\subsection}
{\normalfont\large\scshape\bfseries}{\thesubsection}{1em}{}
\titleformat{\subsubsection}
{\normalfont\normalsize\scshape\bfseries}{\thesubsubsection}{1em}{}
\titleformat{\paragraph}[runin]
{\normalfont\normalsize\scshape\bfseries}{\theparagraph}{1em}{}
\titleformat{\subparagraph}[runin]
{\normalfont\normalsize\scshape\bfseries}{\thesubparagraph}{1em}{}
\titlespacing*{\chapter}
{0pt}{50pt}{40pt}
\titlespacing*{\section}
{0pt}{3.5ex plus 1ex minus .2ex}{2.3ex plus .2ex}
\titlespacing*{\subsection}
{0pt}{3.25ex plus 1ex minus .2ex}{1.5ex plus .2ex}
\titlespacing*{\subsubsection}{0pt}{3.25ex plus 1ex minus .2ex}{1.5ex plus .2ex}
\titlespacing*{\paragraph}
{0pt}{3.25ex plus 1ex minus .2ex}{1em}
\titlespacing*{\subparagraph} {\parindent}{3.25ex plus 1ex minus .2ex}{1em}

