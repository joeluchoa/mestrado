% -------------------------------------------------------------------- %
\chapter{Experimentos}
\label{cap:experimentos}

Nós iremos executar uma comparação experimental de diferentes métodos 
propostos para o o problema de caminhos mínimos com um único recurso.  
Desde que a performance dos algoritmos é variável para diferentes tipos 
de grafos, nós iremos exeperimentá-los com dados reais e randomicos 
usnado os seguintes tipos de grafos: grafos em grade, grafos de ruas, e 
grafos de curva de aproximação.

Todas as nossas experiencias foram executadas em um laptop Sony Vaio...
Rodando Ubuntu 12.04... Os códigos foram implementados usando a 
linguagem de programação C++... e compilados com GNU g++...

\section{Dados de Teste}

Nós usamos os seguintes três tipos grafos:

\defi{DEM}: Modelos digitais de elevação (\emph{digital elevation 
models}) são grafos em forma de grade onde cada vértice tem um valor de 
altura associado. Nós usamos exemplos de modelos de elevações da Europa 
sedidos por Mark Ziegelmann (\cite{mark:2001}). 

Nossos exemplos \textsc{DEM} são bi-direcionads, ou seja, $m ~ 4n$.  Nós 
usamos o valor absoluto para as diferenças de altura dós vértices como 
função custo, esses valores estão no intervalor $\[0,600\]$. Nós usamos 
inteiros aleatórios dentro do intervalo $\[10,20\]$ como consumo de 
recursos. Nós estamos interessados em minimizar a diferença de altura 
acumulada no caminho com limitação no comprimento do caminho.

\defi{ROAD}: Nós usamos grafos de ruas dos Estatudos Unidos fornecidos 
por Mark Ziegelmann. Os arcos modelando ruas sáo novamente 
bi-direcionados, e a estrutura nos dá $m ~ 2.5m$. Nós usamos 
congestionamento como função de custo. Nós definimos os congestionamento 
como inteiros entre $\[0,100\]$. Nós usamos a distancia euclidiana entre 
os pontos finais dos arcos como função recurso. Estas distancias são 
núemros de ponto flutuante no intervalo $\[\0,7]$. Nós estamos 
interessados em minimizar o congestionamento sujeito a um comprimento 
limidado.

\defi{CURVE}: Nos problema de curvas de aproximação nós requemos 
aproximar uma piecewise linear curve por uma nova curva com menos pontos 
de quebra. Esto é muito importante problemas de compressão de dados em 
áreas como cartografia, computação gráfica, e processamento de sinais 
\citep{dahl:00, nygaard:00} tem estudado como modeloar este problema 
como uma \rcsp.

Assumindo que os pontos de quebra na curva dada ocorrem na ordem $v_1, 
v_2, \cdots, v_n$, nós usamos os pontos de quebra como vértices e 
adicionamos arcos $v_iv_j$ para cada $i < j$. O custo dos arcos é 
atribuido como um erro de aproximação que é introduzido por tomar o 
atalho ao invés da curva original. 

Nós usamos um teste de sinal randomico como função linear. Nós assumimos 
que um pode alcançar os $20$ próximos pontos, assim $m ~ 20n$. Nós 
usamos a \emph{$l_1$-metric} para computar os erros de aproximação e 
obtemos valores reais no intervalo $[0,300]$ como custos.

% Compressão de curvas
%Como temos consumo de recursos idênticos e $k \leq n$, o problema não é 
%mais $\mathcal{NP}$-difícil, tendo em vista que o algoritmo por 
%programação dinâmica tem complexidade de templo de $O(km) = O(n^3)$.

