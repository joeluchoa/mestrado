\newcommand{\e}{\textrm{~e~}}

%Conceitos gerais
\newcommand{\FaceExt}{f_{\infty}}

%XinHe
\newcommand{\FaceEsq}[1]{face_{esq}(#1)}
\newcommand{\FaceDir}[1]{face_{dir}(#1)}

\newcommand{\FaceAcima}[1]{face_{acima}(#1)}
\newcommand{\FaceAbaixo}[1]{face_{abaixo}(#1)}

\newcommand{\snRede}{\textit{S,N}\textrm{-rede~}}
\newcommand{\olRede}{\textit{O,L}\textrm{-rede~}}

\newcommand{\stgrafo}{\textit{s,t}\textrm{-grafo~}}
\newcommand{\stgrafos}{\textit{s,t}\textrm{-grafos~}}

\newcommand{\stgrafosN}{\textit{s,t}\textrm{-grafos}}

\newcommand{\FaceExtEsq}{f^{esq}_{\infty}}
\newcommand{\FaceExtDir}{f^{dir}_{\infty}}

\newcommand{\FaceExtAcima}{f^{acima}_{\infty}}
\newcommand{\FaceExtAbaixo}{f^{abaixo}_{\infty}}

%--------------------------------------

\newtheorem{problema}{Problema}
%\newtheorem{definicao}{Definição}[chapter]
%\newtheorem{teorema}{Teorema}[chapter]
%\newtheorem{conjectura}[teorema]{Conjectura}
%\newtheorem{proposicao}[teorema]{Proposição}
%\newtheorem{lema}[teorema]{Lema}

%\newcommand{\qed}{$\Box$}
\newenvironment{baretheorem}%
         {\begin{list}%
                {}%
                {%
                 %\addtolength{\partopsep}{0.5ex}%
                 \setlength{\partopsep}{1ex}% igual em prova
%                \addtolength{\topsep}{-0.5ex}%
                 \setlength{\topsep}{0.8ex}% igual prova
                 \setlength{\leftmargin}{\parindent}%
                 \setlength{\rightmargin}{\parindent}%
                 \setlength{\itemindent}{0ex}%
                 \setlength{\labelsep}{\itemindent}%
                 \setlength{\labelwidth}{0ex}%
                 \setlength{\parsep}{1ex}% parskip is 1.5ex
                 \setlength{\listparindent}{0ex}%
                }%
          \item%
         }%
         {\end{list}%
         }
\newcounter{theoremnumber}[chapter]
\renewcommand{\thetheoremnumber}{\thechapter.\arabic{theoremnumber}}
\newenvironment{taggednumberedtheorem}[2]%
               {%
                \refstepcounter{theoremnumber}%
                \begin{baretheorem}%
                \textbf{#1\hspace{0.7ex}\thetheoremnumber}%
                \ifthenelse{\equal{#2}{}}% Lamport p.195
                        {\hspace{1pt}:\hspace{1ex}\ignorespaces}%
                        {\hspace{1ex}\textup{(#2)}:\hspace{1ex}\ignorespaces}%
                \slshape%
               }%
               {\end{baretheorem}%
               }

\newcounter{definitionnumber}[chapter]
\renewcommand{\thedefinitionnumber}{\thechapter.\arabic{definitionnumber}}
\newenvironment{taggednumbereddefinition}[2]%
               {%
                \refstepcounter{definitionnumber}%
                \begin{baretheorem}%
                \textbf{#1\hspace{0.7ex}\thedefinitionnumber}%
                \ifthenelse{\equal{#2}{}}% Lamport p.195
                        {\hspace{1pt}:\hspace{1ex}\ignorespaces}%
                        {\hspace{1ex}\textup{(#2)}:\hspace{1ex}\ignorespaces}%
                \slshape%
               }%
               {\end{baretheorem}%
               }


\newenvironment{lema}[1]%
         {\begin{taggednumberedtheorem}{Lema}{#1}%
         }%
         {\end{taggednumberedtheorem}%
         }
\newenvironment{teorema}[1]%
         {\begin{taggednumberedtheorem}{Teorema}{#1}%
         }%
         {\end{taggednumberedtheorem}%
         }
\newenvironment{fato}[1]%
         {\begin{taggednumberedtheorem}{Fato}{#1}%
         }%
         {\end{taggednumberedtheorem}%
         }
\newenvironment{definicao}[1]%
         {\begin{taggednumbereddefinition}{Definição}{#1}%
         }%
         {\end{taggednumbereddefinition}%
         }
\newenvironment{corolario}[1]%
         {\begin{taggednumberedtheorem}{Corolário}{#1}%
         }%
         {\end{taggednumberedtheorem}%
         }

\newenvironment{bareproof}%
         {\begin{list}%
                {}%
                {%
                 \setlength{\partopsep}{1ex}% como em baretheorem
                 \setlength{\topsep}{0.8ex}% como em baretheorem
                 \setlength{\leftmargin}{0ex}%
                 \setlength{\labelwidth}{0ex}% 
                 \setlength{\labelsep}{1ex}%
                 \setlength{\itemindent}{\parindent}% 
                 \addtolength{\itemindent}{\labelsep}%
                 \setlength{\parsep}{1ex}% parskip is 1.5ex
                 \setlength{\listparindent}{\parindent}%
                }%
         }%
         {\end{list}}
\newenvironment{prova}[1]
         {\begin{bareproof}%
          \item[Demonstração:]%
            #1%
         }%
         {\hfill\qed\ignorespacesafterend
          \end{bareproof}%
         }

\newenvironment{exemplo}[1]%
         {\begin{taggednumberedtheorem}{Exemplo}{#1}%
         }%
         {\end{taggednumberedtheorem}%
         }

%\renewcommand{\thetheorem}{\thechapter.\arabic{theorem}}

\def\({\left(}
\def\){\right)}  
\def\<{\langle}
\def\>{\rangle}
\let\:=\colon

% MATHEMATICS
%%%%%%%%%%%%%
  \newcommand{\binomial}% to be used in displaystyle only
             [2]%
             {\big(\!\begin{array}{c}%
                 #1\\[-0.5ex]%
                 #2%
                 \end{array}%
               \!\big)}
  \newcommand{\floor}[1]{\lfloor #1 \rfloor}
  \newcommand{\ceil}[1]{\lceil #1 \rceil}
  \newcommand{\seq}[1]{( #1 )}
  \newcommand{\tam}[1]{\langle #1 \rangle}
% número total de dígitos
  \newcommand{\conjunto}[1]{\{\,#1\,\}}
  \newcommand{\conj}[1]{\{#1\}}
  \newcommand{\twodots}{\,.\,.\,} % in place of \ldots
  \newcommand{\card}[1]{{\mid}#1{\mid}} 
% $\card{X}$ é o mesmo que $|X|$
  \newcommand{\dcard}[1]{{\parallel}#1{\parallel}} 
% $\dcard{X}$ é o mesmo que $||X||$
%\renewcommand{\emptyset}{\mbox{\textup{\O}}} % slash O
%\renewcommand{\emptyset}{\mbox{\textit{\O}}} % slash O
  \newcommand{\eps}{\varepsilon}
% \newcommand{\qed}{\mbox{\hspace{0.2ex}\rule{0.7ex}{0.9ex}}}
%  \newcommand{\qed}{$\Box$}
% \newcommand{\Cdot}{\cdot}
% \newcommand{\Cdot}{}
  \newcommand{\Cdot}{\hspace{1pt}} % produto de vetores $w\Cdot x$
  %\newcommand{\Ddot}{\hspace{0.8pt}} % produto mat por vet e vet por matr
  \newcommand{\Edot}{\hspace{0.8pt}} % produto de matrizes $A\Edot B$
% \newcommand{\dotplus}{\stackrel{.}{+}}
% soma disjunta de grafo com conjunto de arestas $G\dotplus F$
  \newcommand{\dotplus}{+}
  \newcommand{\dotcup}{\stackrel{.}{\cup}}
%% união disjunta de dois conjuntos
%\newcommand{\wt}{\widetilde} % matriz transposta
%\newcommand{\wt}[1]{#1^{\!\top}} % matriz transposta
 \newcommand{\wt}[1]{#1^{\scriptscriptstyle\top}} % matriz transposta
%\newcommand{\tq}{\ :\ } % tal que (no primal-dual)
 \newcommand{\tq}{:} % tal que (no primal-dual)

\usepackage{enumitem}
%usage: \defProblema{label}{nome}{entrada}{saida}

%\newenvironment{mydescription}{%
%   \renewcommand*\descriptionlabel[1]{\hspace\labelsep\rm\textsc{#1}:~}%
%   \begin{description}%
%}{%
%   \end{description}%
%}

\newcommand\descriptionDefProblabel[1]{\hspace{\labelsep}\rm\textsc{#1}:~}
\newenvironment{descriptionDefProb}{%
   \let\descriptionlabel\descriptionDefProblabel
   \description
}{%
   \enddescription
}

\newcommand{\defProblem}[4]{%
	%\parindent=0cm%
	\begin{problema}%
		\label{#1}%
		\hspace{.1cm}%
		\textsc{#2}%
 		\begin{descriptionDefProb}[style=multiline,leftmargin=3.5cm, labelindent=1cm, topsep=-0.1ex, itemsep=0.1ex]%
				\item[Entrada] #3%
				\item[Saida]  #4%
		\end{descriptionDefProb}%
	\end{problema}%
}
