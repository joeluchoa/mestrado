% -------------------------------------------------------------------- %
\chapter*{Abstract}

\begin{center}
  \Large{Resource constrained shortest path}
\end{center}



The problem of choosing a route to a trip, where we want
minimize the distance of the path is a major problem in
computing. In this basic form, this is the shortest path problem. But 
sometimes, besides the length we need to consider more  
parameters for selecting a good path. A common parameters
to consider is the consumption of resources in a limited budget. A 
shortest path with these additional constraints is called resource 
constrained shortest path - \textsc{RCSP}.

This paper has two main objectives: to present a literature review of 
the problem \textsc{RCSP}, focusing on exact algorithms for
the case where we have a single resource, and implement and compare
some algorithms, observing them in practical situations.



The Shortest Path (\spp) problem is among the fundamental problems of 
computer
science. It's been deeply studied and subject of many publications. Also, many
efficient solutions (polynomial time algorithms) are known for this
problem. The
SP is widely applied in many fields of science, not only computer
science. These
situations usually need to transport a load between two or more specific spots
of a network. This action must be taken optimally regarding to some criterion,
for instance the least cost, or the least time or maximum reliability.

While new solutions for SP were presented, new demands were issued
too, with new
variations for the problem. One of these variations comes from the
fact that, in
a real scenario, a combination of many criteria must be optimized, for instance
a path with least cost and least time. This problem is known as Multiobjective
Shortest Path. Since it's not possible to optimize all criteria at once, one of
them is chosen to represent the cost function to be minimized and the others to
represent resources with defined boundary. This variation, known as Resource
Constrained Shortest Path (\rcsp), was the object of the present study.

By adding resource constraints, the \spp becomes $\mathcal{NP}$-hard, 
even in acyclic graphs
with only one resource constrained and all resource consumption being positive.
There are reductions from the famous NP-hard problems Knapsack and Partition to
our problem.

In many fields, are found theoretical and practical problems that may be
expressed as a Resource Constrained Shortest Path Problem, which
motivated us to
study this problem in order to summarize enough information to researchers and
developers involved with this problem. This paper presents a detailed
bibliographic revision to \rcsp, focusing on the development of exact 
algorithms
for the case when there is only only one resource and on the implementation and
comparison of the main known algorithms in practical situations.


\noindent \textbf{Keywords:} Combinatorial Optimization, shortest path 
with constraints.

