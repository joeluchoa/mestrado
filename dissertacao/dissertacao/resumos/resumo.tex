\documentclass[10pt,a4paper]{article}
\usepackage[utf8x]{inputenc}
\usepackage{ucs}
\usepackage{amsmath}
\usepackage{amsfonts}
\usepackage{amssymb}
\usepackage{makeidx}
\usepackage[brazil]{babel}
\author{Joel Silva Uchoa}
\title{Resumo da Tese de Doutorado de Mark Ziegelmann}
\begin{document}
\maketitle

\section{Resumo}

O clássico problema de caminhos mínimos, de encontrar um caminho de 
custo mínimo entre dois vértices em um grafo, é eficientemente resolvido 
em tempo polinomial. Contudo, em muitas aplicações nós também temos 
adicionalmente um orçamento ou restrições de recursos no caminho. Estes 
problema é conhecido como caminhos mínimos com restrições e infelizmente 
pertence a classe de problemas ``difíceis'', para os quais não se 
conhece algoritmo com complexidade de tempo polinomial. Nesta tese, nós 
propomos um método de duas fases para o problema de caminhos mínimos com 
restrições. Nós primeiramente resolvemos uma relaxação para obter um 
limite superior e inferior, e depois eliminamos a folga com um algoritmo 
esperto de ranqueamento de caminhos para obter a solução exata. Nós 
comparamos métodos antigos e novos, tanto teoricamente como 
experimentalmente.

\section{Capítulo 1 - Introdução}

O problema de caminhas mínimos é um dos problemas fundamentais em 
Ciência da Computação. Ele tem sido muito estudado e algoritmos 
polinomiais eficientes são conhecidos. O problema de caminhos mínimos 
surge, com muita frequência, na prática, desde que uma grande variedade 
de aplicações querer o envio de algum material (exemplo: um pacote de 
dados, uma camada telefônica ou um veículo) entre dois específicos 
pontos em uma rede, tão rápido, ou barato quanto possível.

Contudo, na prática, muitas vezes não estamos somente interessado no 
caminho mais barato ou rápido, e sim estamos interessados em uma 
combinação de diferentes critérios, exemplo, nós queremos ter um caminho 
que é tanto barato quanto rápido. Isto é conhecido como o bi-critério ou 
multi-critério caminho mínimo. Como otimizar todos os critérios pode não 
ser possível, nós escolhemos uma critério como função de custo que 
queremos minimizar, os outros ficam como funções de consumo de recursos 
e impomos um orçamento para limitar o consumo máximo destes recursos em 
um caminho. O \emph{RSPC} consiste em procurar o caminho de custo mínimo 
entre dois nós cujo o consumo de recursos satisfaz os limites 
estabelecidos.

Em redes de comunicação, nos temos problema de roteamento de qualidade 
de serviço (\emph{QoS}). Nós estamo procurando por um caminho de custo 
mínimo que obedece um dado limite de atraso ou nível de 
segurança/confiabilidade. Isto é novamente um caminho mínimo com 
restrições. Mais informações podem ser encontradas nos artigos de 
\textsc{Orda (1998) e Xue (2000)}.

Existe também outras aplicações que podem ser modeladas como caminhos 
mínimos com restrições:

\textsc{Eliman e Kohler (1997)} mostraram como modelar duas aplicações 
de engenharia como caminhos mínimos com restrições (com múltiplos 
recursos): sequencias ótimas para os processos de tratamento de esgoto e 
composição de paredes energéticas de custo mínimo (??).

\textsc{Dahl e Realfsen (2000) e Nygaard (2000)} estudaram o problema de 
aproximação de curva linear e mostraram como modelar isto como um 
caminho mínimo com restrições (\textsc{ver Sessão 3.5}). Eles procuram 
um caminho que corresponda %TODO

\subsection{Aplicações}

O problema de caminhos mínimos tem um grande número de aplicações 
práticas:

A primeira aplicação que vem a mente é o planejamento de rotas em redes 
de tráfego. Nós queremos ir de $A$ até $B$, por exemplo, e gostaríamos 
de minimizar os trechos congestionados enquanto impomos uma restrição de 
tamanho de percurso no caminho \textsc{Ver figura 1.1}.  
Alternativamente, nós queremos ir de $A$ até $B$ tão rápido quando 
possível mas temos uma restrição de orçamento em relação ao combustível 
ou a pedágios.

\textsc{Figura 1.1 Um mapa com ponto A e B, via congestionadas, o menor 
caminho, e o menor caminho minimizando congestionamento.}

\section{Capítulo 3 - Caminhos Mínimos com Recursos Limitados}

Neste capítulo nós consideramos uma variante do clássico problema de 
caminhos mínimos (\emph{SP}), o problema de caminhos mínimos com 
restrições (sobre recursos) (\emph{CSP}). Diferentemente do problema 
original \emph{SP}, o \emph{CSP} é $\mathcal{NP}$-completo. Como existem 
importante aplicações que podem ser modeladas como uma instância do 
\emph{CSP}, nós estamos interessados em resolver o problema tão 
eficiente quanto possível.

Nós começamos o capítulo com uma discussão sobre trabalhos anteriores.  
Como em problemas muito difíceis, nós vamos primeiros vamos olhar uma 
relaxação de um programa linear que modela o \emph{CSP}. Começando da 
formulação do programa linear inteiro, nós extraímos um método 
combinatório para resolver a relaxação com a combinação de intuição 
geométrica simples e teoria em otimização. No caso de um único recurso 
nossa abordagem é equivalente a métodos previamente propostos, contudo, 
nós somos os primeiros a provar um limite de tempo polinomial para o 
algoritmo. Nós também obtemos o primeiro método combinatório para 
resolver, de forma exata, a relaxação no caso de múltiplos recursos.  
Resolver a relaxação nos dá um limite superior e inferior. Nós iremos 
então rever velhos métodos e apresentar novos métodos para eliminar ao 
máximo a falga da dualidade para obter um método exato com duas fases 
para o \emph{CSP}. Nós fechamos o capítulo com uma detalhada comparação 
experimental dos diferentes métodos.

\subsection{Definição do problema}

O problema de caminhos mínimos com restrições (de recursos) (\emph{CSP}) 
requer a computação de um caminho de menor custo que seja limitado por 
um conjunto de restrições de recursos. Mais precisamente, nós temos um 
grafo $G=(V,E)$ com $|V|=n$ e $|E|=m$, um vértice de origem $s$ e um 
vértice de destino $t$, e $k$ limites de recursos $\lambda^{(1)}$ até 
$\lambda^{(k)}$. Cada aresta $e$ tem um custo $c_e$ e consume 
$r^{(i)}_e$ unidades do recurso $i$, $1 \le i \le k$. Assumimos custos e 
recursos como não negativos. Eles são aditivos ao longo dos caminhos. A 
meta é encontrar um caminho com menor custo de $s$ até $t$ que satisfaça 
as restrições de consumo de recursos.

O caso especial $k=1$ é chamado de caso com recurso simples 
(\emph{single resource case}) que tem sido mais estudado previamente 
(ver Sessão 3.2) e nós iremos considerar este caso caso na Sessão 3.3.2.  
O caso com múltiplos recurso ($k>1$) irá ser discutido na Sessão 3.3.4.

\subsection{Trabalhos anteriores}


Nós damos aqui uma visão geral dos trabalhos anteriores sobre o problema 
de caminhos mínimos com restrições.

\subsection{Complexidade}

Mesmo que o problema de caminhos mínimos (sem restrições), nós iremos 
ver que a introdução de uma única restrição de recurso torna o problema 
$\mathcal{NP}$-completo.

\emph{CSP} é listado como um problema \textsc{ND30} (\emph{shortest 
weight-constrained path}) em \textsc{Garey e Johnson (1979)} e reportado 
como sendo $\mathcal{NP}$-completo  por redução ao problema da 
\textsc{Partição}. \textsc{Hanler e Zang (1980)} fizeram uma 
interessante redução ao bem conhecido problema da \textsc{Moshila} 
(\emph{knapsack problem}) que é muito similar:

Nós temos um conjunto de $n-1$ itens, cada um tem um valor $v_j$ e um 
peso $w_j$, para $j = 1, \cdots, n - 1$. A meta é colocar itens na 
\textsc{Mochila} tal que o limite de peso $\lambda$ não seja excedido e 
o valor dos itens escolhidos seja maximizado. O problema da 
\textsc{Mochila} (\emph{KP}) pode ser modelado como:

\[
  \begin{array}{lrcl}
  \textrm{maximizar}	& \displaystyle\sum_{j=1}^{n-1}{v_jx_j}	&     	&   
  \\
  \textrm{sujeito à}	& \displaystyle\sum_{j=1}^{n-1}{w_jx_j}	& \le 	& 
  \lambda \\
    & x_j \in \{0, 1\}     						&  		& j = 1, \cdots, n - 1 \\
  \end{array}
\]

Onde $v_j$, $w_j$, $\lambda$ são inteiros positivos. Agora nós vamos 
configurar um grafo direcionado de $n$ vértices com dois arcos paralelos 
entre cada de vértices $j$ e $j+1$, para $j = 1, \cdots, n - 1$. Seja 
$c^{(1)}_{j,j+1} = M - v_j$, $r^{(1)}_{j,j+1} = w_j$ e
$c^{(2)}_{j,j+1} = M      $, $r^{(2)}_{j,j+1} = 0  $ os parâmetros do 
primeiro e segundo arco para $j = 1, \cdots, n - 1$, respectivamente, 
onde $M = max\{v_j : j = 1, \cdots, n - 1\}$ (\textsc{ver Figura 3.1}).

\textsc{Figura 3.1}

Então é evidente que \emph{KP} pode ser resolvido encontrando um caminho 
mínimo (em relação ao parâmetro $c$) do vértice $1$ até o vértice $n$, 
sujeito a consumir uma quantidade de recursos (parâmetro $r$) limitada 
por $\lambda$. Como o \emph{KP} é $\mathcal{NP}$-completo, podemos 
deduzir que o \emph{CSP} é também $\mathcal{NP}$-completo.

\subsection{Programação dinâmica recursiva}

O primeiro estudo lidando como o \emph{CSP} com um único recurso foi 
feito por \textsc{Joksch (1966)} que apresentou um algoritmo baseado em 
programação dinâmica (\textsc{ver tambem Lawler (1976)}).

Nós chamamos um caminho de $i$ até $j$ de $r$-caminho se o consumo de 
recursos do $ij$-caminho é menor ou igual a $r$. Nós procuramos o 
$\lambda$-caminho mínimo de $1$ à $n$. Digamos que $c_j(r)$ seja o custo 
do menor $r$-caminho do vértice $i$ até o vértice $j$. Então temos a 
seguinte definição recursiva:

$$ c_j(r) = min\{
c_j(r - 1), \displaystyle\min_{(i,j) \in E, r_{ij} \le r}\{c_i(r - 
r_{ij}) + c_{ij}\}
\} $$

Setando $c_1(r) = 0$ para $1 \le r \le \lambda$ e $c_j(0) = \infty$ para 
$j = 2, \cdots, n$, nós podemos recursivamente computar o valor ótimo 
para o problema \emph{CSP} que é dado por $c_n(\lambda)$.

A complexidade de tempo deste método é $O(m\lambda)$ e o consumo de 
espaço é $O(n\lambda)$. Assim a programação dinâmica dá-nos um 
pseudopolinômial algoritmo para o \emph{CSP}, e também para o problema 
da \textsc{Mochila}.

Como no caso do problema da \textsc{Mochila}, a abordagem por 
programação dinâmica pode também ser usada para obter um \emph{PTAS} 
para o \emph{CSP} utilizando arredondamento e escala, que é discutido na 
próxima sub-sessão.

As abordagens de \emph{labeling} discutidas na Sessão 3.2.4 construídas 
em cima da programação dinâmica recursiva faz uso do fato de que 
$c_j(r)$ é uma função escada (?) (\emph{step function}).

\subsection{$\epsilon$-aproximação}

\emph{CSP} é fracamente $\mathcal{NP}$-completo, já que nós temos uma 
simples formulação pseudo polinomial por programação dinâmica.  
\textsc{Hassin (1992)} aplicou a técnica padrão de \emph{rounding and 
scaling} para obter um \emph{fully polynomial $\epsilon$-approximation 
scheme} (\emph{PTAS}) para o \emph{CSP}. Vamos rever, resumidamente, 
este método agora.

Na sessão anterior nós vimos um procedimento simples baseado em 
programação dinâmica. Para nossos propósitos aqui, um outro algoritmo 
recursivo é mais útil:
Digamos que $g_j(c)$ denota o consumo de recursos de um $1j$-caminho 
mínimo que custa no máximo $c$. Então a seguinte recursão pode ser 
definida:

$$ g_j(c) = min\{
g_j(c - 1), \displaystyle\min_{(i,j) \in E, c_{ij} \le c}\{g_i(c - 
c_{ij}) + r_{ij}\}
\} $$

Setando $g_1(c) = 0$ para $c = 0, \cdots, OPT$ e $g_j(0) = \infty$ para 
$2, \cdots, n$, nós podemos iterativamente computar o valor ótimo para 
nosso problema.

Note que $OPT$ não é um valor conhecido a priori, mas temos que $OPT = 
min \{c | g_n(c) \le \lambda\}$. Assim, $g_j(c)$ deve ser computado 
primeiramente para $c=1$ e $j=1,\cdots,n$, depois para $c=2$, e assim 
sucessivamente até que $g_j(c) \le \lambda$ pela primeira vez, e então 
que setamos $OPT$ com o valor atual de $c$. A complexidade desde 
algoritmo é $O(m OPT)$.

Agora, digamos que $V$ seja um certo valor, e suponha que queremos 
testar se $OPT \ge V$ ou não. Um procedimento polinomial que responde 
essa questão pode ser estendido em um algoritmo polinomial parar 
encontrar $OPT$ simplesmente usando uma uma busca binário. Como nosso 
problema é $\mathcal{NP}$-difícil, temos que nos satisfazer com um teste 
mais fraco.

Tomemos um $\epsilon$ fixo, $0 < \epsilon < 1$. Agora, nós explicamos um 
teste polinomial $\epsilon$-aproximado com as seguintes propriedades: se 
tal teste devolve uma saída positiva, então definitivamente $OPT \ge V$.  
Se ele revolver uma saída negativa, então nós sabemos que $OPT < (V + 
\epsilon)$.

O teste arrendonda o custo $c_{ij}$ das arestas, substituindo seu valor 
por:

$$ \left\lfloor \frac{ c_{ij} }{ \epsilon V / (n-1) } \right\rfloor 
\cdot \frac{\epsilon V}{(n-1)} $$.

Isto diminui todos os custos de arestas em no máximo $\epsilon V / 
(n-1)$, e todos os custos de caminhos em no máximo $\epsilon V$. Agora o 
problema pode ser resolvido aplicando o algoritmo anterior ao grafo com 
os custo das arestas escalados para $\lfloor c_{ij} / (\epsilon V / 
(n-1)) \rfloor$. Os valores de $g_j(c)$ para $j = 2, ..., n$, são 
primeiro computados para $c=1$, depois para $c=1,2,\cdots$ até que
$g_n(c) \le \lambda$ para algum $c=\hat{c}<(n-1)/\epsilon$, ou $c \ge 
(n-1)/\epsilon$.

No primeiro caso, um $\lambda$-caminho de custo de no máximo
$$ \frac{V\epsilon}{n - 1} \hat{c} + V\epsilon < V(1+\epsilon)$$
foi encontrado, e segue que $OPT < V(1+\epsilon)$. No segundo caso, cada 
$\lambda$-caminho tem $c' \ge (n-1)/\epsilon$ ou $c \ge V$, então $OPT 
\ge V$. Assim, o teste funciona como queríamos.

A complexidade de tempo é polinomial cara fixado o $\epsilon$ é 
explicada em seguida: Tomar a parte inteira de um número não negativo no 
intervalo $\{0, \cdots, U\}$ pode ser feito em tempo $O(\lg(U))$ usando 
busca binária. Arrendondar o custo das aresta toma tempo 
$O(mlg(n/\epsilon))$ desde que nós escalamos somente as arestas com 
custo menor que $V$ (o resultado é no máximo $(n-1)/\epsilon)$. Depois 
executamos $O(n\epsilon)$ iterações do algoritmo acima que novamente 
toma tempo $O(|E|\lg(n/\epsilon))$. E essa é também a complexidade do 
procedimento de teste inteiro.

Agora nós usamos este teste para chegar a um \emph{PTAS} baseado em 
escalar e arrendondar: Para aproximar $OPT$ nós primeiramente 
determinamos um limite superior ($UB$) e um limite inferior ($LB$).  O 
limite superior $UB$ pode ser setado como a soma das $n-1$ arestas com 
maior custo, ou o custo da caminho que consome menos recursos. O limite 
inferior $LB$ pode ser setado como $0$ ou o caminho de menor custo.

Se $UB \le (1 + \epsilon)LB$, então $UB$ é uma $\epsilon$-aproximação de 
$OPT$. Suponha que $UB > (1+\epsilon)LB$. Seja $V$ um dado valor $LB < V 
< UB/(1+\epsilon)$. O procedimento \textsc{Teste} pode agora ser 
aplicado para melhorar os limites para $OPT$. Especificamente, ou $LB$ é 
aumentado ou $UB$ é diminuído para $V(1+\epsilon)$. Executando uma 
sequência de testes, a razão $UB/LB$ pode ser reduzida. Uma vez que a 
razão atinga o valor de uma constante predefinida, digamos $2$, então 
uma $\epsilon$-aproximação pode ser obtida aplicando-se o algoritmo por 
programação dinâmica para o grafo com os custo das arestas escalados 
para $\lfloor c_{ij} / (LB/(n-1)) \rfloor$. O erro final é no máximo 
$\epsilon LB < \epsilon OPT$.

A complexidade de tempo para o último passo é $O(|E|n/\epsilon)$. A 
redução da razão $UB/LB$ é melhor alcançada por busca binária no 
intervalo $(LB, UB)$ em escala logarítmica. Depois de cada teste nós 
atualizamos os limites. Para garantir uma rápida redução de razão nós 
executamos o teste com o valor $x$ tal que $UB/x=x/LB$, que é $x = 
(LB\cdot UB)^{1/2}$. O número de testes necessários para reduzir a razão 
para abaixo de $2$ é $O(\lg\lg(UB/LB))$ e cada teste toma tempo 
$O(|E|n/\epsilon)$. \textsc{Hassin (1992)} mostrou como computar um 
valor inteiro próximo a $O(UB\cdot LB)^{1/2}$ em tempo 
$O(\lg\lg(UB/LB))$. Isto dá uma complexidade de tempo, total de 
$O(\lg\lg(UB/LB)(|E|(n\epsilon) + \lg\lg(UB/LB)))$ para este algoritmo 
$\epsilon$-aproximado como um $PTAS$ para o \emph{CSP}.

\textsc{Hassin (1992)} também mostrou uma \emph{PTAS} cuja a 
complexidade depende somente do número de variáveis e $1/\epsilon$ e 
possui complexidade de tempo de $O(|E|(n^2/\epsilon)\lg(n/\epsilon))$.
O melhor $PTAS$ foi obtido por \textsc{Phillips (1993)} que atingiu a 
complexidade de tempo de $O(|E|(n/\epsilon) + 
(n^2/\epsilon)\lg(n/\epsilon))$.


\end{document}
