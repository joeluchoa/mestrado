\direitos{Em sua gradua��o, na UFG -- Universidade Federal de Goi�s, foi monitor das disciplinas de Matem�tica Discreta e Teoria dos Grafos do Instituto de Inform�tica -- INF, foi pesquisador do CNPq em um trabalho de inicia��o cient�fica atrav�s de uma parceria UFG/PUC-Rio, eleito representante dos alunos de Ci�ncia da Computa��o junto ao Conselho Diretor do INF, presidente do Centro Acad�mico de Ci�ncia da Computa��o, integrante da primeira equipe a representar o INF/UFG na Maratona de Programa��o -- competi��o brasileira da \textit{International Collegiate Programming Contest}, organizada pela \textit{Association for Computing Machinery} (ACM-ICPC) -- e da primeira equipe do INF/UFG a se qualificar para uma fase nacional da mesma competi��o, participou da organiza��o de dois ETI's -- Encontro de Tecnologia e Inform�tica -- na UFG e realizou interc�mbio estudantil pelo per�odo de um ano na Universidade do Algarve, Portugal, ao abrigo do Programa Luso-Brasileiro Santander Universidade.}

