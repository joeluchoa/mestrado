%==========================
% O Problema do Caminho M�nimo com Restri��o de Subcaminhos
%==========================
O Problema do Caminho M�nimo com Restri��o de Subcaminhos (SPPFP -- \textit{Shortest Path Problem with Forbidden Paths}) � uma variante do Problema do Caminho M�nimo na qual deseja-se encontrar um caminho de menor custo entre uma origem e todos os destinos poss�veis, sem que algum subcaminho proibido seja percorrido durante o processo. 

Formalmente, dado um grafo direcionado e aresta valorado $G(V,A)$, onde $V$ e $A$ s�o, respectivamente, o conjunto de v�rtices e o conjunto de arcos pertencentes a $G$, dados ainda o v�rtice $s \in V$ como v�rtice origem,  $c : A \rightarrow \sR$ como a fun��o de custo dos arcos, e um conjunto de caminhos proibidos $F = \{u \leadsto v, w \leadsto z, \ldots \}$, $u,v,w,z \in V$, encontre um caminho mais curto $P$ de $s$ a todos os outros v�rtices $t \in V$, sob a restri��o de que nenhum caminho $f \in F$ seja um subcaminho de $P$ \cite{Villeneuve:2005}.

Em \cite{Szeider:2003} � provado que o problema em encontrar um caminho m�nimo simples com restri��o de subcaminhos � $\mathcal{NP}$-Completo para determinadas classes de grafos, mesmo considerando caminhos proibidos pequenos, com apenas duas arestas \cite{Ahmed:2009}. A prova utiliza uma redu��o do 3-SAT, problema este apresentado por \cite{Garey:1979} como um problema $\mathcal{NP}$-Completo.
% ===MOSTRAR QUE EH NP PRA ALGUMAS CLASSES, NAO PARA TODAS===