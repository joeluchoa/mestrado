% ================================
% Problema do Caminho Evitando Pares Proibidos
% ================================

O problema surgiu com o nome de Problema do Caminho com Restri��o de Pares Imposs�veis (IPCPP -- \textit{Impossible Pairs Constrained Path Problem}) e sua aplica��o era voltada para valida��o e teste automatizados de \textit{software}. A ideia era tornar os testes mais eficientes considerando apenas aqueles caminhos de um programa que continham no m�ximo um ramo de cada par de ramos mutuamente n�o-execut�veis \cite{Kolman:2009}. Adicionalmente, era poss�vel criar um conjunto de testes para percorrer todos os ramos de um programa, testando todas as suas sa�das e validando assim seu funcionamento.

O Problema do Caminho Evitando Pares Proibidos (PAFPP -- \textit{Path Avoiding Forbidden Pairs Problem}) � um problema no qual deseja-se encontrar, em um grafo direcionado, um caminho que inicie em um v�rtice origem e termine em um v�rtice destino, de tal forma que nenhum dos pares de v�rtices proibidos esteja presente neste caminho escolhido. Vale ressaltar que um caminho $P$ pode conter um v�rtice de um par proibido $f$, mas nunca os dois. Al�m disso, um v�rtice $v$ selecionado em um par proibido � marcado como utilizado tamb�m em todos os outros pares proibidos em que se mostra presente. Assim, tomemos os pares  $f_1 = (v,w)$ e $f_2 = (v,z)$ como pares proibidos, e um caminho $P$  contendo o v�rtice $w$ de $f_1$. Agora, olhando para $f_2$, $P$ n�o pode conter $v$, uma vez que a utiliza��o deste v�rtice de $f_2$ implica sua utiliza��o tamb�m em $f_1$ e o consequente aparecimento de $f_1$ em $P$. Entretanto, caso o v�rtice $v$ n�o esteja em $P$, os v�rtices $w$ e $z$ podem aparecer em $P$ sem nenhum problema.

Seguindo a defini��o de \cite{Kolman:2009}, dado um grafo $G(V,A)$, onde $V$ � o conjunto de v�rtices e $A$ o conjunto de arcos, um v�rtice origem $s$, um v�rtice destino $t$, tal que $s,t \in V$, e um conjunto de pares de v�rtices proibidos $F =\{(a,b), (c,d), \ldots \}$, tal que $F \subset (VxV)$, o problema consiste em encontrar um caminho $s \leadsto t$ que contenha no m�ximo um v�rtice de cada par em $F$, ou reconhe�a que tal caminho n�o existe.

Em \cite{Gabow:1976} foi provado que o problema � $\mathcal{NP}$-Completo em grafos direcionados ac�clicos.

 