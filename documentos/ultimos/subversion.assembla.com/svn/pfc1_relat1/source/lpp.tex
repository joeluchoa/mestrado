% ================================
% O Problema do Caminho Mais Longo
% ================================

O Problema do Caminho Mais Longo (LPP -- \textit{Longest Path Problem}) � o problema no qual deseja-se encontrar o  caminho de maior custo entre um ponto de origem e um ponto de destino. O custo neste caso pode ser a quantidade de v�rtices no caminho ou uma fun��o que leva em considera��o valores associados �s arestas.

De um modo mais formal, o Problema do Caminho Mais Longo pede que, dado um grafo n�o-direcionado $G(V,A)$ e um $K \geq 0$, $K \in \mathds{N}$, decida se existe ou n�o em $G$ um caminho simples de tamanho $N$, tal que $N \geq K$. Um caminho simples entende-se como um caminho que n�o cont�m ciclos, ou seja, n�o pode haver repeti��es de v�rtices ao longo de todo o caminho.

O LPP � outro problema que, embora resolvido de forma eficiente para algumas poucas classes de grafos, n�o possui uma solu��o conhecida de ordem polinomial para o caso geral \cite{Uehara:2005}. Uma prova de que o LPP pertence � classe de problemas $\mathcal{NP}$-Completo utiliza a redu��o do Caminho Hamiltoniano ao mesmo.

Para a prova de tal complexidade do LPP, considere a variante deste problema onde dado um grafo n�o-direcionado $G(V,A)$, dois v�rtices $s,t \in V$, e um inteiro positivo $K \leq |V|$, decida se $G$ possui um caminho simples $s \leadsto t$ com uma quantidade $N \geq K$ de arestas. Agora, o Caminho Hamiltoniano pode ser reduzido ao LPP descrevendo-o da seguinte forma: dado um grafo n�o-direcionado $H$ e $v,w \in V(H)$ dois de seus v�rtices, decida se h� um caminho hamiltoniano $v \leadsto w$ entre os dois v�rtices. Como este �ltimo problema � provado ser $\mathcal{NP}$-Completo em \cite{Garey:1979}, temos que LPP pertence tamb�m � mesma classe de complexidade.