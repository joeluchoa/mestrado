No Problema do Roteamento de Ve�culos com Capacidade (CVRP -- \textit{Capacitated Vehicle Routing Problem}) deseja-se encontrar em um grafo um conjunto de rotas �timas para que uma frota de ve�culos id�nticos, com capacidade limitada de carga, atendam as demandas de um conjunto de clientes \cite{Pecin:2010}. Saindo de um �nico dep�sito os ve�culos devem passar por todos os pontos estabelecidos, realizar a tarefa estipulada em cada ponto de parada e retornar ao dep�sito de origem. Existem dois tipos de tarefas que s�o empregadas neste problema: carregamento e entrega de mercadorias. No primeiro o ve�culo deve sair vazio do dep�sito, coletar uma certa quantidade de mercadorias em cada ponto espeficificado ao longo de sua rota e retornar ao dep�sito. No segundo o ve�culo deve sair carregado do dep�sito, entregar uma quantidade de mercadoria em cada ponto de sua rota e retornar vazio ao dep�sito. Em ambos os casos devem ser respeitados os limites de carga dos ve�culos, as demandas dos clientes devem ser atendidas integralmente e as rotas selecionadas devem apresentar o menor custo poss�vel.

Segue uma descri��o formal do problema. Seja $G(V,A)$ um grafo, onde $V$ � o conjunto de v�rtices que representam $n = |V|-1$ clientes e um dep�sito, e $A$ o conjunto de arcos que representam as ruas que ligam clientes e dep�sito, e sejam $c(i,j)$ o custo de percorrer o caminho $i \leadsto j$ ($i,j \in V$), $C$ a capacidade de carga de todos os $K$ ve�culos de transporte ($C,K \in \mathds{N}$), $d_i$ a demanda do cliente $i$ ($i = 1, \ldots, n$), encontre um conjunto de rotas $R \in V$ ($|R| \leq K$) que atenda todas as demandas dos clientes e que minimize o custo total das rotas \cite{Machado:2002}. Segue a formula��o em PLI.
$$\begin{array}{rl}
\text{Minimizar:} & \displaystyle\sum_{i=1}^{n} v_ix_i \\
\end{array}$$
