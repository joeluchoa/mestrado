\section{Motiva��o}

\subsection*{Problema}

\begin{frame}{Provisionamento de sinais}
	\begin{itemize}
		\item Servi�o de interliga��o de sinais.
		\item Permite que sinais do ponto $a$ sejam enviados ao ponto $b$.
	\end{itemize}
	\begin{figure}
	\pgfuseimage{interligacao}
   	\caption{Destacamos um poss�vel caminho interligando $a$ e $b$.}
	\end{figure}
\end{frame}

\begin{frame}{Processo de provisionamento}
 \begin{block}{Requisitos do cliente}
	\begin{itemize}
		\item Quantidade de sinais.
		\item Tipos de sinais.
		\item Origem e destino.
		\item Qualidade de servi�o.
	\end{itemize}
 \end{block}
\begin{figure}
	\pgfuseimage{aprovisionamento}
   	%\caption{Caminho entre $a$ e $b$ com cada elemento exibindo as quantidades usada e a m�xima de transmiss�o.}
	\end{figure}
\end{frame}

\begin{frame}{Processo de provisionamento}
\begin{block}{Considera��es}
	\begin{itemize}
		\item Custo � fun��o do n�mero de equipamentos usados.
		\item N�mero de caminhos depende da quantidade de sinais.
		\item Crit�rio de desempate � a dist�ncia total.
	\end{itemize}
\end{block}
\begin{figure}
\pgfuseimage{aprovisionamento2}
%\caption{Caminho entre $a$ e $b$ com cada elemento exibindo as quantidades usada e a m�xima de transmiss�o.}
\end{figure}
\end{frame}



%% \begin{frame}{Algoritmo final entregue}
%% \framesubtitle{Algoritmo de Katoh, Ibaraki e Mine(KIM)}
%%   \begin{tikzpicture}[scale=0.85]
%%     \tikzstyle{node}=%
%%     [%
%%       inner sep=0pt,%
%%       outer sep=0pt,%
%%       ball color=example text.fg,%
%%       circle%
%%     ]
%%         \node[node,minimum size=8pt,label=left:{s}] at (0,3) (S)  {};
%%         \node[node] at (1,3) (A)  {};
%%         \node[node] at (2,3) (B)  {};
%%         \node[node] at (3,3) (C)  {};
%%         \node[node] at (4,1) (D)  {};
%%         \node[node,minimum size=8pt,label=right:{t}] at (11,3) (T) {};
%%        
%%         \node[color=red] at (6,3.2) (P1) {P1};
%% 
%%         \path [-,thick,red]  
%%                   (S) edge[] (T)
%%                   (S) edge (A)
%%                   (A) edge (B)
%%                   (B) edge (C);
%%        \onslide<2->{
%%         \path [-,thick,blue]  
%%                   (B) edge[bend right] (D)
%%                   (D) edge[bend right] (T);
%%          \node[color=blue] at (6,1) (P2) {P2};
%% 
%%         }
%%         \onslide<5->{
%%         \path [-,thick,black,dashed]  
%%                   (A.north) edge[in=90] (T.north);
%%                  \node[color=black] at (6,5.8) (Pc) {Pc};
%% }
%%         \onslide<3->{
%%         \path [-,thick,black,dashed]  
%%                   (C.north) edge[bend left] (T.north);
%%                  \node[color=black] at (6,4.5) (Pb) {Pb};
%% }
%%        \onslide<4->{
%%         \path [-,thick,black,dashed]  
%%                   (D) edge[in=-90,out=-90] (T);
%%                  \node[color=black] at (6,0) (Pa) {Pa};
%% }
%%         \end{tikzpicture}
%%         
%%         
%%         %\begin{figure}
%% 	%\pgfuseimage{kimCaminhos}
%% 	%\caption{\\
%% 	%	$P1$ : caminho m�nimo entre $s$ e $t$. \\
%% %		$P2$ : caminho m�nimo diferente de $P1$. \\ 
%% %		$Pa$,$Pb$,$Pc$: caminhos candidatos.}	
%% %	\end{figure}
%% \end{frame}

