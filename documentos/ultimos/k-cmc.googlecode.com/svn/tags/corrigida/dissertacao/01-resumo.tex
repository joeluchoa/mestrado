%\pagestyle{empty}
%%\addcontentsline{toc}{chapter}{Resumo}
%\longpage

\centerline{\bf Resumo}

\vspace{0.4cm}

\noindent

%\chapter*{Resumo}
Tratamos da generaliza��o do problema da gera��o de caminho m�nimo, 
no qual n�o apenas um, mas v�rios caminhos de menores custos devem ser produzidos.
O problema dos $k$-menores caminhos consiste em listar os $k$ caminhos de menores custos conectando um par de v�rtices.

Esta disserta��o trata de algoritmos para gera��o de $k$-menores caminhos em grafos sim�tricos 
com custos n�o-negativos, bem como algumas implementa��es destes.
 
 
\noindent \textbf{Palavras-chave:} caminhos m�nimos, $k$-menores caminhos.
 
%\vspace{2.0cm}

%\centerline{\bf Abstract} \vspace{0.4cm}

%\noindent
%We consider a long-studied generalization of the shortest path problem,
%in which not one but several short paths must be produced.
%The $k$-shortest (simple) paths problem is to list the $k$ paths 
%connecting a given source-destination pair in the digraph with minimum total length.

%This dissertation deals with $k$-shortest simple paths algorithms 
%designed for non-negative costs, undirected graphs and 
%some implementations of them. 




