% Copyright 2007 by Till Tantau
%
% This file may be distributed and/or modified
%
% 1. under the LaTeX Project Public License and/or
% 2. under the GNU Public License.
%
% See the file doc/licenses/LICENSE for more details.



\documentclass{beamer}
\mode<presentation>

% Setup appearance:

\usetheme{Darmstadt}
\usefonttheme[onlylarge]{structurebold}
\setbeamerfont*{frametitle}{size=\normalsize,series=\bfseries}
\setbeamertemplate{navigation symbols}{}


% Standard packages

\usepackage[portuguese]{babel}
\usepackage[T1]{fontenc}
\usepackage[latin1]{inputenc}
\usepackage{times}


%\usepackage[brazil]{babel}
%\usepackage[latin1]{inputenc}
%\usepackage{times}
%\usepackage[T1]{fontenc}

%% definicao personalizada
\usepackage{../estilo/estilo}
%\usepackage{../estilo/tkz-graph}
%\usepackage{../estilo/tkz-berge}


% Setup TikZ

\usepackage{tikz}
\usetikzlibrary{arrows,petri,trees,decorations.pathmorphing,backgrounds,positioning,fit}
\usepackage[np]{numprint}
\tikzstyle{block}=[dra opacity=0.7,line width=1.4cm]
\usepackage{tkz-berge}
\usetikzlibrary{mindmap}

% Author, Title, etc.

\title[$k$-menores caminhos] 
{%
  Defesa de disserta��o de mestrado \\
  $k$-menores caminhos%
}
\author{F�bio Pisaruk}
\institute[IME/USP]
{%
Instituto de Matem�tica e Estat�stica \\
Universidade de S�o Paulo%
}
\date{\today}

\pgfdeclareimage[height=0.5cm]{ime-logo}{../figs/ime-arquimedes}
\pgfdeclareimage[height=4cm]{interligacao}{../figs/interligacao}
\pgfdeclareimage[height=3cm]{aprovisionamento}{../figs/aprovisionamento}
\pgfdeclareimage[height=3cm]{aprovisionamento2}{../figs/aprovisionamento2}
\pgfdeclareimage[height=3cm]{algoritmoAntigo}{../figs/algoritmoAntigo}
\pgfdeclareimage[height=5cm]{prefixo}{../figs/prefixo}
%\pgfdeclareimage[height=5cm]{allpaths}{../dissertacao/figs/simulacao/simulacao_all_paths}

\logo{\pgfuseimage{ime-logo}}

% The main document

\includeonly{
introducao,
k-caminhos,
yen,
arvores_prefixos,
particoes,
%particoes_yen,
kim,
experimentos,
conclusoes,
extras
}

%% converte eps to pdf on the fly

\usepackage{epstopdf}
 
\begin{document}

\begin{frame}
  \titlepage
\end{frame}

\begin{frame}{Sum�rio}
   %Este trabalho trata de algoritmos para gera��o de $k$-menores caminhos em grafos sim�tricos.
  \tableofcontents
\end{frame}


\section{Introdu��o}

\subsection{Defini��o}
\begin{frame}{Introdu��o}
  \begin{block}{SP - \em{shortest path}}<1->
  \end{block}
  \begin{block}{RCSP - \em{resource constrained shortest path}}<2->
  \end{block}
\end{frame}

\subsection{Aplica��es}
\begin{frame}{Introdu��o - Aplica��es}
  \begin{block}{QoS}<1->
  \end{block}
\end{frame}


\subsection{Objetivos}

\subsection{Preliminares}

\subsection{Organiza��o}


\section{$k$-menores caminhos}
\subsection*{Defini��o}
\begin{frame}{Problema $k$-menores caminhos}
  \begin{quote}
    \textbf{Problema} \kCM$(V,A,c,s,t,k)$:\\
   Dados: 	
 	\begin{itemize}   
 	\item Grafo: $(V,A)$
 	\item Fun��o custo: $c$
 	\item V�rtices: $s$ e $t$ 
 	\item Inteiro positivo: $k$ 
	\end{itemize}	
	Encontrar os $k$-menores caminhos de $s$ a~$t$.
	\end{quote}
\end{frame}

\begin{frame}{Exemplo}
  \begin{columns}
    \column{.7\textwidth}
  \begin{tikzpicture}[auto,thick,sloped,node distance=15mm]
    \tikzstyle{node}=%
    [%
      minimum size=8pt,%
      inner sep=0pt,%
      outer sep=0pt,%
      ball color=example text.fg,%
      circle%
    ]
        \node[node,ball color=red]  (S) [label=left:{s}] {};
        \node[node] (B) [right of=S,label=below:{b}] {};
        \node[node] (G) [right of=B,label=45:{g}] {};
        \node[node] (L) [right of=G,label=45:{l}] {};
        \node[node,ball color=red,label=above:{t}] (T) [right of=L,label=above:{t}] {};
        \node[node] (F) [above of=G,label=45:{f}] {};
        \node[node] (E) [above of=F,label=above:{e}] {};
        \node[node] (H) [below of=G,label=right:{h}] {};
        \node[node] (I) [below of=H,label=right:{i}] {};
        \node[node] (C) [above of=S,label=above:{c}] {};
        \node[node] (A) [below of=S,label=left:{a}] {};
        \node[node] (J) [above of=L,label=above:{j}] {};
        \node[node] (D) [left of=E,label=above:{d}] {};

          
          \path [-,thick,black] (S) edge (A)
                                    edge (B)
                                    edge (C)
                                (B) edge (C)
                                    edge (D)
                                    edge (F)
                                    edge (G)
                                    edge (H)
                                (D) edge (E)   
                                (A) edge (I)
                                (I) edge (H)
                                    edge (T)
                                (E) edge (F)
                                    edge (J)
                                (F) edge (G)
                                (F) edge (L)
                                (G) edge (H)
                                    edge (L)  
                                (J) edge (L)
                                (L) edge (T)
                                (H) edge (L)
                                (C) edge (D);

          \uncover<2>{
          \path[-,thick,red,line width=1mm] (S) edge (A)
                            (A) edge (I)
                            (I) edge (T);
          }
          \uncover<3>{
          \path[-,thick,red,line width=1mm] (S) edge (B)
                            (B) edge (F)
                            (F) edge (L)
                            (L) edge (T) ;
          }
          \uncover<4>{
          \path[-,thick,red,line width=1mm] (S) edge (B)
                            (B) edge (G)
                            (G) edge (L)
                            (L) edge (T) ;
          }
          \uncover<5>{
          \path[-,thick,red,line width=1mm] (S) edge (B)
                            (B) edge (H)
                            (H) edge (L)
                            (L) edge (T) ;
          }
          \uncover<6>{
          \path[-,thick,red,line width=1mm] (S) edge (B)
                            (B) edge (H)
                            (H) edge (I)
                            (I) edge (T) ;
          }
          \uncover<7>{
          \path[-,thick,red,line width=1mm] (S) edge (A)
                            (A) edge (I)
                            (I) edge (H)
                            (H) edge (L)
                            (L) edge (T);
          }
  \end{tikzpicture}

    \column{.3\textwidth}
   \begin{block}{Caminhos}
      \onslide<2->{$\seq{s,a,i,t}$} \\
      \onslide<3->{$\seq{s,b,f,l,t}$} \\
      \onslide<4->{$\seq{s,b,g,l,t}$} \\
      \onslide<5->{$\seq{s,b,h,l,t}$} \\
      \onslide<6->{$\seq{s,b,h,i,t}$} \\
      \onslide<7->{$\seq{s,a,i,h,l,t}$}
 \end{block}

  \end{columns}  

\end{frame}

%\begin{frame}{Exemplo}
%	\only<1>
%	{%
%	\begin{figure}
%	\pgfuseimage{grafoExemplo1}
%	\caption{6-caminhos m�nimos}
%	\end{figure}
%	}
%	\only<2>
%	{%
%	\begin{figure}
%	\pgfuseimage{grafoExemplo2}
%	\caption{	$\seq{s,a,i,t}$
%		}
%	\end{figure}
%	}
%	\only<3>
%	{%
%	\begin{figure}
%	\pgfuseimage{grafoExemplo3}
%	\caption{	$\seq{s,a,i,t}$,
%			$\seq{s,b,f,l,t}$
%		}
%	\end{figure}
%	}
%	\only<4>
%	{%
%	\begin{figure}
%	\pgfuseimage{grafoExemplo4}
%	\caption{	$\seq{s,a,i,t}$,
%			$\seq{s,b,f,l,t}$,
%			$\seq{s,b,g,l,t}$
%		}
%	\end{figure}
%	}
%	\only<5>
%	{%
%	\begin{figure}
%	\pgfuseimage{grafoExemplo5}
%	\caption{	$\seq{s,a,i,t}$,
%			$\seq{s,b,f,l,t}$,
%			$\seq{s,b,g,l,t}$,
%			$\seq{s,b,h,l,t}$
%	}
%	\end{figure}
%	}
%	\only<6>
%	{%
%	\begin{figure}
%	\pgfuseimage{grafoExemplo6}
%	\caption{	$\seq{s,a,i,t}$,
%			$\seq{s,b,f,l,t}$,
%			$\seq{s,b,g,l,t}$,
%			$\seq{s,b,h,l,t}$,
%			$\seq{s,b,h,i,t}$
%	}
%	\end{figure}
%	}
%	\only<7>
%	{%
%	\begin{figure}
%	\pgfuseimage{grafoExemplo7}
%	\caption{	$\seq{s,a,i,t}$,
%			$\seq{s,b,f,l,t}$,
%			$\seq{s,b,g,l,t}$,
%			$\seq{s,b,h,l,t}$,
%			$\seq{s,b,h,i,t}$,
%			$\seq{s,a,i,h,l,t}$
%	}
%	\end{figure}
%	}
%
%\end{frame}


\subsection*{Solu��es}

\begin{frame}{Algoritmo existente na empresa}
\framesubtitle{Busca exaustiva}
	\onslide<6->
	{	
	\alert{Consumo elevado de mem�ria}.
	}
\begin{center}
\begin{tikzpicture}[auto,thick,sloped,node distance=7mm]
\tikzstyle{node}=%
[%
  minimum size=8pt,%
  inner sep=0pt,%
  outer sep=0pt,%
  ball color=example text.fg,%
  circle%
]
    \node[node,ball color=red] 	(A) [label=above:{a}]  {};
    \node[node,right of=A,node distance=20mm] 		(C) [label=45:{c}] {};
    \node[node,above of=C] 		(B) [label=above:{b}] {};
    \node[node,below of=C] 		(D) [label=right:{d}] {};
    \node[node,right of=C,,node distance=20mm,ball color=red] 		(E) [label=above:{e}] {};
      
    \path (A) edge (B);
    \path (A) edge (C);
	 \path (A) edge (D);
    \path (B) edge (C);
    \path (C) edge (D);
    \path (D) edge (E);
    \path (C) edge (E);
    \path (B) edge (E);
%\pgfuseimage{algoritmoAntigo}
\end{tikzpicture}
\end{center}
	\begin{columns}[t]
	  \begin{column}{2cm}
	    \onslide<2->
		{
		\begin{block}
			\small{$\seq{a,b}$\\$\seq{a,c}$\\$\seq{a,d}$}
	  	\end{block}
		}

	  \end{column}

	  \begin{column}{2cm}
	    \onslide<3->
	    {
	    \begin{block}
		    \small{$\seq{a,b,c}$\\\textcolor{red}{$\seq{a,b,e}$}\\$\seq{a,c,b}$\\\textcolor{red}{$\seq{a,c,e}$}\\$\seq{a,c,d}$\\$\seq{a,d,c}$\\\textcolor{red}{$\seq{a,d,e}$}}
		 \end{block}	    
	    }
	  \end{column}
          \begin{column}{2cm}
   	 \onslide<4->
	    {%
	    \begin{block}
	    	\small{\textcolor{red}{$\seq{a,b,c,e}$}\\$\seq{a,b,c,d}$\\\textcolor{red}{$\seq{a,c,b,e}$}\\\textcolor{red}{$\seq{a,c,d,e}$}\\\textcolor{red}{$\seq{a,d,c,e}$}\\$\seq{a,d,c,b}$}
       \end{block}	     
       }
	  \end{column}
          \begin{column}{2cm}
	 \onslide<5->
	    {
	  	 \begin{block}
	   	 \small{\textcolor{red}{$\seq{a,b,c,d,e}$}\\\textcolor{red}{$\seq{a,d,c,b,e}$}}
		 \end{block}	    	    
	    }
	  \end{column}
	\end{columns}
\end{frame}

\begin{frame}{Primeiro algoritmo implementado}
\framesubtitle{Busca em profundidade iterativa}
\begin{center}
\begin{tikzpicture}[auto,thick,sloped,node distance=7mm]
\tikzstyle{node}=%
[%
  minimum size=8pt,%
  inner sep=0pt,%
  outer sep=0pt,%
  ball color=example text.fg,%
  circle%
]
    \node[node,ball color=red] 	(A) [label=above:{a}]  {};
    \node[node,right of=A,node distance=20mm] 		(C) [label=45:{c}] {};
    \node[node,above of=C] 		(B) [label=above:{b}] {};
    \node[node,below of=C] 		(D) [label=right:{d}] {};
    \node[node,right of=C,,node distance=20mm,ball color=red] 		(E) [label=above:{e}] {};
      
    \path (A) edge (B);
    \path (A) edge (C);
	 \path (A) edge (D);
    \path (B) edge (C);
    \path (C) edge (D);
    \path (D) edge (E);
    \path (C) edge (E);
    \path (B) edge (E);
%\pgfuseimage{algoritmoAntigo}
\end{tikzpicture}
\end{center}
	\begin{columns}[t]
	  \begin{column}{2cm}
		\onslide<1->
		{%
		\begin{block}{n�vel 1}
		\end{block}
		}
	  \end{column}
	  \begin{column}{2cm}
		\onslide<2->
		{%
   	    \begin{block}{n�vel 2}
		 \textcolor{red}{$\seq{a,b,e}$}\\\textcolor{red}{$\seq{a,c,e}$}\\\textcolor{red}{$\seq{a,d,e}$}
	    \end{block}
		}
	  \end{column}
          \begin{column}{2cm}
	\onslide<3->
		{%
		\begin{block}{n�vel 3}
    	\textcolor{red}{$\seq{a,b,c,e}$}\\\textcolor{red}{$\seq{a,c,b,e}$}\\\textcolor{red}{$\seq{a,c,d,e}$}\\\textcolor{red}{$\seq{a,d,c,e}$}
	        \end{block}
		}
	  \end{column}
     \begin{column}{2cm}
	\onslide<4->
		{%
           \begin{block}{n�vel 4}
           \textcolor{red}{$\seq{a,b,c,d,e}$}\\\textcolor{red}{$\seq{a,d,c,b,e}$}
           \end{block}
		}
          \end{column}
      	\end{columns}
	\onslide<5->
	{%		
	\alert{Menor consumo de mem�ria e maior consumo de processador}.
	}
\end{frame}

\begin{frame}{M�todo gen�rico}
\begin{algoritmo}

\textbf{M�todo} \Generico{} $(V,A,c,s,t,k)$ %\\[2mm]
   
0\x $\Pcal \larr \mbox{conjunto dos caminhos de $s$ a~$t$}$ 

1\x \para{} $i=1,\ldots,k$ \faca %\\[1mm]

2\xx  $P_i \larr \mbox{caminho de custo m�nimo em $\Pcal$}$ %\\[1mm]

3\xx  $\Pcal \larr \Pcal - P_i$

4\x \devolva{} $\seq{P_1,\ldots,P_k}$

\end{algoritmo}

\end{frame}

\begin{frame}

%% \begin{tikzpicture}
%% \foreach \x/\y/\tokennumber in {0/2/1,1/2/2,2/2/3,
%% 0/1/4,1/1/5,2/1/6,
%% 0/0/7,1/0/8,2/0/9}
%% \node [place,structured tokens={1,...,\tokennumber}] at (\x,\y) {};
%% \end{tikzpicture}
\begin{columns}[t]
\column{.5\textwidth}
\only<1>{
\begin{center}
\begin{tikzpicture}
\tikzstyle{every place}=[draw=blue,fill=blue!20,thick,minimum size=50mm]
\tikzstyle{every token}=[draw=blue,fill=blue!20,thick,minimum size=10mm]
\node[place,label=above:$\Pcal$] {}
[children are tokens,token distance=10mm]
child {node [token,fill=red]  {$P_1$}}
child {node [token,fill=red]  {$P_2$}}
child {node [token,fill=red]  {$P_3$}}
child {node [token,fill=red]  {$P_4$}}
child {node [token,fill=red]  {$P_5$}}
child {node [token,fill=red] {$P_k$}}
child {node [token,fill=red] {$P_{k+1}$}}
child {node [token,fill=red] {$\ldots$}};
\end{tikzpicture}
\end{center}
}
\only<2>{
\begin{center}
\begin{tikzpicture}
\tikzstyle{every place}=[draw=blue,fill=blue!20,thick,minimum size=50mm]
\tikzstyle{every token}=[draw=blue,fill=blue!20,thick,minimum size=10mm]
\node[place,label=above:$\Pcal$] {}
[children are tokens,token distance=10mm]
child {node [token,fill=red]  {$P_2$}}
child {node [token,fill=red]  {$P_3$}}
child {node [token,fill=red]  {$P_4$}}
child {node [token,fill=red]  {$P_5$}}
child {node [token,fill=red] {$P_k$}}
child {node [token,fill=red] {$P_{k+1}$}}
child {node [token,fill=red] {$\ldots$}};
\end{tikzpicture}
\end{center}
}
\only<3>{
\begin{center}
\begin{tikzpicture}
\tikzstyle{every place}=[draw=blue,fill=blue!20,thick,minimum size=50mm]
\tikzstyle{every token}=[draw=blue,fill=blue!20,thick,minimum size=10mm]
\node[place,label=above:$\Pcal$] {}
[children are tokens,token distance=10mm]
child {node [token,fill=red]  {$P_3$}}
child {node [token,fill=red]  {$P_4$}}
child {node [token,fill=red]  {$P_5$}}
child {node [token,fill=red] {$P_k$}}
child {node [token,fill=red] {$P_{k+1}$}}
child {node [token,fill=red] {$\ldots$}};
\end{tikzpicture}
\end{center}
}
\only<4>{
\begin{center}
\begin{tikzpicture}
\tikzstyle{every place}=[draw=blue,fill=blue!20,thick,minimum size=50mm]
\tikzstyle{every token}=[draw=blue,fill=blue!20,thick,minimum size=10mm]
\node[place,label=above:$\Pcal$] {}
[children are tokens,token distance=10mm]
child {node [token,fill=red] {$P_{k+1}$}}
child {node [token,fill=red] {$\ldots$}};
\end{tikzpicture}
\end{center}
}
\column{.5\textwidth}
\only<1>{
\begin{center}
\begin{tikzpicture}
\tikzstyle{every place}=[draw=blue,fill=blue!20,thick,minimum size=50mm]
\tikzstyle{every token}=[draw=blue,fill=blue!20,thick,minimum size=10mm]
\node[place,label=above:$\Qcal$] {}
[children are tokens,token distance=10mm];
\end{tikzpicture}
\end{center}
}
\only<2>{
\begin{center}
\begin{tikzpicture}
\tikzstyle{every place}=[draw=blue,fill=blue!20,thick,minimum size=50mm]
\tikzstyle{every token}=[draw=blue,fill=blue!20,thick,minimum size=10mm]
\node[place,label=above:$\Qcal$] {}
[children are tokens,token distance=10mm]
child {node [token,fill=red]  {$P_1$}};
\end{tikzpicture}
\end{center}
}
\only<3>{
\begin{center}
\begin{tikzpicture}
\tikzstyle{every place}=[draw=blue,fill=blue!20,thick,minimum size=50mm]
\tikzstyle{every token}=[draw=blue,fill=blue!20,thick,minimum size=10mm]
\node[place,label=above:$\Qcal$] {}
[children are tokens,token distance=10mm]
child {node [token,fill=red]  {$P_1$}}
child {node [token,fill=red]  {$P_2$}};
\end{tikzpicture}
\end{center}
}
\only<4>{
\begin{center}
\begin{tikzpicture}
\tikzstyle{every place}=[draw=blue,fill=blue!20,thick,minimum size=50mm]
\tikzstyle{every token}=[draw=blue,fill=blue!20,thick,minimum size=10mm]
\node[place,label=above:$\Qcal$] {}
[children are tokens,token distance=10mm]
child {node [token,fill=red]  {$P_1$}}
child {node [token,fill=red]  {$P_2$}}
child {node [token,fill=red]  {$P_3$}}
child {node [token,fill=red]  {$P_4$}}
child {node [token,fill=red]  {$P_5$}}
child {node [token,fill=red]  {$\ldots$}}
child {node [token,fill=red] {$P_k$}};
\end{tikzpicture}
\end{center}
}
\end{columns}
\end{frame}


%% \begin{frame}{Exemplo: �rvore de caminhos}
%% 	$\seq{s,a,i,t}$,
%% 	$\seq{s,b,f,l,t}$,
%% 	$\seq{s,b,g,l,t}$,
%% 	$\seq{s,b,h,l,t}$,
%% 	$\seq{s,b,h,i,t}$,
%% 	$\seq{s,a,i,h,l,t}$
%% \begin{center}
%% \begin{tikzpicture}[auto,thick,sloped]
%%   \tikzstyle{node}=%
%%   [%
%%     minimum size=8pt,%
%%     inner sep=0pt,%
%%     outer sep=0pt,%
%%     ball color=example text.fg,%
%%     circle%
%%   ]
%%   \tikzstyle{leaf}=%
%%   [%
%%     minimum size=10pt,%
%%     inner sep=0pt,%
%%     outer sep=0pt,%
%%     ball color=red,%
%%     circle%
%%   ]
%% 
%% % Set the overall layout of the tree
%% \tikzstyle{level 1}=[level distance=1cm, sibling distance=5cm]
%% \tikzstyle{level 2}=[level distance=1cm, sibling distance=2cm]
%% 
%% % Define styles for bags and leafs
%% \tikzstyle{bag} = [text width=4em, text centered]
%% \tikzstyle{end} = [circle, minimum width=3pt,fill, inner sep=0pt]
%%   \node [leaf,label=above:{s}] {} [->] 
%%     child {
%% 	node [node,label=above:{a}] {} 
%% 	child {
%% 		node [node,label=right:{i}] {} 
%% 		child 	{
%% 				node [leaf,label=below:{t1}] {} 	
%% 				edge from parent 
%% 			}
%% 		child 	{
%% 				node [node,label=right:{h}] {} 
%% 				child {
%% 					node [node,label=right:{l}] {} 
%% 					child{
%% 						node [leaf,label=below:{t6}] {} 
%% 						edge from parent 
%% 					}
%% 					edge from parent 
%% 				}
%% 				edge from parent 
%% 			}
%% 		edge from parent 
%% 		}
%% 	edge from parent 
%%         }	
%% 	child {
%% 		node [node,label=above:{b}] {} 	
%%    		child {
%% 			node [node,label=above:{f}] {} 	
%%    			child {
%% 				node [node,label=left:{l}] {} 
%% 				child{
%% 					node [leaf,label=below:{t2}] {} 
%% 					edge from parent 
%% 				}
%% 				edge from parent 
%% 			}
%% 			edge from parent 
%% 		}	
%% 	  	child {
%% 			node [node,label=left:{g}] {} 	
%%    			child {
%% 				node [node,label=left:{l}] {} 
%% 				child{
%% 					node [leaf,label=below:{t3}] {} 
%% 					edge from parent
%% 				}
%% 				edge from parent 
%% 			}
%% 			edge from parent 
%% 		}		
%% 	  	child {
%% 			node [node,label=above:{h}] {} 	
%%    			child {
%% 				node [node,label=above:{l}] {} 
%% 				child{
%% 					node [leaf,label=below:{t4}] {} 
%% 					edge from parent
%% 				}
%% 				edge from parent 
%% 			}
%%    			child {
%% 				node [node,label=above:{i}] {} 
%% 				child{
%% 					node [leaf,label=below:{t5}] {} 
%% 					edge from parent 
%% 				}
%% 				edge from parent 
%% 			}
%% 			edge from parent 
%% 		}	
%% 		edge from parent 
%% 	};
%% \end{tikzpicture}
%% \end{center}
%% \end{frame}






\section*{�rvore dos prefixos}

\begin{frame}{Exemplo de constru��o}
  \begin{columns}  
    \column{.4\textwidth}
   \begin{tikzpicture}[auto,thick,sloped,node distance=10mm,scale=0.7]
    \tikzstyle{node}=%
    [%
      minimum size=8pt,%
      inner sep=0pt,%
      outer sep=0pt,%
      ball color=example text.fg,%
      circle%
    ]
        \node[label distance=1mm,node,ball color=red,label=left:{s}]  (S)  {};
        \node[node] (B) [right of=S,label=below:{b}] {};
        \node[node] (G) [right of=B,label=45:{g}] {};
        \node[node,right of=G] (L) [label distance=5mm,label=45:{l}] {};
        \node[node,ball color=red,label=above:{t}] (T) [right of=L,label=above:{t}] {};
        \node[node] (F) [above of=G,label=45:{f}] {};
        \node[node] (E) [above of=F,label=above:{e}] {};
        \node[node] (H) [below of=G,label=right:{h}] {};
        \node[node] (I) [below of=H,label=right:{i}] {};
        \node[node] (C) [above of=S,label=above:{c}] {};
        \node[node] (A) [below of=S,label=left:{a}] {};
        \node[node] (J) [right of=E,label=above:{j}] {};
        \node[node] (D) [left of=E,label=above:{d}] {};

          
          \path [-,thick,black] (S) edge (A)
                                    edge (B)
                                    edge (C)
                                (B) edge (C)
                                    edge (D)
                                    edge (F)
                                    edge (G)
                                    edge (H)
                                (D) edge (E)   
                                (A) edge (I)
                                (I) edge (H)
                                    edge (T)
                                (E) edge (F)
                                    edge (J)
                                (F) edge (G)
                                (F) edge (L)
                                (G) edge (H)
                                    edge (L)  
                                (J) edge (L)
                                (L) edge (T)
                                (H) edge (L)
                                (C) edge (D);

          \uncover<2-3>{
          \path[-,thick,red,line width=1mm] (S) edge (A)
                            (A) edge (I)
                            (I) edge (T);
          }
          \uncover<4-5>{
          \path[-,thick,red,line width=1mm] (S) edge (B)
                            (B) edge (F)
                            (F) edge (L)
                            (L) edge (T) ;
          }
          \uncover<6-7>{
          \path[-,thick,red,line width=1mm] (S) edge (B)
                            (B) edge (G)
                            (G) edge (L)
                            (L) edge (T) ;
          }
          \uncover<8-9>{
          \path[-,thick,red,line width=1mm] (S) edge (B)
                            (B) edge (H)
                            (H) edge (L)
                            (L) edge (T) ;
          }
          \uncover<10-11>{
          \path[-,thick,red,line width=1mm] (S) edge (B)
                            (B) edge (H)
                            (H) edge (I)
                            (I) edge (T) ;
          }
          \uncover<12-13>{
          \path[-,thick,red,line width=1mm] (S) edge (A)
                            (A) edge (I)
                            (I) edge (H)
                            (H) edge (L)
                            (L) edge (T);
          }
  \end{tikzpicture}

    \column{.6\textwidth}
\begin{tikzpicture}[auto,thick,sloped,scale=0.7]
  \tikzstyle{node}=%
  [%
    minimum size=8pt,%
    inner sep=0pt,%
    outer sep=0pt,%
    ball color=example text.fg,%
    circle%
  ]
  \tikzstyle{leaf}=%
  [%
    minimum size=10pt,%
    inner sep=0pt,%
    outer sep=0pt,%
    ball color=red,%
    circle%
  ]

% Set the overall layout of the tree
\tikzstyle{level 1}=[level distance=1cm, sibling distance=5cm]
\tikzstyle{level 2}=[level distance=1cm, sibling distance=2cm]

% Define styles for bags and leafs
\tikzstyle{bag} = [text width=4em, text centered]
\tikzstyle{end} = [circle, minimum width=3pt,fill, inner sep=0pt]
\uncover<3-4>{
  \node [leaf,label=above:{s}] (raiz){} [] 
    child {
	node [node,label=right:{a}] (sa) {} 
  	child {
		node [node,label=right:{i}] (sai) {} 
		child 	{
				node [leaf,label=below:{t1}] (sait){} 	
				edge from parent 
			}
		edge from parent 
	}
	edge from parent
   };
	\path[opacity=1,color=red,->] [-latex]
	(raiz) edge (sa)
	(sa) edge (sai)
	(sai) edge (sait);
}
\uncover<5-6>{
  \node [leaf,label=above:{s}] (raiz) {} [] 
    	child {
		node [node,label=above:{a}] {} [->]
		child {
			node [node,label=right:{i}] {} [->]
			child 	{
					node [leaf,label=below:{t1}] {} [->]	
					edge from parent 
				}
			edge from parent 
			}
		edge from parent 
        }	
	child {
		node [node,label=above:{b}] (sb){} 	
   		child {
			node [node,label=left:{f}] (sbf) {} 	
   			child {
				node [node,label=left:{l}] (sbfl) {} 
				child{
					node [leaf,label=below:{t2}] (sbflt) {} 
					edge from parent 
				}
				edge from parent 
			}
			edge from parent 
		}	
		edge from parent 
	};
	\path[opacity=1,color=red,->] [-latex]
	(raiz) edge (sb)
	(sb) edge (sbf)
	(sbf) edge (sbfl)
	(sbfl) edge (sbflt);
}
\uncover<7-8>{
  \node [leaf,label=above:{s}] (raiz) {} [] 
    child {
	node [node,label=above:{a}] {} [->]
	child {
		node [node,label=right:{i}] {} [->]
		child 	{
				node [leaf,label=below:{t1}] {} [->]	
				edge from parent 
			}
		edge from parent 
		}
	edge from parent 
        }	
	child {
		node [node,label=above:{b}] (sb) {} []	
   		child {
			node [node,label=above:{f}]  {} [->] 	
   			child {
				node [node,label=left:{l}] {} [->]
				child{
					node [leaf,label=below:{t2}] {} [->]
					edge from parent 
				}
				edge from parent 
			}
			edge from parent 
		}	
	  	child {
			node [node,label=left:{g}] (sbg){} 	
   			child {
				node [node,label=left:{l}] (sbgl) {} 
				child{
					node [leaf,label=below:{t3}] (sbglt) {} 
					edge from parent
				}
				edge from parent 
			}
			edge from parent 
		}		
		edge from parent 
	};
	\path[opacity=1,color=red,->] [-latex]
	(raiz) edge (sb)
	(sb) edge (sbg)
	(sbg) edge (sbgl)
	(sbgl) edge (sbglt);
}
\uncover<9-10>{
  \node [leaf,label=above:{s}] (raiz) {} [] 
    child {
	node [node,label=above:{a}] {} [->]
	child {
		node [node,label=right:{i}] {}  
		child 	{
				node [leaf,label=below:{t1}] {} 	
				edge from parent 
			}
		edge from parent 
		}
	edge from parent 
        }	
	child {
		node [node,label=above:{b}] (sb) {} 	
   		child {
			node [node,label=above:{f}] {} 	[->]
   			child {
				node [node,label=left:{l}] {} 
				child{
					node [leaf,label=below:{t2}] {} 
					edge from parent 
				}
				edge from parent 
			}
			edge from parent 
		}	
	  	child {
			node [node,label=left:{g}] {} [->]	
   			child {
				node [node,label=left:{l}] {} 
				child{
					node [leaf,label=below:{t3}] {} 
					edge from parent
				}
				edge from parent 
			}
			edge from parent 
		}		
	  	child {
			node [node,label=right:{h}] (sbh) {} 	
   			child {
				node [node,label=right:{l}] (sbhl) {} 
				child{
					node [leaf,label=below:{t4}] (sbhlt) {} 
					edge from parent
				}
				edge from parent 
			}
			edge from parent 
		}	
		edge from parent 
	};
	\path[opacity=1,color=red,->] [-latex]
	(raiz) edge (sb)
	(sb) edge (sbh)
	(sbh) edge (sbhl)
	(sbhl) edge (sbhlt);

}
\uncover<11-12>{
  \node [leaf,label=above:{s}] (raiz) {} [] 
    child {
	node [node,label=above:{a}] {} [->]
	child {
		node [node,label=right:{i}] {} 
		child 	{
				node [leaf,label=below:{t1}] {} 	
				edge from parent 
			}
		edge from parent 
		}
	edge from parent 
        }	
	child {
		node [node,label=above:{b}] (sb){} 	
   		child {
			node [node,label=above:{f}] {} 	[->]
   			child {
				node [node,label=left:{l}] {} 
				child{
					node [leaf,label=below:{t2}] {} 
					edge from parent 
				}
				edge from parent 
			}
			edge from parent 
		}	
	  	child {
			node [node,label=left:{g}] {} 	[->]
   			child {
				node [node,label=left:{l}] {} 
				child{
					node [leaf,label=below:{t3}] {} 
					edge from parent
				}
				edge from parent 
			}
			edge from parent 
		}		
	  	child {
			node [node,label=above:{h}] (sbh){} 	
   			child {
				node [node,label=above:{l}] {} [->]
				child{
					node [leaf,label=below:{t4}] {} 
					edge from parent
				}
				edge from parent 
			}
   			child {
				node [node,label=above:{i}] (sbhi){} 
				child{
					node [leaf,label=below:{t5}] (sbhit) {} 
					edge from parent 
				}
				edge from parent 
			}
			edge from parent 
		}	
		edge from parent 
	};
	\path[opacity=1,color=red,->] [-latex]
	(raiz) edge (sb)
	(sb) edge (sbh)
	(sbh) edge (sbhi)
	(sbhi) edge (sbhit);
}
\uncover<13-14>{
  \node [leaf,label=above:{s}] (raiz) {} [] 
    child {
	node [node,label=above:{a}] (sa) {} 
	child {
		node [node,label=right:{i}] (sai){} 
		child[grow=down] 	{
				node [leaf,label=below:{t1}] {} [->]	
				edge from parent 
			}
		child 	{
				node [node,label=right:{h}] (saih){} 
				child {
					node [node,label=right:{l}](saihl) {} 
					child{
						node [leaf,label=below:{t6}] (saihlt) {} 
						edge from parent 
					}
					edge from parent 
				}
				edge from parent 
			}
		edge from parent 
		}
	edge from parent 
        }	
	child {
		node [node,label=above:{b}] {} [->]	
   		child {
			node [node,label=above:{f}] {} 	
   			child {
				node [node,label=left:{l}] {} 
				child{
					node [leaf,label=below:{t2}] {} 
					edge from parent 
				}
				edge from parent 
			}
			edge from parent 
		}	
	  	child {
			node [node,label=left:{g}] {} [->]	
   			child {
				node [node,label=left:{l}] {} 
				child{
					node [leaf,label=below:{t3}] {} 
					edge from parent
				}
				edge from parent 
			}
			edge from parent 
		}		
	  	child {
			node [node,label=above:{h}] {} [->]
   			child {
				node [node,label=above:{l}] {} 
				child{
					node [leaf,label=below:{t4}] {} 
					edge from parent
				}
				edge from parent 
			}
   			child {
				node [node,label=above:{i}] {} 
				child{
					node [leaf,label=below:{t5}] {} 
					edge from parent 
				}
				edge from parent 
			}
			edge from parent 
		}	
		edge from parent 
	};
	\path[opacity=1,color=red,->] [-latex]
	(raiz) edge (sa)
	(sa) edge (sai)
	(sai) edge (saih)
	(saih) edge (saihl)	
	(saihl) edge (saihlt);
}
\end{tikzpicture}
\end{columns}
\end{frame}

\section{Parti��es}

 \begin{frame}{Quest�es}
 \begin{itemize}
 \item<1-> $k$ �, potencialmente, exponencial no n�mero de v�rtices
 \item<2-> um grafo completo com $15$ v�rtices possui $16.926.797.485$ de caminhos entre dois v�rtices fixados
 \item<3-> particionar os caminhos utilizando a �rvore dos prefixos
 \end{itemize}
 \end{frame}
  \begin{frame}
 \begin{block}{Descri��o}
 \begin{itemize}
 \item $\Qcal$: cole��o de caminhos
 \item $(N,E,f)$: �rvore dos prefixos de $\Qcal$ 
 \item Para cada n� $u$ da �rvore temos uma parti��o $\pi_u$
 \item $\Pi = \{\pi_u, u \in (N,E,f)\}$
 \end{itemize}
 \end{block}
%\end{frame}
 
 \begin{block}{Parti��o $\pi_u$}
\begin{itemize}
 \item caminhos com ponta inicial na raiz
 \item compartilham o prefixo $f(R_u)$
 \item n�o possuem arcos em $A_u$
 \end{itemize}
 \end{block}
 \end{frame}

\begin{frame}
\only<1>{
\alert{$\Pi$ cont�m todos os caminhos de $s$ a $t$}
\begin{center}
\begin{tikzpicture}
\tikzstyle{every place}=[draw=blue,fill=blue!20,thick,minimum size=50mm]
\tikzstyle{every token}=[draw=blue,fill=blue!20,thick,minimum size=10mm]
\node[place,label=above:$\Pi$] {}
[children are tokens,token distance=10mm]
child {node [token,fill=red]  {$P_1$}}
child {node [token,fill=red]  {$P_2$}}
child {node [token,fill=red]  {$P_3$}}
child {node [token,fill=red]  {$P_4$}}
child {node [token,fill=red]  {$P_5$}}
child {node [token,fill=red] {$P_k$}}
child {node [token,fill=red] {$P_{k+1}$}}
child {node [token,fill=red] {$\ldots$}};
\end{tikzpicture}
\end{center}
}
\only<2>{
\alert{$\Qcal \larr P_1=\seq{s,a,i,t}$}
\begin{center}
\begin{tikzpicture}
\tikzstyle{every place}=[draw=blue,fill=blue!20,thick,minimum size=70mm]
\tikzstyle{every token}=[draw=blue,fill=blue!20,thick,minimum size=10mm]
\node[place,label=above:$\Pi$] {}
[children are tokens,token distance=30mm]
child {node [token,fill=red,label=above:{$\pi_s$}]  {} }
child {node [token,fill=red,label=above:{$\pi_a$}]  {}}
child {node [token,fill=red,label=above:{$\pi_i$}]  {}};
\end{tikzpicture}
\end{center}
}
\only<3>{
\alert{$\Qcal \larr \seq{s,a,i,t} \cup \seq{s,b,f,l,t}$}
\begin{center}
\begin{tikzpicture}
\tikzstyle{every place}=[draw=blue,fill=blue!20,thick,minimum size=70mm]
\tikzstyle{every token}=[draw=blue,fill=blue!20,thick,minimum size=10mm]
\node[place,label=above:$\Pi$] {}
[children are tokens,token distance=30mm]
child {node [token,fill=red,label=above:{$\pi_s$}]  {} }
child {node [token,fill=red,label=above:{$\pi_{a}$}]  {}}
child {node [token,fill=red,label=above:{$\pi_{i}$}]  {}}
child {node [token,fill=red,label=above:{$\pi_{b}$}]  {}}
child {node [token,fill=red,label=above:{$\pi_{l}$}]  {}};

\end{tikzpicture}
\end{center}
}

\end{frame}

%%%%%%%%%%%%%%%%%%%%%%%%%%%%%%%%%%%%%%%%%%%%%%%%%%%%%%%%%%%%%%%%%%%%%%%%%%%%%%%%%%%%%%%%%%%%%%%%%%%

\begin{frame}{Exemplo}
  \begin{columns}  
    \column{.4\textwidth}
   \begin{tikzpicture}[auto,thick,sloped,node distance=10mm,scale=0.7]
    \tikzstyle{node}=%
    [%
      minimum size=8pt,%
      inner sep=0pt,%
      outer sep=0pt,%
      ball color=example text.fg,%
      circle%
    ]
        \node[label distance=1mm,node,ball color=red,label=left:{s}]  (S)  {};
        \node[node] (B) [right of=S,label=below:{b}] {};
        \node[node] (G) [right of=B,label=45:{g}] {};
        \node[node,right of=G] (L) [label distance=5mm,label=45:{l}] {};
        \node[node,ball color=red,label=above:{t}] (T) [right of=L,label=above:{t}] {};
        \node[node] (F) [above of=G,label=45:{f}] {};
        \node[node] (E) [above of=F,label=above:{e}] {};
        \node[node] (H) [below of=G,label=right:{h}] {};
        \node[node] (I) [below of=H,label=right:{i}] {};
        \node[node] (C) [above of=S,label=above:{c}] {};
        \node[node] (A) [below of=S,label=left:{a}] {};
        \node[node] (J) [right of=E,label=above:{j}] {};
        \node[node] (D) [left of=E,label=above:{d}] {};

          
          \path [-,thick,black] (S) edge (A)
                                    edge (B)
                                    edge (C)
                                (B) edge (C)
                                    edge (D)
                                    edge (F)
                                    edge (G)
                                    edge (H)
                                (D) edge (E)   
                                (A) edge (I)
                                (I) edge (H)
                                    edge (T)
                                (E) edge (F)
                                    edge (J)
                                (F) edge (G)
                                (F) edge (L)
                                (G) edge (H)
                                    edge (L)  
                                (J) edge (L)
                                (L) edge (T)
                                (H) edge (L)
                                (C) edge (D);

          \uncover<2-5>{
          \path[-,thick,red,line width=1mm] (S) edge (A)
                            (A) edge (I)
                            (I) edge (T);
          }
          \uncover<6-11>{
          \path[-,thick,red,line width=1mm] (S) edge (B)
                            (B) edge (F)
                            (F) edge (L)
                            (L) edge (T) ;
          }
          \uncover<12>{
          \path[-,thick,red,line width=1mm] (S) edge (B)
                            (B) edge (G)
                            (G) edge (L)
                            (L) edge (T) ;
          }
%%           \uncover<5>{
%%           \path[-,thick,red,line width=1mm] (S) edge (B)
%%                             (B) edge (H)
%%                             (H) edge (L)
%%                             (L) edge (T) ;
%%           }
%%           \uncover<6>{
%%           \path[-,thick,red,line width=1mm] (S) edge (B)
%%                             (B) edge (H)
%%                             (H) edge (I)
%%                             (I) edge (T) ;
%%           }
%%           \uncover<7>{
%%           \path[-,thick,red,line width=1mm] (S) edge (A)
%%                             (A) edge (I)
%%                             (I) edge (H)
%%                             (H) edge (L)
%%                             (L) edge (T);
%%           }
  \end{tikzpicture}

    \column{.6\textwidth}
\begin{tikzpicture}[auto,thick,sloped,scale=0.7]
  \tikzstyle{node}=%
  [%
    minimum size=8pt,%
    inner sep=0pt,%
    outer sep=0pt,%
    ball color=example text.fg,%
    circle%
  ]
  \tikzstyle{leaf}=%
  [%
    minimum size=10pt,%
    inner sep=0pt,%
    outer sep=0pt,%
    ball color=red,%
    circle%
  ]
  \tikzstyle{ghost}=%
  [%
    minimum size=8pt,%
    inner sep=0pt,%
    outer sep=0pt,%
    ball color=gray,%
    circle%
  ]

% Set the overall layout of the tree
\tikzstyle{level 1}=[level distance=1cm, sibling distance=5cm]
\tikzstyle{level 2}=[level distance=1cm, sibling distance=2cm]

% Define styles for bags and leafs
\tikzstyle{bag} = [text width=4em, text centered]
\tikzstyle{end} = [circle, minimum width=3pt,fill, inner sep=0pt]
\uncover<3-6>{
  \node [leaf,label=above:{s}] (raiz){} [] 
    child {
	node [node,label=right:{a}] (sa) {} 
  	child {
		node [node,label=right:{i}] (sai) {} 
		child 	{
				node [leaf,label=below:{t1}] (sait){} 	
				edge from parent 
			}
		edge from parent 
	}
	edge from parent
   };
	\path[opacity=1,color=red,->] [-latex]
	(raiz) edge (sa)
	(sa) edge (sai)
	(sai) edge (sait);
}
%%particao de pi_s 
\uncover<4>{
  	\node [leaf,label=above:{s}] (raiz){} [] 
	 child[sibling distance=15mm]{
	 										node[ghost,label=left:{b}] {} 
	 										child[grow=down] {
	 																node {...}
							 										child[grow=down] {
							 											node[leaf,label=below:{t}] {}
		 																edge from parent[dashed]
							 										}
	 																edge from parent[dashed]
	 																}
	 										edge from parent[dashed]
	 }
  	 child[sibling distance=15mm] {
				node [node,label=right:{a}] (sa) {} 
			  	child {
						node [node,label=right:{i}] (sai) {} 
						child{
								node [leaf,label=below:{t1}] (sait){} 	
								edge from parent 
						}
				edge from parent 
	}
	edge from parent
   }
   child[sibling distance=15mm]{
	 										node[ghost,label=right:{c}] {} 
	 										child[grow=down] {
	 																node {...}
							 										child[grow=down] {
							 											node[leaf,label=below:{t}] {}
		 																edge from parent[dashed]
							 										}
	 																edge from parent[dashed]
	 																}
	 										edge from parent[dashed]
	 }
;
   
	\path[opacity=1,color=red,->] [-latex]
	(raiz) edge (sa)
	(sa) edge (sai)
	(sai) edge (sait);
}

%%particao de pi_sai 
\uncover<5>{
  	\node [leaf,label=above:{s}] (raiz){} [] 
  	 child[sibling distance=15mm] {
				node [node,label=right:{a}] (sa) {} 
			  	child {
						node [node,label=right:{i}] (sai) {} 
						child[grow=down]{
								node [leaf,label=below:{t1}] (sait){} 	
								edge from parent 
						}
  					   child{
	 										node[ghost,label=right:{h}] {} 
	 										child[grow=down] {
	 																node {...}
						 										child[grow=down] {
						 											node[leaf,label=below:{t}] {}
	 																edge from parent[dashed]
						 										}
	 																edge from parent[dashed]
	 																}
	 										edge from parent[dashed]
	 					}
						edge from parent 
				}
	edge from parent
   };
   
	\path[opacity=1,color=red,->] [-latex]
	(raiz) edge (sa)
	(sa) edge (sai)
	(sai) edge (sait);
}

%caminhos s,b,f,l,t
\uncover<7>{
  \node [leaf,label=above:{s}] (raiz) {} [] 
    	child[sibling distance=15mm] {
		node [node,label=above:{a}] {} [->]
		child {
			node [node,label=right:{i}] {} [->]
			child 	{
					node [leaf,label=below:{t1}] {} [->]	
					edge from parent 
				}
			edge from parent 
			}
		edge from parent 
        }	
	child[sibling distance=15mm] {
		node [node,label=above:{b}] (sb){} 	
   		child {
			node [node,label=right:{f}] (sbf) {} 	
   			child {
				node [node,label=right:{l}] (sbfl) {} 
				child{
					node [leaf,label=below:{t2}] (sbflt) {} 
					edge from parent 
				}
				edge from parent 
			}
			edge from parent 
		}	
		edge from parent 
	};
	\path[opacity=1,color=red,->] [-latex]
	(raiz) edge (sb)
	(sb) edge (sbf)
	(sbf) edge (sbfl)
	(sbfl) edge (sbflt);
}

%particao pi_s
\uncover<8>{
  \node [leaf,label=above:{s}] (raiz) {} [] 
    	child[sibling distance=15mm] {
		node [node,label=above:{a}] {} [->]
		child {
			node [node,label=right:{i}] {} [->]
			child 	{
					node [leaf,label=below:{t1}] {} [->]	
					edge from parent 
				}
			edge from parent 
			}
		edge from parent 
        }	
	child[sibling distance=15mm] {
		node [node,label=right:{b}] (sb){} 	
   		child {
			node [node,label=right:{f}] (sbf) {} 	
   			child {
				node [node,label=right:{l}] (sbfl) {} 
				child{
					node [leaf,label=below:{t2}] (sbflt) {} 
					edge from parent 
				}
				edge from parent 
			}
			edge from parent 
		}	
		edge from parent 
	}
   child[sibling distance=15mm]{
	 										node[ghost,label=right:{c}] {} 
	 										child[grow=down] {
	 																node {...}
							 										child[grow=down] {
							 											node[leaf,label=below:{t}] {}
		 																edge from parent[dashed]
							 										}
	 																edge from parent[dashed]
	 																}
	 										edge from parent[dashed]
	 }	
	;
	\path[opacity=1,color=red,->] [-latex]
	(raiz) edge (sb)
	(sb) edge (sbf)
	(sbf) edge (sbfl)
	(sbfl) edge (sbflt);
}

%particao pi_sb
\uncover<9>{
  \node [leaf,label=above:{s}] (raiz) {} [] 
    	child[sibling distance=40mm] {
		node [node,label=above:{a}] {} [->]
		child {
			node [node,label=left:{i}] {} [->]
			child 	{
					node [leaf,label=below:{t1}] {} [->]	
					edge from parent 
				}
			edge from parent 
			}
		edge from parent 
        }	
	child[sibling distance=40mm] {
			node [node,label=above:{b}] (sb){} 	
			child[sibling distance=15mm]{
	 										node[ghost,label=above:{c}] {} 
	 										child[grow=down] {
	 																node {...}
							 										child[grow=down] {
							 											node[leaf,label=below:{t}] {}
		 																edge from parent[dashed]
							 										}
	 																edge from parent[dashed]
	 																}
	 										edge from parent[dashed]
	 		}	 		
			child[sibling distance=15mm]{
	 										node[ghost,label=right:{d}] {} 
	 										child[grow=down] {
	 																node {...}
							 										child[grow=down] {
							 											node[leaf,label=below:{t}] {}
		 																edge from parent[dashed]
							 										}
	 																edge from parent[dashed]
	 																}
	 										edge from parent[dashed]
	 		}	
	  		child[sibling distance=15mm] {
						node [node,label=right:{f}] (sbf) {} 	
			   			child {
							node [node,label=right:{l}] (sbfl) {} 
							child{
								node [leaf,label=below:{t2}] (sbflt) {} 
								edge from parent 
							}
							edge from parent 
						}
						edge from parent 
			}	 			
			child[sibling distance=15mm]{
	 										node[ghost,label=right:{g}] {} 
	 										child[grow=down] {
	 																node {...}
							 										child[grow=down] {
							 											node[leaf,label=below:{t}] {}
		 																edge from parent[dashed]
							 										}
	 																edge from parent[dashed]
	 																}
	 										edge from parent[dashed]
	 		}		 		
			child[sibling distance=15mm]{
	 										node[ghost,label=right:{h}] {} 
	 										child[grow=down] {
	 																node {...}
							 										child[grow=down] {
							 											node[leaf,label=below:{t}] {}
		 																edge from parent[dashed]
							 										}
	 																edge from parent[dashed]
	 																}
	 										edge from parent[dashed]
	 		}	
		edge from parent 
	};
	\path[opacity=1,color=red,->] [-latex]
	(raiz) edge (sb)
	(sb) edge (sbf)
	(sbf) edge (sbfl)
	(sbfl) edge (sbflt);
}

%particao pi_sbf
\uncover<10>{
  \node [leaf,label=above:{s}] (raiz) {} [] 
    	child[sibling distance=40mm] {
		node [node,label=above:{a}] {} [->]
		child {
			node [node,label=left:{i}] {} [->]
			child 	{
					node [leaf,label=below:{t1}] {} [->]	
					edge from parent 
				}
			edge from parent 
			}
		edge from parent 
        }	
	child[sibling distance=40mm] {
		node [node,label=above:{b}] (sb){} 	
   		child {
						node [node,label=right:{f}] (sbf) {} 	
						child[sibling distance=15mm]{
				 										node[ghost,label=right:{e}] {} 
				 										child[grow=down] {
				 																node {...}
										 										child[grow=down] {
										 											node[leaf,label=below:{t}] {}
					 																edge from parent[dashed]
										 										}
				 																edge from parent[dashed]
				 																}
				 										edge from parent[dashed]
				 		}		 		
						child[sibling distance=15mm] {
									node [node,label=right:{l}] (sbfl) {} 
									child{
										node [leaf,label=below:{t2}] (sbflt) {} 
										edge from parent 
									}
									edge from parent 
						}
						child[sibling distance=15mm]{
				 										node[ghost,label=right:{g}] {} 
				 										child[grow=down] {
				 																node {...}
										 										child[grow=down] {
										 											node[leaf,label=below:{t}] {}
					 																edge from parent[dashed]
										 										}
				 																edge from parent[dashed]
				 																}
				 										edge from parent[dashed]
				 		}		 				 		
						edge from parent 
			}	
			edge from parent 
	};
	\path[opacity=1,color=red,->] [-latex]
	(raiz) edge (sb)
	(sb) edge (sbf)
	(sbf) edge (sbfl)
	(sbfl) edge (sbflt);
}

%particao pi_sbfl
\uncover<11>{

  \node [leaf,label=above:{s}] (raiz) {} [] 
	child[sibling distance=40mm] {
			node [node,label=above:{a}] {} [->]
			child {
				node [node,label=right:{i}] {} [->]
				child 	{
						node [leaf,label=below:{t1}] {} [->]	
						edge from parent 
					}
				edge from parent 
				}
			edge from parent 
	}	
	child[sibling distance=40mm] {
		node [node,label=above:{b}] (sb){} 	
		child[sibling distance=15mm] {
					node [node,label=right:{f}] (sbf) {} 	
					child {
								node [node,label=right:{l}] (sbfl) {} 
								child{
				 										node[ghost,label=right:{g}] {} 
				 										child[grow=down] {
				 																node {...}
										 										child[grow=down] {
										 											node[leaf,label=below:{t}] {}
					 																edge from parent[dashed]
										 										}
				 																edge from parent[dashed]
				 																}
				 										edge from parent[dashed]
								}								
								child[level distance=20mm,grow=down]{
									node [leaf,label=below:{t2}] (sbflt) {} 
									edge from parent 
								}
								child{
				 										node[ghost,label=right:{h}] {} 
				 										child[grow=down] {
				 																node {...}
										 										child[grow=down] {
										 											node[leaf,label=below:{t}] {}
					 																edge from parent[dashed]
										 										}
				 																edge from parent[dashed]
				 																}
				 										edge from parent[dashed]
								}
								child{
				 										node[ghost,label=right:{j}] {} 
				 										child[grow=down] {
				 																node {...}
										 										child[grow=down] {
										 											node[leaf,label=below:{t}] {}
					 																edge from parent[dashed]
										 										}
				 																edge from parent[dashed]
				 																}
				 										edge from parent[dashed]
								}
								edge from parent
				}
					edge from parent 
		}	
		edge from parent 
	};
	
	\path[opacity=1,color=red,->] [-latex]
	(raiz) edge (sb)
	(sb) edge (sbf)
	(sbf) edge (sbfl)
	(sbfl) edge (sbflt);

}
\end{tikzpicture}
\end{columns}
\end{frame}


\section{\Yen}

%% \subsection*{Parti��es}
%%  \begin{frame}
%%  \begin{itemize}
%%  \item<1-> $k$ �, potencialmente, exponencial no n�mero de v�rtices
%%  \item<2-> um grafo completo com $15$ v�rtices possui $16.926.797.485$ de caminhos entre dois v�rtices fixados
%%  \item<3-> particionar os caminhos utilizando a �rvore dos prefixos
%%  \end{itemize}
%%  \end{frame}
%%   \begin{frame}
%%  \begin{block}{Descri��o}
%%  \begin{itemize}
%%  \item $\Qcal$: cole��o de caminhos
%%  \item $(N,E,f)$: �rvore dos prefixos de $\Qcal$ 
%%  \item Para cada n� $u$ da �rvore temos uma parti��o $\pi_u$
%%  \item $\Pi = \{\pi_u, u \in (N,E,f)\}$
%%  \end{itemize}
%%  \end{block}
%% %\end{frame}
%%  
%%  \begin{block}{Parti��o $\pi_u$}
%% \begin{itemize}
%%  \item caminhos com ponta inicial na raiz
%%  \item compartilham o prefixo $f(R_u)$
%%  \item n�o possuem arcos em $A_u$
%%  \end{itemize}
%%  \end{block}
%%  \end{frame}
%% 
%% \begin{frame}
%% \only<1>{
%% \alert{$\Pi$ cont�m todos os caminhos de $s$ a $t$}
%% \begin{center}
%% \begin{tikzpicture}
%% \tikzstyle{every place}=[draw=blue,fill=blue!20,thick,minimum size=50mm]
%% \tikzstyle{every token}=[draw=blue,fill=blue!20,thick,minimum size=10mm]
%% \node[place,label=above:$\Pi$] {}
%% [children are tokens,token distance=10mm]
%% child {node [token,fill=red]  {$P_1$}}
%% child {node [token,fill=red]  {$P_2$}}
%% child {node [token,fill=red]  {$P_3$}}
%% child {node [token,fill=red]  {$P_4$}}
%% child {node [token,fill=red]  {$P_5$}}
%% child {node [token,fill=red] {$P_k$}}
%% child {node [token,fill=red] {$P_{k+1}$}}
%% child {node [token,fill=red] {$\ldots$}};
%% \end{tikzpicture}
%% \end{center}
%% }
%% \only<2>{
%% \alert{$\Qcal \larr P_1=\seq{s,a,i,t}$}
%% \begin{center}
%% \begin{tikzpicture}
%% \tikzstyle{every place}=[draw=blue,fill=blue!20,thick,minimum size=70mm]
%% \tikzstyle{every token}=[draw=blue,fill=blue!20,thick,minimum size=10mm]
%% \node[place,label=above:$\Pi$] {}
%% [children are tokens,token distance=30mm]
%% child {node [token,fill=red,label=above:{$\pi_s$}]  {} }
%% child {node [token,fill=red,label=above:{$\pi_a$}]  {}}
%% child {node [token,fill=red,label=above:{$\pi_i$}]  {}};
%% \end{tikzpicture}
%% \end{center}
%% }
%% \only<3>{
%% \alert{$\Qcal \larr \seq{s,a,i,t} \cup \seq{s,b,f,l,t}$}
%% \begin{center}
%% \begin{tikzpicture}
%% \tikzstyle{every place}=[draw=blue,fill=blue!20,thick,minimum size=70mm]
%% \tikzstyle{every token}=[draw=blue,fill=blue!20,thick,minimum size=10mm]
%% \node[place,label=above:$\Pi$] {}
%% [children are tokens,token distance=30mm]
%% child {node [token,fill=red,label=above:{$\pi_s$}]  {} }
%% child {node [token,fill=red,label=above:{$\pi_{a}$}]  {}}
%% child {node [token,fill=red,label=above:{$\pi_{i}$}]  {}}
%% child {node [token,fill=red,label=above:{$\pi_{b}$}]  {}}
%% child {node [token,fill=red,label=above:{$\pi_{l}$}]  {}};
%% 
%% \end{tikzpicture}
%% \end{center}
%% }
%% 
%% \end{frame}
%% 
%% %%%%%%%%%%%%%%%%%%%%%%%%%%%%%%%%%%%%%%%%%%%%%%%%%%%%%%%%%%%%%%%%%%%%%%%%%%%%%%%%%%%%%%%%%%%%%%%%%%%
%% 
%% \begin{frame}{Exemplo}
%%   \begin{columns}  
%%     \column{.4\textwidth}
%%    \begin{tikzpicture}[auto,thick,sloped,node distance=10mm,scale=0.7]
%%     \tikzstyle{node}=%
%%     [%
%%       minimum size=8pt,%
%%       inner sep=0pt,%
%%       outer sep=0pt,%
%%       ball color=example text.fg,%
%%       circle%
%%     ]
%%         \node[label distance=1mm,node,ball color=red,label=left:{s}]  (S)  {};
%%         \node[node] (B) [right of=S,label=below:{b}] {};
%%         \node[node] (G) [right of=B,label=45:{g}] {};
%%         \node[node,right of=G] (L) [label distance=5mm,label=45:{l}] {};
%%         \node[node,ball color=red,label=above:{t}] (T) [right of=L,label=above:{t}] {};
%%         \node[node] (F) [above of=G,label=45:{f}] {};
%%         \node[node] (E) [above of=F,label=above:{e}] {};
%%         \node[node] (H) [below of=G,label=right:{h}] {};
%%         \node[node] (I) [below of=H,label=right:{i}] {};
%%         \node[node] (C) [above of=S,label=above:{c}] {};
%%         \node[node] (A) [below of=S,label=left:{a}] {};
%%         \node[node] (J) [right of=E,label=above:{j}] {};
%%         \node[node] (D) [left of=E,label=above:{d}] {};
%% 
%%           
%%           \path [-,thick,black] (S) edge (A)
%%                                     edge (B)
%%                                     edge (C)
%%                                 (B) edge (C)
%%                                     edge (D)
%%                                     edge (F)
%%                                     edge (G)
%%                                     edge (H)
%%                                 (D) edge (E)   
%%                                 (A) edge (I)
%%                                 (I) edge (H)
%%                                     edge (T)
%%                                 (E) edge (F)
%%                                     edge (J)
%%                                 (F) edge (G)
%%                                 (F) edge (L)
%%                                 (G) edge (H)
%%                                     edge (L)  
%%                                 (J) edge (L)
%%                                 (L) edge (T)
%%                                 (H) edge (L)
%%                                 (C) edge (D);
%% 
%%           \uncover<2-5>{
%%           \path[-,thick,red,line width=1mm] (S) edge (A)
%%                             (A) edge (I)
%%                             (I) edge (T);
%%           }
%%           \uncover<6-11>{
%%           \path[-,thick,red,line width=1mm] (S) edge (B)
%%                             (B) edge (F)
%%                             (F) edge (L)
%%                             (L) edge (T) ;
%%           }
%%           \uncover<12>{
%%           \path[-,thick,red,line width=1mm] (S) edge (B)
%%                             (B) edge (G)
%%                             (G) edge (L)
%%                             (L) edge (T) ;
%%           }
%% %%           \uncover<5>{
%% %%           \path[-,thick,red,line width=1mm] (S) edge (B)
%% %%                             (B) edge (H)
%% %%                             (H) edge (L)
%% %%                             (L) edge (T) ;
%% %%           }
%% %%           \uncover<6>{
%% %%           \path[-,thick,red,line width=1mm] (S) edge (B)
%% %%                             (B) edge (H)
%% %%                             (H) edge (I)
%% %%                             (I) edge (T) ;
%% %%           }
%% %%           \uncover<7>{
%% %%           \path[-,thick,red,line width=1mm] (S) edge (A)
%% %%                             (A) edge (I)
%% %%                             (I) edge (H)
%% %%                             (H) edge (L)
%% %%                             (L) edge (T);
%% %%           }
%%   \end{tikzpicture}
%% 
%%     \column{.6\textwidth}
%% \begin{tikzpicture}[auto,thick,sloped,scale=0.7]
%%   \tikzstyle{node}=%
%%   [%
%%     minimum size=8pt,%
%%     inner sep=0pt,%
%%     outer sep=0pt,%
%%     ball color=example text.fg,%
%%     circle%
%%   ]
%%   \tikzstyle{leaf}=%
%%   [%
%%     minimum size=10pt,%
%%     inner sep=0pt,%
%%     outer sep=0pt,%
%%     ball color=red,%
%%     circle%
%%   ]
%%   \tikzstyle{ghost}=%
%%   [%
%%     minimum size=8pt,%
%%     inner sep=0pt,%
%%     outer sep=0pt,%
%%     ball color=gray,%
%%     circle%
%%   ]
%% 
%% % Set the overall layout of the tree
%% \tikzstyle{level 1}=[level distance=1cm, sibling distance=5cm]
%% \tikzstyle{level 2}=[level distance=1cm, sibling distance=2cm]
%% 
%% % Define styles for bags and leafs
%% \tikzstyle{bag} = [text width=4em, text centered]
%% \tikzstyle{end} = [circle, minimum width=3pt,fill, inner sep=0pt]
%% \uncover<3-6>{
%%   \node [leaf,label=above:{s}] (raiz){} [] 
%%     child {
%% 	node [node,label=right:{a}] (sa) {} 
%%   	child {
%% 		node [node,label=right:{i}] (sai) {} 
%% 		child 	{
%% 				node [leaf,label=below:{t1}] (sait){} 	
%% 				edge from parent 
%% 			}
%% 		edge from parent 
%% 	}
%% 	edge from parent
%%    };
%% 	\path[opacity=1,color=red,->] [-latex]
%% 	(raiz) edge (sa)
%% 	(sa) edge (sai)
%% 	(sai) edge (sait);
%% }
%% %%particao de pi_s 
%% \uncover<4>{
%%   	\node [leaf,label=above:{s}] (raiz){} [] 
%% 	 child[sibling distance=15mm]{
%% 	 										node[ghost,label=left:{b}] {} 
%% 	 										child[grow=down] {
%% 	 																node {...}
%% 							 										child[grow=down] {
%% 							 											node[leaf,label=below:{t}] {}
%% 		 																edge from parent[dashed]
%% 							 										}
%% 	 																edge from parent[dashed]
%% 	 																}
%% 	 										edge from parent[dashed]
%% 	 }
%%   	 child[sibling distance=15mm] {
%% 				node [node,label=right:{a}] (sa) {} 
%% 			  	child {
%% 						node [node,label=right:{i}] (sai) {} 
%% 						child{
%% 								node [leaf,label=below:{t1}] (sait){} 	
%% 								edge from parent 
%% 						}
%% 				edge from parent 
%% 	}
%% 	edge from parent
%%    }
%%    child[sibling distance=15mm]{
%% 	 										node[ghost,label=right:{c}] {} 
%% 	 										child[grow=down] {
%% 	 																node {...}
%% 							 										child[grow=down] {
%% 							 											node[leaf,label=below:{t}] {}
%% 		 																edge from parent[dashed]
%% 							 										}
%% 	 																edge from parent[dashed]
%% 	 																}
%% 	 										edge from parent[dashed]
%% 	 }
%% ;
%%    
%% 	\path[opacity=1,color=red,->] [-latex]
%% 	(raiz) edge (sa)
%% 	(sa) edge (sai)
%% 	(sai) edge (sait);
%% }
%% 
%% %%particao de pi_sai 
%% \uncover<5>{
%%   	\node [leaf,label=above:{s}] (raiz){} [] 
%%   	 child[sibling distance=15mm] {
%% 				node [node,label=right:{a}] (sa) {} 
%% 			  	child {
%% 						node [node,label=right:{i}] (sai) {} 
%% 						child[grow=down]{
%% 								node [leaf,label=below:{t1}] (sait){} 	
%% 								edge from parent 
%% 						}
%%   					   child{
%% 	 										node[ghost,label=right:{h}] {} 
%% 	 										child[grow=down] {
%% 	 																node {...}
%% 						 										child[grow=down] {
%% 						 											node[leaf,label=below:{t}] {}
%% 	 																edge from parent[dashed]
%% 						 										}
%% 	 																edge from parent[dashed]
%% 	 																}
%% 	 										edge from parent[dashed]
%% 	 					}
%% 						edge from parent 
%% 				}
%% 	edge from parent
%%    };
%%    
%% 	\path[opacity=1,color=red,->] [-latex]
%% 	(raiz) edge (sa)
%% 	(sa) edge (sai)
%% 	(sai) edge (sait);
%% }
%% 
%% %caminhos s,b,f,l,t
%% \uncover<7>{
%%   \node [leaf,label=above:{s}] (raiz) {} [] 
%%     	child[sibling distance=15mm] {
%% 		node [node,label=above:{a}] {} [->]
%% 		child {
%% 			node [node,label=right:{i}] {} [->]
%% 			child 	{
%% 					node [leaf,label=below:{t1}] {} [->]	
%% 					edge from parent 
%% 				}
%% 			edge from parent 
%% 			}
%% 		edge from parent 
%%         }	
%% 	child[sibling distance=15mm] {
%% 		node [node,label=above:{b}] (sb){} 	
%%    		child {
%% 			node [node,label=right:{f}] (sbf) {} 	
%%    			child {
%% 				node [node,label=right:{l}] (sbfl) {} 
%% 				child{
%% 					node [leaf,label=below:{t2}] (sbflt) {} 
%% 					edge from parent 
%% 				}
%% 				edge from parent 
%% 			}
%% 			edge from parent 
%% 		}	
%% 		edge from parent 
%% 	};
%% 	\path[opacity=1,color=red,->] [-latex]
%% 	(raiz) edge (sb)
%% 	(sb) edge (sbf)
%% 	(sbf) edge (sbfl)
%% 	(sbfl) edge (sbflt);
%% }
%% 
%% %particao pi_s
%% \uncover<8>{
%%   \node [leaf,label=above:{s}] (raiz) {} [] 
%%     	child[sibling distance=15mm] {
%% 		node [node,label=above:{a}] {} [->]
%% 		child {
%% 			node [node,label=right:{i}] {} [->]
%% 			child 	{
%% 					node [leaf,label=below:{t1}] {} [->]	
%% 					edge from parent 
%% 				}
%% 			edge from parent 
%% 			}
%% 		edge from parent 
%%         }	
%% 	child[sibling distance=15mm] {
%% 		node [node,label=right:{b}] (sb){} 	
%%    		child {
%% 			node [node,label=right:{f}] (sbf) {} 	
%%    			child {
%% 				node [node,label=right:{l}] (sbfl) {} 
%% 				child{
%% 					node [leaf,label=below:{t2}] (sbflt) {} 
%% 					edge from parent 
%% 				}
%% 				edge from parent 
%% 			}
%% 			edge from parent 
%% 		}	
%% 		edge from parent 
%% 	}
%%    child[sibling distance=15mm]{
%% 	 										node[ghost,label=right:{c}] {} 
%% 	 										child[grow=down] {
%% 	 																node {...}
%% 							 										child[grow=down] {
%% 							 											node[leaf,label=below:{t}] {}
%% 		 																edge from parent[dashed]
%% 							 										}
%% 	 																edge from parent[dashed]
%% 	 																}
%% 	 										edge from parent[dashed]
%% 	 }	
%% 	;
%% 	\path[opacity=1,color=red,->] [-latex]
%% 	(raiz) edge (sb)
%% 	(sb) edge (sbf)
%% 	(sbf) edge (sbfl)
%% 	(sbfl) edge (sbflt);
%% }
%% 
%% %particao pi_sb
%% \uncover<9>{
%%   \node [leaf,label=above:{s}] (raiz) {} [] 
%%     	child[sibling distance=40mm] {
%% 		node [node,label=above:{a}] {} [->]
%% 		child {
%% 			node [node,label=left:{i}] {} [->]
%% 			child 	{
%% 					node [leaf,label=below:{t1}] {} [->]	
%% 					edge from parent 
%% 				}
%% 			edge from parent 
%% 			}
%% 		edge from parent 
%%         }	
%% 	child[sibling distance=40mm] {
%% 			node [node,label=above:{b}] (sb){} 	
%% 			child[sibling distance=15mm]{
%% 	 										node[ghost,label=above:{c}] {} 
%% 	 										child[grow=down] {
%% 	 																node {...}
%% 							 										child[grow=down] {
%% 							 											node[leaf,label=below:{t}] {}
%% 		 																edge from parent[dashed]
%% 							 										}
%% 	 																edge from parent[dashed]
%% 	 																}
%% 	 										edge from parent[dashed]
%% 	 		}	 		
%% 			child[sibling distance=15mm]{
%% 	 										node[ghost,label=right:{d}] {} 
%% 	 										child[grow=down] {
%% 	 																node {...}
%% 							 										child[grow=down] {
%% 							 											node[leaf,label=below:{t}] {}
%% 		 																edge from parent[dashed]
%% 							 										}
%% 	 																edge from parent[dashed]
%% 	 																}
%% 	 										edge from parent[dashed]
%% 	 		}	
%% 	  		child[sibling distance=15mm] {
%% 						node [node,label=right:{f}] (sbf) {} 	
%% 			   			child {
%% 							node [node,label=right:{l}] (sbfl) {} 
%% 							child{
%% 								node [leaf,label=below:{t2}] (sbflt) {} 
%% 								edge from parent 
%% 							}
%% 							edge from parent 
%% 						}
%% 						edge from parent 
%% 			}	 			
%% 			child[sibling distance=15mm]{
%% 	 										node[ghost,label=right:{g}] {} 
%% 	 										child[grow=down] {
%% 	 																node {...}
%% 							 										child[grow=down] {
%% 							 											node[leaf,label=below:{t}] {}
%% 		 																edge from parent[dashed]
%% 							 										}
%% 	 																edge from parent[dashed]
%% 	 																}
%% 	 										edge from parent[dashed]
%% 	 		}		 		
%% 			child[sibling distance=15mm]{
%% 	 										node[ghost,label=right:{h}] {} 
%% 	 										child[grow=down] {
%% 	 																node {...}
%% 							 										child[grow=down] {
%% 							 											node[leaf,label=below:{t}] {}
%% 		 																edge from parent[dashed]
%% 							 										}
%% 	 																edge from parent[dashed]
%% 	 																}
%% 	 										edge from parent[dashed]
%% 	 		}	
%% 		edge from parent 
%% 	};
%% 	\path[opacity=1,color=red,->] [-latex]
%% 	(raiz) edge (sb)
%% 	(sb) edge (sbf)
%% 	(sbf) edge (sbfl)
%% 	(sbfl) edge (sbflt);
%% }
%% 
%% %particao pi_sbf
%% \uncover<10>{
%%   \node [leaf,label=above:{s}] (raiz) {} [] 
%%     	child[sibling distance=40mm] {
%% 		node [node,label=above:{a}] {} [->]
%% 		child {
%% 			node [node,label=left:{i}] {} [->]
%% 			child 	{
%% 					node [leaf,label=below:{t1}] {} [->]	
%% 					edge from parent 
%% 				}
%% 			edge from parent 
%% 			}
%% 		edge from parent 
%%         }	
%% 	child[sibling distance=40mm] {
%% 		node [node,label=above:{b}] (sb){} 	
%%    		child {
%% 						node [node,label=right:{f}] (sbf) {} 	
%% 						child[sibling distance=15mm]{
%% 				 										node[ghost,label=right:{e}] {} 
%% 				 										child[grow=down] {
%% 				 																node {...}
%% 										 										child[grow=down] {
%% 										 											node[leaf,label=below:{t}] {}
%% 					 																edge from parent[dashed]
%% 										 										}
%% 				 																edge from parent[dashed]
%% 				 																}
%% 				 										edge from parent[dashed]
%% 				 		}		 		
%% 						child[sibling distance=15mm] {
%% 									node [node,label=right:{l}] (sbfl) {} 
%% 									child{
%% 										node [leaf,label=below:{t2}] (sbflt) {} 
%% 										edge from parent 
%% 									}
%% 									edge from parent 
%% 						}
%% 						child[sibling distance=15mm]{
%% 				 										node[ghost,label=right:{g}] {} 
%% 				 										child[grow=down] {
%% 				 																node {...}
%% 										 										child[grow=down] {
%% 										 											node[leaf,label=below:{t}] {}
%% 					 																edge from parent[dashed]
%% 										 										}
%% 				 																edge from parent[dashed]
%% 				 																}
%% 				 										edge from parent[dashed]
%% 				 		}		 				 		
%% 						edge from parent 
%% 			}	
%% 			edge from parent 
%% 	};
%% 	\path[opacity=1,color=red,->] [-latex]
%% 	(raiz) edge (sb)
%% 	(sb) edge (sbf)
%% 	(sbf) edge (sbfl)
%% 	(sbfl) edge (sbflt);
%% }
%% 
%% %particao pi_sbfl
%% \uncover<11>{
%% 
%%   \node [leaf,label=above:{s}] (raiz) {} [] 
%% 	child[sibling distance=40mm] {
%% 			node [node,label=above:{a}] {} [->]
%% 			child {
%% 				node [node,label=right:{i}] {} [->]
%% 				child 	{
%% 						node [leaf,label=below:{t1}] {} [->]	
%% 						edge from parent 
%% 					}
%% 				edge from parent 
%% 				}
%% 			edge from parent 
%% 	}	
%% 	child[sibling distance=40mm] {
%% 		node [node,label=above:{b}] (sb){} 	
%% 		child[sibling distance=15mm] {
%% 					node [node,label=right:{f}] (sbf) {} 	
%% 					child {
%% 								node [node,label=right:{l}] (sbfl) {} 
%% 								child{
%% 				 										node[ghost,label=right:{g}] {} 
%% 				 										child[grow=down] {
%% 				 																node {...}
%% 										 										child[grow=down] {
%% 										 											node[leaf,label=below:{t}] {}
%% 					 																edge from parent[dashed]
%% 										 										}
%% 				 																edge from parent[dashed]
%% 				 																}
%% 				 										edge from parent[dashed]
%% 								}								
%% 								child[level distance=20mm,grow=down]{
%% 									node [leaf,label=below:{t2}] (sbflt) {} 
%% 									edge from parent 
%% 								}
%% 								child{
%% 				 										node[ghost,label=right:{h}] {} 
%% 				 										child[grow=down] {
%% 				 																node {...}
%% 										 										child[grow=down] {
%% 										 											node[leaf,label=below:{t}] {}
%% 					 																edge from parent[dashed]
%% 										 										}
%% 				 																edge from parent[dashed]
%% 				 																}
%% 				 										edge from parent[dashed]
%% 								}
%% 								child{
%% 				 										node[ghost,label=right:{j}] {} 
%% 				 										child[grow=down] {
%% 				 																node {...}
%% 										 										child[grow=down] {
%% 										 											node[leaf,label=below:{t}] {}
%% 					 																edge from parent[dashed]
%% 										 										}
%% 				 																edge from parent[dashed]
%% 				 																}
%% 				 										edge from parent[dashed]
%% 								}
%% 								edge from parent
%% 				}
%% 					edge from parent 
%% 		}	
%% 		edge from parent 
%% 	};
%% 	
%% 	\path[opacity=1,color=red,->] [-latex]
%% 	(raiz) edge (sb)
%% 	(sb) edge (sbf)
%% 	(sbf) edge (sbfl)
%% 	(sbfl) edge (sbflt);
%% 
%% }
%% \end{tikzpicture}
%% \end{columns}
%% \end{frame}

%%%%%%%%%%%%%%%%%%%%%%%%%%%%%%%%%%%%%%%%%%%%%%%%%%%%%%%%%%%%%%%%

\subsection*{M�todo de Yen}
\begin{frame}{M�todo \YenGenerico}
\begin{algoritmo}

\textbf{M�todo} \YenGenerico{} $(V,A,c,s,t,k)$ %\\[2mm]
   
0\x $\Pi \larr \{\mbox{conjunto dos caminhos de $s$ a~$t$}\}$

1\x $\Qcal \larr \emptyset $

2\x \para{} $i=1,\ldots,k$ \faca %\\[1mm]

3\xx  $\Lcal  \larr \{P_{\pi} : P_{\pi} \ \mbox{� caminho m�nimo na parte $\pi$
de~$\Pi$}\}$

4\xx  $P_i \larr \mbox{caminho de custo m�nimo em $\Lcal$}$ %\\[1mm]

5\xx  $\Qcal \larr \Qcal \cup \{P_i\}$

6\xx  $\Pi \larr \AtualizeGenerico~(V,A,s,t,\Qcal)$

7\x \devolva{} $\seq{P_1,\ldots,P_k}$

\end{algoritmo}

\end{frame}

\begin{frame}{\AtualizeGenerico}

\begin{algoritmo}

\textbf{Algoritmo} \AtualizeGenerico{} $(V,A,s,t,\Qcal)$ %\\[2mm]
   
0\x $\Pi \larr \emptyset \quad \quad \Pcal \larr \Pcal_{st} - \Qcal$

1\x $(N,E,f) \larr$ �rvore dos prefixos de $\Qcal$

2\x \para{} \cada{} $u \in N$ que n�o � uma folha \faca %\\[1mm]

3\xx  $\pi_u \larr \{\mbox{caminhos em $\Pcal$ com prefixo $f(R_u)$}$

\xxxxx \quad e que n�o possuem arcos em $A_u \}$

4\xx  $\Pi \larr  \Pi \cup \{\pi_u\}$ %\\[1mm]

5\x \devolva{} $\Pi$

\end{algoritmo}\end{frame}

\begin{frame}{Exemplo}
  \begin{columns}  
    \column{.4\textwidth}
  \begin{tikzpicture}[auto,thick,sloped,node distance=10mm,scale=0.7]
    \tikzstyle{node}=%
    [%
      minimum size=8pt,%
      inner sep=0pt,%
      outer sep=0pt,%
      ball color=example text.fg,%
      circle%
    ]
        \node[node,ball color=red]  (S) [label=left:{s}] {};
        \node[node] (B) [right of=S,label=below:{b}] {};
        \node[node] (G) [right of=B,label=45:{g}] {};
        \node[node] (L) [right of=G,label=45:{l}] {};
        \node[node,ball color=red,label=above:{t}] (T) [right of=L,label=above:{t}] {};
        \node[node] (F) [above of=G,label=45:{f}] {};
        \node[node] (E) [above of=F,label=above:{e}] {};
        \node[node] (H) [below of=G,label=right:{h}] {};
        \node[node] (I) [below of=H,label=right:{i}] {};
        \node[node] (C) [above of=S,label=above:{c}] {};
        \node[node] (A) [below of=S,label=left:{a}] {};
        \node[node] (J) [right of=E,label=above:{j}] {};
        \node[node] (D) [left of=E,label=above:{d}] {};

          
          \path [-,thick,black] (S) edge (A)
                                    edge (B)
                                    edge (C)
                                (B) edge (C)
                                    edge (D)
                                    edge (F)
                                    edge (G)
                                    edge (H)
                                (D) edge (E)   
                                (A) edge (I)
                                (I) edge (H)
                                    edge (T)
                                (E) edge (F)
                                    edge (J)
                                (F) edge (G)
                                (F) edge (L)
                                (G) edge (H)
                                    edge (L)  
                                (J) edge (L)
                                (L) edge (T)
                                (H) edge (L)
                                (C) edge (D);
        \uncover<3>{  
	\path[-,thick,red,line width=1mm] (S) edge (B)
                            (B) edge (H)
                            (H) edge (L)
                            (L) edge (T) ;
	}
           \end{tikzpicture}

    \column{.6\textwidth}
\begin{tikzpicture}[auto,thick,sloped,scale=0.7]
  \tikzstyle{node}=%
  [%
    minimum size=8pt,%
    inner sep=0pt,%
    outer sep=0pt,%
    ball color=example text.fg,%
    circle%
  ]
  \tikzstyle{leaf}=%
  [%
    minimum size=10pt,%
    inner sep=0pt,%
    outer sep=0pt,%
    ball color=red,%
    circle%
  ]

% Set the overall layout of the tree
\tikzstyle{level 1}=[level distance=1.5cm, sibling distance=6cm]
\tikzstyle{level 2}=[level distance=2cm, sibling distance=2cm]

% Define styles for bags and leafs
\tikzstyle{bag} = [text width=4em, text centered]
\tikzstyle{end} = [circle, minimum width=3pt,fill, inner sep=0pt]
\only<1>{
 \node[draw,color=red] at (3.5,0) {$f(R_u) = \seq{s,b}$};
 \node[color=blue,rectangle,draw,dashed,opacity=.5,minimum width=10pt] at (2.5,-2.6)  (Au) 
          {$A_u$ \mbox{ }\mbox{ }\mbox{ }\mbox{ }\mbox{ }\mbox{ }\mbox{ }\mbox{ }\mbox{ }\mbox{ }\mbox{ }\mbox{ }\mbox{ }\mbox{ }\mbox{ }\mbox{ }\mbox{ }\mbox{ }\mbox{ }\mbox{ }\mbox{ }\mbox{ }\mbox{ }\mbox{ }\mbox{ }\mbox{ }\mbox{ }};
  \node [leaf,label=above:{s}] (raiz) {} [] 
    child {
	node [node,label=above:{a}] {} [->]
	child {
		node [node,label=right:{i}] {} 
		child 	{
				node [leaf,label=below:{t1}] {} 	
				edge from parent 
			}
		edge from parent 
		}
	edge from parent 
        }	
	child {
		node [node,label=right:{$f(u)$=b}] (sb) {} 	
   		child {
			node [node,label=left:{f},ball color=blue](sbf) {} 	
   			child {
				node [node,label=left:{l}] {} [->]
				child{
					node [leaf,label=below:{t2}] {} 
					edge from parent 
				}
				edge from parent 
			}
			edge from parent 	
	        }	
	  	child {
			node [node,label=right:{g},ball color=blue] (sbg) {} 	
   			child {
				node [node,label=left:{l}] {} [->]
				child{
					node [leaf,label=below:{t3}] {} 
					edge from parent
				}
				edge from parent 
			}
			edge from parent 
		}		
		edge from parent
 	};
	\path[opacity=1,color=red,->] [-latex]
	(raiz) edge (sb);
	\path[opacity=1,color=blue,->] [-latex]
	(sb) edge (sbf);
	\path[opacity=1,color=blue,->] [-latex]
	(sb) edge (sbg);
}

\only<2>{
 \node[draw,color=red] at (3.5,0) {$f(R_u) = \seq{s,b}$};
% \node[color=blue,rectangle,draw,dashed,opacity=.5,minimum width=10pt] at (2.5,-2.2)  (Au) 
%          {$A_u$ \mbox{ }\mbox{ }\mbox{ }\mbox{ }\mbox{ }\mbox{ }\mbox{ }\mbox{ }\mbox{ }\mbox{ }\mbox{ }\mbox{ }\mbox{ }\mbox{ }\mbox{ }\mbox{ }\mbox{ }\mbox{ }\mbox{ }\mbox{ }\mbox{ }\mbox{ }\mbox{ }\mbox{ }\mbox{ }\mbox{ }\mbox{ }};
  \node [leaf,label=above:{s}] (raiz) {} [] 
    child {
	node [node,label=above:{a}] {} [->]
	child {
		node [node,label=right:{i}] {} 
		child 	{
				node [leaf,label=below:{t1}] {} 	
				edge from parent 
			}
		edge from parent 
		}
	edge from parent 
        }	
	child {
		node [node,label=right:{$f(u)$=b}] (sb) {} 	
   		child  [sibling distance=1.5cm]{
			node [node,label=left:{f},ball color=blue](sbf) {} 	
   			child {
				node [node,label=left:{l}] {} [->]
				child{
					node [leaf,label=below:{t2}] {} 
					edge from parent 
				}
				edge from parent 
			}
			edge from parent 	
	        }	
	  	child  [sibling distance=1.5cm]{
			node [node,label=right:{g},ball color=blue] (sbg) {} 	
   			child {
				node [node,label=left:{l}] {} [->]
				child{
					node [leaf,label=below:{t3}] {} 
					edge from parent
				}
				edge from parent 
			}
			edge from parent 
		}		
		child [sibling distance=1.5cm]{
			node [node,label=right:{c},ball color=gray] (sbc) {} [] 
			child{
				node [node,ball color=gray] {\ldots} [->]	
				child{
					node [node,label=below:{t},ball color=red] {} []	
		  			edge from parent 
				}
   				edge from parent 
			}
   			edge from parent 
		}	
		child [sibling distance=1.5cm]{
			node [node,label=right:{d},ball color=gray] (sbd) {} []
			child{
				node [node,ball color=gray] {\ldots} [->]	
				child{
					node [node,label=below:{t},ball color=red] {} []	
		  			edge from parent 
				}
   				edge from parent 
			}
   			edge from parent 
		}	
		child [sibling distance=1.5cm]{
			node [node,label=right:{h},ball color=gray] (sbh) {} []
			child{
				node [node,ball color=gray] {\ldots} [->]	
				child{
					node [node,label=below:{t},ball color=red] {} []	
		  			edge from parent 
				}
   				edge from parent 
			}
   			edge from parent 
		}	
	edge from parent
 	};
	\path[opacity=1,color=red,->] [-latex]
	(raiz) edge (sb);
	\path[opacity=1,color=blue,->] [-latex]
	(sb) edge (sbf);
	\path[opacity=1,color=blue,->] [-latex]
	(sb) edge (sbg);
	\path[dashed,opacity=1,color=gray,->] [-latex]
	(sb) edge (sbc);
	\path[dashed,opacity=1,color=gray,->] [-latex]
	(sb) edge (sbd);
	\path[dashed,opacity=1,color=gray,->] [-latex]
	(sb) edge (sbh);
	\draw[opacity=0.5,dashed,color=gray,fill] (2.5,-8.5) rectangle (6.8,-2.3);	
	\node at (6.5,-2.5) {$\pi_u$};
}
\only<3>{
 \node[draw,color=red] at (3.5,0) {$f(R_u) = \seq{s,b}$};
% \node[color=blue,rectangle,draw,dashed,opacity=.5,minimum width=10pt] at (2.5,-2.2)  (Au) 
%          {$A_u$ \mbox{ }\mbox{ }\mbox{ }\mbox{ }\mbox{ }\mbox{ }\mbox{ }\mbox{ }\mbox{ }\mbox{ }\mbox{ }\mbox{ }\mbox{ }\mbox{ }\mbox{ }\mbox{ }\mbox{ }\mbox{ }\mbox{ }\mbox{ }\mbox{ }\mbox{ }\mbox{ }\mbox{ }\mbox{ }\mbox{ }\mbox{ }};
  \node [leaf,label=above:{s}] (raiz) {} [] 
    child {
	node [node,label=above:{a}] {} [->]
	child {
		node [node,label=right:{i}] {} 
		child 	{
				node [leaf,label=below:{t1}] {} 	
				edge from parent 
			}
		edge from parent 
		}
	edge from parent 
        }	
	child {
		node [node,label=right:{$f(u)$=b}] (sb) {} 	
   		child  [sibling distance=1.5cm]{
			node [node,label=left:{f},ball color=blue](sbf) {} 	
   			child {
				node [node,label=left:{l}] {} [->]
				child{
					node [leaf,label=below:{t2}] {} 
					edge from parent 
				}
				edge from parent 
			}
			edge from parent 	
	        }	
	  	child  [sibling distance=1.5cm]{
			node [node,label=right:{g},ball color=blue] (sbg) {} 	
   			child {
				node [node,label=left:{l}] {} [->]
				child{
					node [leaf,label=below:{t3}] {} 
					edge from parent
				}
				edge from parent 
			}
			edge from parent 
		}		
		child [sibling distance=1.5cm]{
			node [node,label=right:{c},ball color=gray] (sbc) {} []
			child{
				node [node,ball color=gray] {\ldots} [->]	
				child{
					node [node,label=below:{t},ball color=red] {} []	
		  			edge from parent 
				}
   				edge from parent 
			}
   			edge from parent 
		}	
		child [sibling distance=1.5cm]{
			node [node,label=right:{d},ball color=gray] (sbd) {} []
			child{
				node [node,ball color=gray] {\ldots} [->]	
				child{
					node [node,label=below:{t},ball color=red] {} []	
		  			edge from parent 
				}
   				edge from parent 
			}
   			edge from parent 
		}	
		child [sibling distance=1.5cm]{
			node [node,label=right:{h},ball color=gray] (sbh) {} []
			child{
				node [node,label=right:{l},ball color=gray] (sbhl) {} []	
				child{
					node [node,label=below:{t},ball color=red] (sbhlt) {} []	
		  			edge from parent 
				}
   				edge from parent 
			}
   			edge from parent 
		}	
	edge from parent
 	};
	\path[opacity=1,color=red,->] [-latex]
	(raiz) edge (sb);
	\path[opacity=1,color=blue,->] [-latex]
	(sb) edge (sbf);
	\path[opacity=1,color=blue,->] [-latex]
	(sb) edge (sbg);
	\path[dashed,opacity=1,color=gray,->] [-latex]
	(sb) edge (sbc);
	\path[dashed,opacity=1,color=gray,->] [-latex]
	(sb) edge (sbd);
	\path[dashed,opacity=1,color=gray,->] [-latex]
	(sb) edge (sbh);
	\draw[opacity=0.5,dashed,color=gray,fill] (2.5,-8.5) rectangle (6.8,-2.3);	
	\node at (6.5,-2.5) {$\pi_u$};
	\path[opacity=1,color=red,->] [-latex]
	(raiz) edge (sb)
	(sb) edge (sbh)
	(sbh) edge (sbhl)
	(sbhl) edge (sbhlt);

}
\end{tikzpicture}
\end{columns}
\end{frame}

%%%%%%%%%%%%%%%%%%%%%%%%%%%%%%%%%%%%%%%%%%%%%%%%%%%%%%%%%%%%%%%%%%%%%%%%%%%%%%%%%%%%%%%%%%%%%%

\begin{frame}{Exemplo}
  \begin{columns}  
    \column{.4\textwidth}
   \begin{tikzpicture}[auto,thick,sloped,node distance=10mm,scale=0.7]
    \tikzstyle{node}=%
    [%
      minimum size=8pt,%
      inner sep=0pt,%
      outer sep=0pt,%
      ball color=example text.fg,%
      circle%
    ]
        \node[label distance=1mm,node,ball color=red,label=left:{s}]  (S)  {};
        \node[node] (B) [right of=S,label=below:{b}] {};
        \node[node] (G) [right of=B,label=45:{g}] {};
        \node[node,right of=G] (L) [label distance=5mm,label=45:{l}] {};
        \node[node,ball color=red,label=above:{t}] (T) [right of=L,label=above:{t}] {};
        \node[node] (F) [above of=G,label=45:{f}] {};
        \node[node] (E) [above of=F,label=above:{e}] {};
        \node[node] (H) [below of=G,label=right:{h}] {};
        \node[node] (I) [below of=H,label=right:{i}] {};
        \node[node] (C) [above of=S,label=above:{c}] {};
        \node[node] (A) [below of=S,label=left:{a}] {};
        \node[node] (J) [right of=E,label=above:{j}] {};
        \node[node] (D) [left of=E,label=above:{d}] {};

          
          \path [-,thick,black] (S) edge (A)
                                    edge (B)
                                    edge (C)
                                (B) edge (C)
                                    edge (D)
                                    edge (F)
                                    edge (G)
                                    edge (H)
                                (D) edge (E)   
                                (A) edge (I)
                                (I) edge (H)
                                    edge (T)
                                (E) edge (F)
                                    edge (J)
                                (F) edge (G)
                                (F) edge (L)
                                (G) edge (H)
                                    edge (L)  
                                (J) edge (L)
                                (L) edge (T)
                                (H) edge (L)
                                (C) edge (D);

          \onslide<2-5>{
          \path[-,thick,red,line width=1mm] (S) edge (A)
                            (A) edge (I)
                            (I) edge (T);
          }
          \only<7-11>{
          \path[-,thick,red,line width=1mm] (S) edge (B)
                            (B) edge (F)
                            (F) edge (L)
                            (L) edge (T) ;
          }
          \uncover<12>{
          \path[-,thick,red,line width=1mm] (S) edge (B)
                            (B) edge (G)
                            (G) edge (L)
                            (L) edge (T) ;
          }
%%           \uncover<5>{
%%           \path[-,thick,red,line width=1mm] (S) edge (B)
%%                             (B) edge (H)
%%                             (H) edge (L)
%%                             (L) edge (T) ;
%%           }
%%           \uncover<6>{
%%           \path[-,thick,red,line width=1mm] (S) edge (B)
%%                             (B) edge (H)
%%                             (H) edge (I)
%%                             (I) edge (T) ;
%%           }
%%           \uncover<7>{
%%           \path[-,thick,red,line width=1mm] (S) edge (A)
%%                             (A) edge (I)
%%                             (I) edge (H)
%%                             (H) edge (L)
%%                             (L) edge (T);
%%           }
  \end{tikzpicture}

    \column{.6\textwidth}
\begin{tikzpicture}[auto,thick,sloped,scale=0.7]
  \tikzstyle{node}=%
  [%
    minimum size=8pt,%
    inner sep=0pt,%
    outer sep=0pt,%
    ball color=example text.fg,%
    circle%
  ]
  \tikzstyle{leaf}=%
  [%
    minimum size=10pt,%
    inner sep=0pt,%
    outer sep=0pt,%
    ball color=red,%
    circle%
  ]
  \tikzstyle{ghost}=%
  [%
    minimum size=8pt,%
    inner sep=0pt,%
    outer sep=0pt,%
    ball color=gray,%
    circle%
  ]

% Set the overall layout of the tree
\tikzstyle{level 1}=[level distance=1cm, sibling distance=5cm]
\tikzstyle{level 2}=[level distance=1cm, sibling distance=2cm]

% Define styles for bags and leafs
\tikzstyle{bag} = [text width=4em, text centered]
\tikzstyle{end} = [circle, minimum width=3pt,fill, inner sep=0pt]
\only<3>{
  \node [leaf,label=above:{s}] (raiz){} [] 
    child {
	node [node,label=right:{a}] (sa) {} 
  	child {
		node [node,label=right:{i}] (sai) {} 
		child 	{
				node [leaf,label=below:{t1}] (sait){} 	
				edge from parent 
			}
		edge from parent 
	}
	edge from parent
   };
	\path[opacity=1,color=red,->] [-latex]
	(raiz) edge (sa)
	(sa) edge (sai)
	(sai) edge (sait);
}
%%particao de pi_s 
\only<4>{
  	\node [leaf,label=above:{s}] (raiz){} [] 
	 child[sibling distance=15mm]{
	 										node[ghost,label=left:{b}] {} 
	 										child[grow=down] {
	 																node {...}
							 										child[grow=down] {
							 											node[leaf,label=below:{t}] {}
		 																edge from parent[dashed]
							 										}
	 																edge from parent[dashed]
	 																}
	 										edge from parent[dashed]
	 }
  	 child[sibling distance=15mm] {
				node [node,label=right:{a}] (sa) {} 
			  	child {
						node [node,label=right:{i}] (sai) {} 
						child{
								node [leaf,label=below:{t1}] (sait){} 	
								edge from parent 
						}
				edge from parent 
	}
	edge from parent
   }
   child[sibling distance=15mm]{
	 										node[ghost,label=right:{c}] {} 
	 										child[grow=down] {
	 																node {...}
							 										child[grow=down] {
							 											node[leaf,label=below:{t}] {}
		 																edge from parent[dashed]
							 										}
	 																edge from parent[dashed]
	 																}
	 										edge from parent[dashed]
	 }
;
   
	\path[opacity=1,color=red,->] [-latex]
	(raiz) edge (sa)
	(sa) edge (sai)
	(sai) edge (sait);
}

%%particao de pi_s minimo
\only<5>{
  	\node [leaf,label=above:{s}] (raiz){} [] 
	 child[sibling distance=15mm]{
	 										node[ghost,label=left:{b}] {} 
	 										child[sibling distance=15mm,grow=down] {
	 																node[ghost,label=left:{f}] {}
	 																child[grow=down]{
	 																	node[ghost,label=left:{l}] {}
	 																	child[grow=down] {
							 												node[leaf,label=below:{t}] {}
		 																	edge from parent[->]
							 											}	
							 											edge from parent[->]
	 																}
	 																edge from parent[->]
	 										}
	 										edge from parent[->]
	 }
  	 child[sibling distance=15mm] {
				node [node,label=right:{a}] (sa) {} 
			  	child {
						node [node,label=right:{i}] (sai) {} 
						child{
								node [leaf,label=below:{t1}] (sait){} 	
								edge from parent 
						}
				edge from parent 
	}
	edge from parent
   }
	child[sibling distance=15mm]{
	 										node[ghost,label=right:{c}] {} 
	 										child[grow=down] {
	 																node {...}
							 										child[grow=down] {
							 											node[leaf,label=below:{t}] {}
		 																edge from parent[dashed]
							 										}
	 																edge from parent[dashed]
	 																}
	 										edge from parent[dashed]
		 }   
   ;
   
	\path[opacity=1,color=red,->] [-latex]
	(raiz) edge (sa)
	(sa) edge (sai)
	(sai) edge (sait);
}


%%particao de pi_sai 
\uncover<6>{
  	\node [leaf,label=above:{s}] (raiz){} [] 
  	 child[sibling distance=15mm] {
				node [node,label=right:{a}] (sa) {} 
			  	child {
						node [node,label=right:{i}] (sai) {} 
						child[grow=down]{
								node [leaf,label=below:{t1}] (sait){} 	
								edge from parent 
						}
  					   child{
	 										node[ghost,label=right:{h}] {} 
	 										child[grow=down] {
	 																node[ghost,label=right:{l}] {}
						 											child[grow=down] {
						 												node[leaf,label=below:{t}] {}
	 																	edge from parent[->]
							 										}
	 																edge from parent[->]
	 										}
	 										edge from parent[->]
	 					}
						edge from parent 
				}
	edge from parent
   };
   
	\path[opacity=1,color=red,->] [-latex]
	(raiz) edge (sa)
	(sa) edge (sai)
	(sai) edge (sait);
}

%caminhos s,b,f,l,t
\uncover<7>{
  \node [leaf,label=above:{s}] (raiz) {} [] 
    	child[sibling distance=15mm] {
		node [node,label=above:{a}] {} [->]
		child {
			node [node,label=right:{i}] {} [->]
			child 	{
					node [leaf,label=below:{t1}] {} [->]	
					edge from parent 
				}
			edge from parent 
			}
		edge from parent 
        }	
	child[sibling distance=15mm] {
		node [node,label=above:{b}] (sb){} 	
   		child {
			node [node,label=right:{f}] (sbf) {} 	
   			child {
				node [node,label=right:{l}] (sbfl) {} 
				child{
					node [leaf,label=below:{t2}] (sbflt) {} 
					edge from parent 
				}
				edge from parent 
			}
			edge from parent 
		}	
		edge from parent 
	};
	\path[opacity=1,color=red,->] [-latex]
	(raiz) edge (sb)
	(sb) edge (sbf)
	(sbf) edge (sbfl)
	(sbfl) edge (sbflt);
}

%particao pi_s
\uncover<8>{
  \node [leaf,label=above:{s}] (raiz) {} [] 
    	child[sibling distance=15mm] {
		node [node,label=above:{a}] {} [->]
		child {
			node [node,label=right:{i}] {} [->]
			child 	{
					node [leaf,label=below:{t1}] {} [->]	
					edge from parent 
				}
			edge from parent 
			}
		edge from parent 
        }	
	child[sibling distance=15mm] {
		node [node,label=right:{b}] (sb){} 	
   		child {
			node [node,label=right:{f}] (sbf) {} 	
   			child {
				node [node,label=right:{l}] (sbfl) {} 
				child{
					node [leaf,label=below:{t2}] (sbflt) {} 
					edge from parent 
				}
				edge from parent 
			}
			edge from parent 
		}	
		edge from parent 
	}
   child[sibling distance=15mm]{
	 										node[ghost,label=right:{c}] {} 
	 										child[grow=down] {
	 																node[ghost,label=right:{b}] {}
							 										child[grow=down]{
		 																node[ghost,label=right:{f}] {}
		 																child[grow=down] {
							 												node[ghost,label=right:{l}] {}
							 												child[grow=down] {
							 													node[leaf,label=below:{t}] {}
		 																		edge from parent[->]
							 												}
			 																edge from parent[->]
							 											}
							 										}
	 																edge from parent[->]
	 																}
	 										edge from parent[->]
	 }	
	;
	\path[opacity=1,color=red,->] [-latex]
	(raiz) edge (sb)
	(sb) edge (sbf)
	(sbf) edge (sbfl)
	(sbfl) edge (sbflt);
}

%particao pi_sb
\uncover<9>{
  \node [leaf,label=above:{s}] (raiz) {} [] 
    	child[sibling distance=40mm] {
		node [node,label=above:{a}] {} [->]
		child {
			node [node,label=left:{i}] {} [->]
			child 	{
					node [leaf,label=below:{t1}] {} [->]	
					edge from parent 
				}
			edge from parent 
			}
		edge from parent 
        }	
	child[sibling distance=40mm] {
			node [node,label=above:{b}] (sb){} 	
			child[sibling distance=15mm]{
	 										node[ghost,label=above:{c}] {} 
	 										child[grow=down] {
	 																node {...}
							 										child[grow=down] {
							 											node[leaf,label=below:{t}] {}
		 																edge from parent[dashed]
							 										}
	 																edge from parent[dashed]
	 																}
	 										edge from parent[dashed]
	 		}	 		
			child[sibling distance=15mm]{
	 										node[ghost,label=right:{d}] {} 
	 										child[grow=down] {
	 																node {...}
							 										child[grow=down] {
							 											node[leaf,label=below:{t}] {}
		 																edge from parent[dashed]
							 										}
	 																edge from parent[dashed]
	 																}
	 										edge from parent[dashed]
	 		}	
	  		child[sibling distance=15mm] {
						node [node,label=right:{f}] (sbf) {} 	
			   			child {
							node [node,label=right:{l}] (sbfl) {} 
							child{
								node [leaf,label=below:{t2}] (sbflt) {} 
								edge from parent 
							}
							edge from parent 
						}
						edge from parent 
			}	 			
			child[sibling distance=15mm]{
	 										node[ghost,label=right:{g}] {} 
	 										child[grow=down] {
	 																node[ghost,label=right:{l}] {}
							 										child[grow=down] {
							 											node[leaf,label=below:{t}] {}
		 																edge from parent[->]
							 										}
	 																edge from parent[->]
	 																}
	 										edge from parent[->]
	 		}		 		
			child[sibling distance=15mm]{
	 										node[ghost,label=right:{h}] {} 
	 										child[grow=down] {
	 																node {...}
							 										child[grow=down] {
							 											node[leaf,label=below:{t}] {}
		 																edge from parent[dashed]
							 										}
	 																edge from parent[dashed]
	 																}
	 										edge from parent[dashed]
	 		}	
		edge from parent 
	};
	\path[opacity=1,color=red,->] [-latex]
	(raiz) edge (sb)
	(sb) edge (sbf)
	(sbf) edge (sbfl)
	(sbfl) edge (sbflt);
}

%particao pi_sbf
\uncover<10>{
  \node [leaf,label=above:{s}] (raiz) {} [] 
    	child[sibling distance=40mm] {
		node [node,label=above:{a}] {} [->]
		child {
			node [node,label=left:{i}] {} [->]
			child 	{
					node [leaf,label=below:{t1}] {} [->]	
					edge from parent 
				}
			edge from parent 
			}
		edge from parent 
        }	
	child[sibling distance=40mm] {
		node [node,label=above:{b}] (sb){} 	
   		child {
						node [node,label=right:{f}] (sbf) {} 	
						child[sibling distance=15mm]{
				 										node[ghost,label=right:{e}] {} 
				 										child[grow=down] {
				 																node {...}
										 										child[grow=down] {
										 											node[leaf,label=below:{t}] {}
					 																edge from parent[dashed]
										 										}
				 																edge from parent[dashed]
				 																}
				 										edge from parent[dashed]
				 		}		 		
						child[sibling distance=15mm] {
									node [node,label=right:{l}] (sbfl) {} 
									child{
										node [leaf,label=below:{t2}] (sbflt) {} 
										edge from parent 
									}
									edge from parent 
						}
						child[sibling distance=15mm]{
				 										node[ghost,label=right:{g}] {} 
				 										child[grow=down] {
				 																node[ghost,label=right:{l}] {}
										 										child[grow=down] {
										 											node[leaf,label=below:{t}] {}
					 																edge from parent[->]
										 										}
				 																edge from parent[->]
				 																}
				 										edge from parent[->]
				 		}		 				 		
						edge from parent 
			}	
			edge from parent 
	};
	\path[opacity=1,color=red,->] [-latex]
	(raiz) edge (sb)
	(sb) edge (sbf)
	(sbf) edge (sbfl)
	(sbfl) edge (sbflt);
}

%particao pi_sbfl
\uncover<11>{

  \node [leaf,label=above:{s}] (raiz) {} [] 
	child[sibling distance=40mm] {
			node [node,label=above:{a}] {} [->]
			child {
				node [node,label=right:{i}] {} [->]
				child 	{
						node [leaf,label=below:{t1}] {} [->]	
						edge from parent 
					}
				edge from parent 
				}
			edge from parent 
	}	
	child[sibling distance=40mm] {
		node [node,label=above:{b}] (sb){} 	
		child[sibling distance=15mm] {
					node [node,label=right:{f}] (sbf) {} 	
					child {
								node [node,label=right:{l}] (sbfl) {} 
								child{
				 										node[ghost,label=right:{g}] {} 
				 										child[grow=down] {
				 																node {...}
										 										child[grow=down] {
										 											node[leaf,label=below:{t}] {}
					 																edge from parent[dashed]
										 										}
				 																edge from parent[dashed]
				 																}
				 										edge from parent[dashed]
								}								
								child[level distance=20mm,grow=down]{
									node [leaf,label=below:{t2}] (sbflt) {} 
									edge from parent 
								}
								child{
				 										node[ghost,label=right:{h}] {} 
				 										child[grow=down] {
				 																node[ghost,label=right:{i}] {}
										 										child[grow=down] {
										 											node[leaf,label=below:{t}] {}
					 																edge from parent[->]
										 										}
				 																edge from parent[->]
				 																}
				 										edge from parent[->]
								}
								child{
				 										node[ghost,label=right:{j}] {} 
				 										child[grow=down] {
				 																node {...}
										 										child[grow=down] {
										 											node[leaf,label=below:{t}] {}
					 																edge from parent[dashed]
										 										}
				 																edge from parent[dashed]
				 																}
				 										edge from parent[dashed]
								}
								edge from parent
				}
					edge from parent 
		}	
		edge from parent 
	};
	
	\path[opacity=1,color=red,->] [-latex]
	(raiz) edge (sb)
	(sb) edge (sbf)
	(sbf) edge (sbfl)
	(sbfl) edge (sbflt);

}
\end{tikzpicture}
\end{columns}
\end{frame}


\subsection*{M�todo de \Yen}
\begin{frame}{M�todo de \Yen}
\begin{algoritmo}

\textbf{Algoritmo} \Yen{} $(V,A,c,s,t,k)$ %\\[2mm]%

1\x $P_1 \larr$ um caminho de custo m�nimo de $s$ a $t$

%1\x $\Lcal \larr \{ \mbox{um caminho de custo m�nimo de $s$ a $t$} \}$

2\x $(N,E,f) \larr $ �rvore dos prefixos de $P_1$ %\emptyset$ \quad {$\rhd$  �rvore vazia} 

3\x \para{} $i=2,\ldots,k$ \faca %\\[1mm]

4\xx  $P_i \larr \mbox{caminho de custo m�nimo em $\Lcal$}$

5\xx  $(N,E,f,\Lcal) \larr 
\xxx \Atualize~(V,A,c,s,t,P_i,N,E,f,\Lcal)$

6\x \devolva{} $\seq{P_1,\ldots,P_k}$

\end{algoritmo}
\end{frame}

\begin{frame}{\Atualize{}}
\begin{algoritmo}

\textbf{Algoritmo} \Atualize{} $(V,A,c,s,t,P,N',E',f',\Lcal')$ %\\[2mm]   
   
0\x $\Lcal \larr \Lcal' - \{P\}$

1\x $R_P \larr $ maior caminho em $(N',E')$ com ponta inicial na 

\xx raiz tal que $Q_P := f(R_P)$ � prefixo de $P$

2\x $(N,E,f) \larr$ \ArvorePrefixos $(N',E',f',P)$

3\x Seja $f(\seq{u_0,\ldots,u_k,\ldots,u_q}) = P$ e 
$R_P = \seq{u_0,\ldots,u_k}$.

4\x \para{} \cada{} $u \in \{u_k,\ldots,u_{q-1}\}$ \faca %\\[1mm]

5\xx  $P_u \larr$ st-caminho de custo m�nimo com prefixo $f(R_u)$
\xxx e que n�o possui arcos em $A_u$

6\xx  $\Lcal \larr  \Lcal \cup \{P_u\}$ %\\[1mm]

7\x \devolva{} $(N,E,f,\Lcal)$

\end{algoritmo}
\end{frame}



\begin{frame}{Exemplo}
  \begin{columns}  
    \column{.4\textwidth}
  \begin{tikzpicture}[auto,thick,sloped,node distance=9mm,scale=0.7]
    \tikzstyle{node}=%
    [%
      minimum size=8pt,%
      inner sep=0pt,%
      outer sep=0pt,%
      ball color=example text.fg,%
      circle%
    ]
        \node[node,ball color=red]  (S) [label=left:{s}] {};
        \node[node] (B) [right of=S,label=below:{b}] {};
        \node[node] (G) [right of=B,label=45:{g}] {};
        \node[node] (L) [right of=G,label=45:{l}] {};
        \node[node,ball color=red,label=above:{t}] (T) [right of=L,label=above:{t}] {};
        \node[node] (F) [above of=G,label=45:{f}] {};
        \node[node] (E) [above of=F,label=above:{e}] {};
        \node[node] (H) [below of=G,label=right:{h}] {};
        \node[node] (I) [below of=H,label=right:{i}] {};
        \node[node] (C) [above of=S,label=above:{c}] {};
        \node[node] (A) [below of=S,label=left:{a}] {};
        \node[node] (J) [right of=E,label=above:{j}] {};
        \node[node] (D) [left of=E,label=above:{d}] {};

          
          \path [-,thick,black] (S) edge (A)
                                    edge (B)
                                    edge (C)
                                (B) edge (C)
                                    edge (D)
                                    edge (F)
                                    edge (G)
                                    edge (H)
                                (D) edge (E)   
                                (A) edge (I)
                                (I) edge (H)
                                    edge (T)
                                (E) edge (F)
                                    edge (J)
                                (F) edge (G)
                                (F) edge (L)
                                (G) edge (H)
                                    edge (L)  
                                (J) edge (L)
                                (L) edge (T)
                                (H) edge (L)
                                (C) edge (D);
        \only<2>{  
	\path[-,thick,red,line width=1mm] (S) edge (B)
                            (B) edge (H)
                            (H) edge (L)
                            (L) edge (T) ;
	}
        \end{tikzpicture}
    \column{.6\textwidth}
\begin{tikzpicture}[auto,thick,sloped,scale=0.7]
  \tikzstyle{node}=%
  [%
    minimum size=8pt,%
    inner sep=0pt,%
    outer sep=0pt,%
    ball color=example text.fg,%
    circle%
  ]
  \tikzstyle{leaf}=%
  [%
    minimum size=10pt,%
    inner sep=0pt,%
    outer sep=0pt,%
    ball color=red,%
    circle%
  ]

% Set the overall layout of the tree
\tikzstyle{level 1}=[level distance=1.5cm, sibling distance=5cm]
\tikzstyle{level 2}=[level distance=1.5cm, sibling distance=2cm]

% Define styles for bags and leafs
\tikzstyle{bag} = [text width=4em, text centered]
\tikzstyle{end} = [circle, minimum width=3pt,fill, inner sep=0pt]
\only<1>{
 \node[draw,color=red] at (3.5,0) {$f(R_u) = \seq{s,b}$};
 \node[color=blue,rectangle,draw,dashed,opacity=.5,minimum width=10pt] at (2.5,-2.2)  (Au) 
          {$A_u$ \mbox{ }\mbox{ }\mbox{ }\mbox{ }\mbox{ }\mbox{ }\mbox{ }\mbox{ }\mbox{ }\mbox{ }\mbox{ }\mbox{ }\mbox{ }\mbox{ }\mbox{ }\mbox{ }\mbox{ }\mbox{ }\mbox{ }\mbox{ }\mbox{ }\mbox{ }\mbox{ }\mbox{ }\mbox{ }\mbox{ }\mbox{ }};
  \node [leaf,label=above:{s}] (raiz) {} [] 
    child {
	node [node,label=above:{a}] {} [->]
	child {
		node [node,label=right:{i}] {} 
		child 	{
				node [leaf,label=below:{t1}] {} 	
				edge from parent 
			}
		edge from parent 
		}
	edge from parent 
        }	
	child {
		node [node,label=right:{$f(u)$=b}] (sb) {} 	
   		child {
			node [node,label=left:{f},ball color=blue](sbf) {} 	
   			child {
				node [node,label=left:{l}] {} [->]
				child{
					node [leaf,label=below:{t2}] {} 
					edge from parent 
				}
				edge from parent 
			}
			edge from parent 	
	        }	
	  	child {
			node [node,label=right:{g},ball color=blue] (sbg) {} 	
   			child {
				node [node,label=left:{l}] {} [->]
				child{
					node [leaf,label=below:{t3}] {} 
					edge from parent
				}
				edge from parent 
			}
			edge from parent 
		}		
		edge from parent
 	};
	\path[opacity=1,color=red,->] [-latex]
	(raiz) edge (sb);
	\path[opacity=1,color=blue,->] [-latex]
	(sb) edge (sbf);
	\path[opacity=1,color=blue,->] [-latex]
	(sb) edge (sbg);
}
\only<2>{
  \node [leaf,label=above:{s}] (raiz) {} [] 
    child {
	node [node,label=above:{a}] {} [->] 
	child {
		node [node,label=right:{i}] {} 
		child 	{
				node [leaf,label=below:{t1}] {} 	
				edge from parent 
			}
		edge from parent 
		}
	edge from parent 
        }	
	child {
		node [node,label=above:{b}] (sb) {} 	
   		child {
			node [node,label=above:{f}] {} 	[->]
   			child {
				node [node,label=left:{l}] {} 
				child{
					node [leaf,label=below:{t2}] {} 
					edge from parent 
				}
				edge from parent 
			}
			edge from parent 
		}	
	  	child {
			node [node,label=left:{g}] {} [->]	
   			child {
				node [node,label=left:{l}] {} 
				child{
					node [leaf,label=below:{t3}] {} 
					edge from parent
				}
				edge from parent 
			}
			edge from parent 
		}		
	  	child {
			node [node,label=above:{h}] (sbh)  {} 	[]
   			child {
				node [node,label=right:{l}] (sbhl) {} 
				child{
					node [leaf,label=below:{t4}] (sbhlt) {} 
					edge from parent
				}
				edge from parent 
			}
			edge from parent 
		}	
		edge from parent 
	};
	\path[opacity=1,color=red,->] [-latex]
	(raiz) edge (sb)
	(sb) edge (sbh)
	(sbh) edge (sbhl)
	(sbhl) edge (sbhlt);
}

\end{tikzpicture}

  \end{columns}
\end{frame}



\section{\KIM}
\begin{frame}{Algoritmo de Katoh, Ibaraki e Mine}
	\begin{itemize}
		\item<1-> Espec�fico para grafos sim�tricos
		\item<2-> M�todo eficiente para o problema do desvio m�nimo
		\item<3-> Diminui��o do n�mero de caminhos candidatos gerados a cada itera��o
		\item<4-> Baseado no m�todo de Yen
		\item<5-> Parti��es definidas apenas para n�s com mais de um filho
		\item<6-> Depende rotina para o problema \PCM{}: $T(n,m)$
		\item<7-> Melhor desempenho assint�tico: $\Theta(kT(n,m))$
	\end{itemize}
\end{frame}

%% \begin{frame}{Teste}
%% \begin{tikzpicture}[style=thick]
%% \foreach \x in {0,1,2}
%% \foreach \y in {0,1,2}
%% {\draw (\y,0) -- (\x,1);}
%% \foreach \x in {0,1,2}{
%% \draw (\x,0) circle (2pt);
%% \draw (\x,1) circle (2pt);}
%% \end{tikzpicture}
%% 
%%         %\begin{figure}
%% 	%\pgfuseimage{kimCaminhos}
%% 	%\end{figure}
%% \end{frame}
%% 
%% 
%% \begin{frame}{Tabela}
%% 
%%   \begin{tikzpicture}[node distance=2cm]
%% \SetUpEdge[lw         = 1.5pt,
%% %              color      = black,
%% %              labelcolor = blue,
%% %              labeltext  = red,
%% %             labelstyle = {above,pos=.5,text=blue}
%%             ]
%%   \GraphInit[vstyle=Shade]
%%     \tikzset{LabelStyle/.style =   {                                   
%%                                   text  = red, font=\tiny}}
%%   \Vertex{a}
%%   \EA(a){b}
%%   \EA(b){d}
%%   \EA(d){e}
%%   \SOEA(b){c}
%%   \Edge[style={>=latex},label=1](a)(b)
%%   \Edge[label=1](b)(c)
%%   \Edge[label=1.0](b)(c)
%%   \Edge[label=1](c)(d)
%%   \Edge[label=1.0](d)(e)
%%   \Edge[label=0](b)(d)
%%   \end{tikzpicture}
%%   \begin{tikzpicture}[node distance=2cm]
%% \SetUpEdge[lw         = 1.5pt,
%% %              color      = black,
%% %              labelcolor = blue,
%% %              labeltext  = red,
%% %             labelstyle = {above,pos=.5,text=blue}
%%             ]
%%   \GraphInit[vstyle=Shade]
%%     \tikzset{LabelStyle/.style =   {                                   
%%                                   text  = red, font=\tiny}}
%%   \Vertex{a}
%%   \EA(a){b}
%%   \EA(b){d}
%%   \EA(d){e}
%%   \SOEA(b){c}
%%   \Edge[style={>=latex},label=1](a)(b)
%%   \Edge[label=1](b)(c)
%%   \Edge[label=1.0](b)(c)
%%   \Edge[label=1](c)(d)
%%   \Edge[label=1.0](d)(e)
%%   \Edge[label=0](b)(d)
%%   \end{tikzpicture}
%%   
%%     \begin{tikzpicture}[node distance=2cm]
%% 	\SetUpEdge[lw         = 1.5pt,
%% %              color      = black,
%% %              labelcolor = blue,
%% %              labeltext  = red,
%% %             labelstyle = {above,pos=.5,text=blue}
%%             ]
%%   \GraphInit[vstyle=Shade]
%%     \tikzset{LabelStyle/.style =   {                                   
%%                                   text  = red, font=\tiny}}
%%   \Vertex{a}
%%   \EA(a){b}
%%   \EA(b){d}
%%   \EA(d){e}
%%   \SOEA(b){c}
%%   \Edge[style={>=latex},label=1](a)(b)
%%   \Edge[label=1](b)(c)
%%   \Edge[label=1.0](b)(c)
%%   \Edge[label=1](c)(d)
%%   \Edge[label=1.0](d)(e)
%%   \Edge[label=0](b)(d)
%%   \end{tikzpicture}
%% 
%% \end{frame}
%% 
%% 
%% \begin{frame}
%% \frametitle{Prim's algorithm}
%% 
%% %% Adjacency matrix of graph
%% %% \  a  b  c  d  e  f  g
%% %% a  x  7     5
%% %% b  7  x  8  9  7
%% %% c     8  x     5
%% %% d  5  9     x 15  6
%% %% e     7  5 15  x  8  9
%% %% f           6  8  x 11
%% %% g              9  11 x
%% 
%% 
%% \tikzstyle{vertex}=[circle,fill=black!25,minimum size=20pt,inner sep=0pt]
%% \tikzstyle{selected vertex} = [vertex, fill=red!24]
%% \tikzstyle{edge} = [draw,thick,-]
%% \tikzstyle{weight} = [font=\small]
%% \tikzstyle{selected edge} = [draw,line width=5pt,-,red!50]
%% \tikzstyle{ignored edge} = [draw,line width=5pt,-,black!20]
%% 
%% 
%% \begin{figure}
%% \begin{tikzpicture}[scale=1.8, auto,swap]
%%     % Draw a 7,11 network
%%     % First we draw the vertices
%%     \foreach \pos/\name in {{(0,2)/a}, {(2,1)/b}, {(4,1)/c},
%%                             {(0,0)/d}, {(3,0)/e}, {(2,-1)/f}, {(4,-1)/g}}
%%         \node[vertex] (\name) at \pos {$\name$};
%%     % Connect vertices with edges and draw weights
%% 
%%  %   \foreach \source/ \dest /\weight in {b/a/7, c/b/8,d/a/5,d/b/9,
%%   %                                       e/b/7, e/c/5,e/d/15,
%%    %                                      f/d/6,f/e/8,
%%     %                                     g/e/9,g/f/11}
%% %        \path[edge] (\source) -- node[weight] {$\weight$} (\dest);
%%         
%%         \path[edge] (a) -- node[weight] {$7$} (b);
%%         \path[edge] (a) -- node[weight] {$5$} (d);
%%     % Start animating the vertex and edge selection. 
%% 
%% %    \foreach \vertex / \fr in {d/1,a/2,f/3,b/4,e/5,c/6,g/7}
%% %        \path<\fr-> node[selected vertex] at (\vertex) {$\vertex$};
%% 
%%     % For convenience we use a background layer to highlight edges
%%     % This way we don't have to worry about the highlighting covering
%%     % weight labels. 
%% 
%%  %   \begin{pgfonlayer}{background}
%%  %       \pause
%%  %       \foreach \source / \dest in {d/a,d/f,a/b,b/e,e/c,e/g}
%%  %           \path<+->[selected edge] (\source.center) -- (\dest.center);
%%  %       \foreach \source / \dest / \fr in {d/b/4,d/e/5,e/f/5,b/c/6,f/g/7}
%%  %           \path<\fr->[ignored edge] (\source.center) -- (\dest.center);
%%  %  \end{pgfonlayer}
%%     
%% \end{tikzpicture}
%% \end{figure}
%% 
%% \end{frame}

\begin{frame}{Exemplo �rvore dos menores caminhos}
\begin{columns}
\column{.6\textwidth}
\tikzstyle{vertex}=[circle,fill=black!25,minimum size=15pt,inner sep=0pt]
\tikzstyle{start} = [vertex, fill=red!24]
\tikzstyle{end} = [vertex, fill=red!24]
\tikzstyle{edge} = [draw,thick,-]
\tikzstyle{weight} = [font=\tiny]
\tikzstyle{selected edge} = [draw,line width=2pt,-,red!50]
\tikzstyle{not used} = [draw,line width=2pt,-,blue!50]
	\begin{tikzpicture}[scale=1.0, auto,swap,node distance=2cm]
        \node[start] (a) {a};
        \node[vertex] (b) [right of=a] {b};
        \node[vertex] (c) [above right of=b] {c};
        \node[vertex] (d) [right of=b] {d};
        \node[vertex] (f) [below right of=b] {f};
        \node[end] (e) [right of=d] {e};
          
         \path[edge] (a) -- node[weight] {$1$} (b);
         \path[edge] (b) -- node[weight] {$1$} (d);
         \path[edge] (d) -- node[weight] {$1$} (e);
         \path[edge] (a) -- node[weight,above] {$10$} (c);
         \path[edge] (b) -- node[weight] {$10$} (c);
         \path[edge] (c) -- node[weight] {$1$} (d);
         \path[edge] (b) -- node[weight] {$1$} (f);
         \path[edge] (f) -- node[weight] {$1$} (d);
         
   \end{tikzpicture}

 
%%%%%%%%%%%%%%%%%%%%%%%%%%%%%%%%%%%%%%%%%%%%%%%%%%%%%%%%%%%%%%%%%%%%%%%%%%%%%
\column{.4\textwidth}
\tikzstyle{node}=[circle,fill=black!25,minimum size=10pt,inner sep=0pt]
\tikzstyle{root} = [node, fill=red!24]
\tikzstyle{leaf} = [node, fill=gray!24]
\tikzstyle{onPath} = [node, fill=red!24]
\tikzstyle{edge} = [draw,thick,-]
\tikzstyle{weight} = [font=\tiny]
\tikzstyle{selected edge} = [draw,thick,-,red!50]
%\tikzstyle{LabelStyle}=[fill=white,sloped]
\onslide<2->{
\begin{tikzpicture}[scale=0.9]
% Set the overall layout of the tree
\tikzstyle{level 1}=[level distance=1.5cm, sibling distance=2.5cm]
\tikzstyle{level 2}=[level distance=1.5cm, sibling distance=2.5cm]
  	\node[root] (a) {a} [grow=down,-] 
  	    child {
				node[node] (ab) {b}
				child[grow=south west]{
						node[node] (abd){d} 
						child[grow=south west]{
							node[node] (abde) {e}
							edge from parent
							node[left,weight] {1}
						}
						child[grow=south east] {
							node[node] {c}
							edge from parent
							node[right,weight] {1}
	 					}
						edge from parent
						node[left,weight] {1}
				}
				child[grow=south east]{
					node[node] {f} 
					edge from parent
					node[left,weight] {1}
				}				
				edge from parent
				node[left,weight] {1}
	 };
	\end{tikzpicture}
\tikzstyle{node}=[circle,fill=black!25,minimum size=10pt,inner sep=0pt]
\tikzstyle{root} = [node, fill=red!24]
\tikzstyle{leaf} = [node, fill=red!24]
\tikzstyle{onPath} = [node, fill=red!24]
\tikzstyle{edge} = [draw,thick,-]
\tikzstyle{weight} = [font=\tiny]
\tikzstyle{selected edge} = [draw,thick,-,red!50]
}
\onslide<3->{
%\tikzstyle{LabelStyle}=[fill=white,sloped]
\begin{tikzpicture}[scale=0.9]
% Set the overall layout of the tree
\tikzstyle{level 1}=[level distance=1.5cm, sibling distance=2.5cm]
\tikzstyle{level 2}=[level distance=1.5cm, sibling distance=2.5cm]
  	\node[root] (e) {e} [grow=up,-] 
  	    child {
				node[node]  (ed) {d} 
				child[grow=north west]{
						node[node] (edb) {b}  
						child{
							node[node] (edba) {a} 
							edge from parent
							node[left,weight] {1}
						}
						edge from parent
						node[left,weight] {1}
				}
				child {
							node[node] {c}
							edge from parent
							node[left,weight] {1}
	 					}
				child[grow=north east] {
							node[node] {f}
							edge from parent
							node[left,weight] {1}
	 					}
				edge from parent
				node[left,weight] {1}
	 }
	     ;
	\end{tikzpicture}
}

\end{columns}
\end{frame}


\subsection*{Problema do desvio de custo m�nimo}
\begin{frame}{Defini��o}
  \begin{quote}
    \textbf{Problema do desvio de custo m�nimo} $(V,A,c,s,t)$:\\
   Dados: 	
 	\begin{itemize}   
 	\item \small Grafo sim�trico com custo: $(V,A,c)$
 	\item V�rtices: $s$ e $t$ 
 	\item $P=\seq{s=a_1,\ldots,a_{n}=t}$ um caminho de custo m�nimo de $s$ a $t$
% 	\item $P_r=\seq{t = a_n , \ldots , a_{1} = s }$
% 	\item $T_s$ uma �rvore de menores caminhos com raiz em $s$
% 	\item $T_t$ uma �rvore de menores caminhos com raiz em $t$
% 	\item $P \in T_s$ e $P_r \in T_t$
	\end{itemize}	
	Encontrar um caminho de custo m�nimo, diferente de $P$, entre $s$ e $t$.
	\end{quote}
\end{frame}


\begin{frame}{Exemplo}

\tikzstyle{vertex}=[circle,fill=black!25,minimum size=15pt,inner sep=0pt]
\tikzstyle{start} = [vertex, fill=red!24]
\tikzstyle{end} = [vertex, fill=red!24]
\tikzstyle{edge} = [draw,thick,-]
\tikzstyle{weight} = [font=\tiny]
\tikzstyle{selected edge} = [draw,line width=2pt,-,red!50]
\tikzstyle{not used} = [draw,line width=2pt,-,blue!50]
\only<1>{
	\begin{tikzpicture}[scale=1.0, auto,swap,node distance=7cm]
        \node[start] (s) {s};
        \node[end] (t) [right of=s] {t};
        
         \path[selected edge] (s) -- node[weight] {} (t);
        
   \end{tikzpicture}
}
\onslide<2->{
	\begin{tikzpicture}[scale=1.0, auto,swap,node distance=2cm]
        \node[start] (s) {s};
        \node[vertex] (desvio) [right of=s] {$\delta$};
        \node[vertex] (u) [below right of=desvio] {u};
        \node (aux) [right of=u] {...};       
        \node[end,node distance=5cm] (t) [right of=desvio] {t};
        \node[end] (t2) [right of=aux] {t};
          
         \path[selected edge,dashed] (s) -- node[weight] {} (desvio);
         \path[selected edge] (desvio) -- node[weight] {} (t);
         \path[not used] (desvio) -- node[weight] {} (u);
         \path[edge,dashed] (u) -- node[weight,dashed] {} (aux);
         \path[edge,dashed] (aux) -- node[weight,dashed] {} (t2);
         
   \end{tikzpicture}
}
\onslide<3->{
	\begin{itemize}
	\item<3-> arco $\delta u \in T_s$
		\begin{itemize}
		\item<4-> $s \underset{T_s}{\longrightarrow} u \underset{T_t}{\longrightarrow} t$
		\end{itemize}
	\item<3-> arco $\delta u \notin T_s$
		\begin{itemize}
			\item<5-> $s \underset{T_s}{\longrightarrow} u \underset{\in A}{\rightarrow} v \underset{T_t}{\longrightarrow} t$
		\end{itemize}

	\end{itemize}
}
\end{frame}

\begin{frame}{Tipos de caminhos}
Um novo caminho de custo m�nimo constru�do atrav�s de $u \in V$ � de um dos dois tipos definidos a seguir:
\begin{description}
\item[tipo I]: $s \underset{T_s}{\longrightarrow} u \underset{T_t}{\longrightarrow} t$.  %Se $\epsilon(u)<\zeta(u)$
\newline
\item[tipo II]: $s \underset{T_s}{\longrightarrow} u \underset{\in A}{\rightarrow} v \underset{T_t}{\longrightarrow} t$. %Se $\epsilon(u)=\zeta(u)$ e $\epsilon(u)<\zeta(v)$
\end{description}

\begin{itemize}
\item $T_s$ uma �rvore de menores caminhos com raiz em $s$
\item $T_t$ uma �rvore de menores caminhos com raiz em $t$
\item $P_r=\seq{t = a_n , \ldots , a_{1} = s }$
\item $P \in T_s$ e $P_r \in T_t$
\end{itemize}
\end{frame}

\begin{frame}{Desvio de custo m�nimo}
%%%%%%%%%%%%%%%%%%%%%%%%%%%%%%%%%%%%%%%%%%%%%%%%%%%%%%%%%%%%%%%%%%%%%%%%%%%%%%%%%%%
%%%%%%%%%%%%%%%%%%%%%%%%%%%%%%%%%%grafo%%%%%%%%%%%%%%%%%%%%%%%%%%%%%%%%%%%%%%%%%%%5
%%%%%%%%%%%%%%%%%%%%%%%%%%%%%%%%%%%%%%%%%%%%%%%%%%%%%%%%%%%%%%%%%%%%%%%%%%%%%%%%%%%
\begin{columns}
\column{.6\textwidth}
\tikzstyle{vertex}=[circle,fill=black!25,minimum size=15pt,inner sep=0pt]
\tikzstyle{start} = [vertex, fill=red!24]
\tikzstyle{end} = [vertex, fill=red!24]
\tikzstyle{edge} = [draw,thick,-]
\tikzstyle{weight} = [font=\tiny]
\tikzstyle{selected edge} = [draw,line width=2pt,-,red!50]
\tikzstyle{not used} = [draw,line width=2pt,-,blue!50]
\only<1>{
	\begin{tikzpicture}[remember picture=false,scale=1.0, auto,swap,node distance=2cm]
        \node[start] (s) {s};
        \node[vertex] (b) [right of=s] {b};
        \node[vertex] (c) [above right of=b] {c};
        \node[vertex] (d) [right of=b] {d};
        \node[vertex] (f) [below right of=b] {f};
        \node[end] (t) [right of=d] {t};
          
         \path[selected edge] (s) -- node[weight] {$1$} (b);
         \path[selected edge] (b) -- node[weight] {$1$} (d);
         \path[selected edge] (d) -- node[weight] {$1$} (t);
         \path[not used] (s) -- node[weight,above] {$10$} (c);
         \path[not used] (b) -- node[weight] {$10$} (c);
         \path[edge] (c) -- node[weight] {$1$} (d);
         \path[edge] (b) -- node[weight] {$1$} (f);
         \path[edge] (f) -- node[weight] {$1$} (d);
         
   \end{tikzpicture}
}
\only<2>{
	\tikzstyle{novo} = [draw,line width=2pt,-,green!50]
	\begin{tikzpicture}[remember picture=false,scale=1.0, auto,swap,node distance=2cm]
        \node[start] (s) {s};
        \node[vertex] (b) [right of=s] {b};
        \node[vertex] (c) [above right of=b] {c};
        \node[vertex] (d) [right of=b] {d};
        \node[vertex] (f) [below right of=b] {f};
        \node[end] (t) [right of=d] {t};
          
         \path[novo] (s) -- node[weight] {$1$} (b);
         \path[novo] (b) -- node[weight] {$1$} (f);
         \path[novo] (f) -- node[weight] {$1$} (d);
         \path[novo] (d) -- node[weight] {$1$} (t);
         \path[selected edge] (b) -- node[weight] {$1$} (d);
         \path[not used] (s) -- node[weight,above] {$10$} (c);
         \path[not used] (b) -- node[weight] {$10$} (c);
         \path[edge] (c) -- node[weight] {$1$} (d);
   \end{tikzpicture}
}
\only<3>{
	\tikzstyle{novo} = [draw,line width=2pt,-,green!50]
	\begin{tikzpicture}[remember picture=false,scale=1.0, auto,swap,node distance=2cm]
        \node[start] (s) {s};
        \node[vertex] (b) [right of=s] {b};
        \node[vertex] (c) [above right of=b] {c};
        \node[vertex] (d) [right of=b] {d};
        \node[vertex] (f) [below right of=b] {f};
        \node[end] (t) [right of=d] {t};
          
         \path[selected edge] (s) -- node[weight] {$1$} (b);
         \path[selected edge] (b) -- node[weight] {$1$} (d);
         \path[novo] (d) -- node[weight] {$1$} (t);
         \path[novo] (s) -- node[weight,above] {$10$} (c);
         \path[not used] (b) -- node[weight] {$10$} (c);
         \path[novo] (c) -- node[weight] {$1$} (d);
         \path[edge] (b) -- node[weight] {$1$} (f);
         \path[edge] (f) -- node[weight] {$1$} (d);
   \end{tikzpicture}
}
\only<4>{
	\tikzstyle{novo} = [draw,line width=2pt,-,green!50]
	\begin{tikzpicture}[remember picture=false,scale=1.0, auto,swap,node distance=2cm]
        \node[start] (s) {s};
        \node[vertex] (b) [right of=s] {b};
        \node[vertex] (c) [above right of=b] {c};
        \node[vertex] (d) [right of=b] {d};
        \node[vertex] (f) [below right of=b] {f};
        \node[end] (t) [right of=d] {t};
          
         \path[novo] (s) -- node[weight] {$1$} (b);
         \path[selected edge] (b) -- node[weight] {$1$} (d);
         \path[novo] (d) -- node[weight] {$1$} (t);
         \path[not used] (s) -- node[weight,above] {$10$} (c);
         \path[novo] (b) -- node[weight] {$10$} (c);
         \path[novo] (c) -- node[weight] {$1$} (d);
         \path[edge] (b) -- node[weight] {$1$} (f);
         \path[edge] (f) -- node[weight] {$1$} (d);
   \end{tikzpicture}
}

   \begin{columns}[t]
   \column{.5\textwidth}
	
	\begin{block}{Tipo I}
	\begin{itemize}   
		\only<2>{\item \small \textcolor{green}{$\seq{s,b,f,d,t}$}}
		\onslide<3->{ \item \small $\seq{s,b,f,d,t}$ }
   \end{itemize}
	\end{block}   

   \column{.5\textwidth}

  	\begin{block}{Tipo II}
	\begin{itemize}   
		\only<3>{\item \small \textcolor{green}{$\seq{s,c,d,t}$}}
		\onslide<4->{\item \small $\seq{s,c,d,t}$}
		\onslide<4->{ \item \small \textcolor{green}{$\seq{s,b,c,d,t}$} }
   \end{itemize}
	\end{block} 
   \end{columns}
%	\begin{itemize}
%  	\only<3>{\item \small \textcolor{green}{$\seq{a,c,d,e}$}}
%  	\only<4>{\item \small \textcolor{green}{$\seq{a,b,c,d,e}$}}
%   \end{itemize}
   
%%%%%%%%%%%%%%%%%%%%%%%%%%%%%%%%%%%%%%%%%%%%%%%%%%%%%%%%%%%%%%%%%%%%%%%%%%%%%
\column{.4\textwidth}
\tikzstyle{node}=[circle,fill=black!25,minimum size=10pt,inner sep=0pt]
\tikzstyle{root} = [node, fill=red!24]
\tikzstyle{leaf} = [node, fill=red!24]
\tikzstyle{onPath} = [node, fill=red!24]
\tikzstyle{edge} = [draw,thick,-]
\tikzstyle{weight} = [font=\tiny]
\tikzstyle{selected edge} = [draw,thick,-,red!50]
%\tikzstyle{LabelStyle}=[fill=white,sloped]
\begin{tikzpicture}[remember picture=true,scale=0.9]
% Set the overall layout of the tree
\tikzstyle{level 1}=[level distance=1.5cm, sibling distance=2.5cm]
\tikzstyle{level 2}=[level distance=1.5cm, sibling distance=2.5cm]
  	\node[root] (a) {s} [grow=down,-] 
  	    child {
				node[onPath] (ab) {b}
				child[grow=south west]{
						node[onPath](abd){d} 
						child[grow=south west]{
							node[onPath] (abde) {t}
							edge from parent
							node[left,weight] {1}
						}
						child[grow=south east] {
							node[node] {c}
							edge from parent
							node[right,weight] {1}
	 					}
						edge from parent
						node[left,weight] {1}
				}
				child[grow=south east]{
					node[node] (abf) {f} 
					edge from parent
					node[left,weight] {1}
				}				
				edge from parent
				node[left,weight] {1}
	 };
  	\path[selected edge] (a) -- node[weight] {} (ab);
   \path[selected edge] (ab) -- node[weight] {} (abd);
   \path[selected edge] (abd) -- node[weight] {} (abde);   
	\end{tikzpicture}
\tikzstyle{node}=[circle,fill=black!25,minimum size=10pt,inner sep=0pt]
\tikzstyle{root} = [node, fill=red!24]
\tikzstyle{leaf} = [node, fill=red!24]
\tikzstyle{onPath} = [node, fill=red!24]
\tikzstyle{edge} = [draw,thick,-]
\tikzstyle{weight} = [font=\tiny]
\tikzstyle{selected edge} = [draw,thick,-,red!50]
%\tikzstyle{LabelStyle}=[fill=white,sloped]
\begin{tikzpicture}[remember picture=true,scale=0.9]
% Set the overall layout of the tree
\tikzstyle{level 1}=[level distance=1.5cm, sibling distance=2.5cm]
\tikzstyle{level 2}=[level distance=1.5cm, sibling distance=2.5cm]
  	\node[root] (e) {t} [grow=up,-] 
  	    child {
				node[onPath]  (ed) {d} 
				child[grow=north west]{
						node[onPath] (edb) {b}  
						child{
							node[onPath] (edba) {s} 
							edge from parent
							node[left,weight] {1}
						}
						edge from parent
						node[left,weight] {1}
				}
				child {
							node[node] (tdc) {c}
							edge from parent
							node[left,weight] {1}
	 					}
				child[grow=north east] {
							node[node] (tdf) {f}
							edge from parent
							node[left,weight] {1}
	 					}
				edge from parent
				node[left,weight] {1}
	 }
	     ;
 	\path[selected edge] (e) -- node[weight] {} (ed);
   \path[selected edge] (ed) -- node[weight] {} (edb);
   \path[selected edge] (edb) -- node[weight] {} (edba);
	\end{tikzpicture}
	\only<2>{
	\begin{tikzpicture}[remember picture=true,overlay,>=latex]
	 	\path[dashed,green] (abf) edge (tdf);
	\end{tikzpicture}
	}
	\only<3>{
	\begin{tikzpicture}[remember picture=true,overlay,>=latex]
	 	\path[dashed,green,bend left] (a) edge (tdc);
	\end{tikzpicture}
	}
		\only<4>{
	\begin{tikzpicture}[remember picture=true,overlay,>=latex]
	 	\path[dashed,green,bend left] (ab) edge (tdc);
	\end{tikzpicture}
	}

\end{columns}
\end{frame}


\begin{frame}{Rotula��es $\epsilon$ e $\zeta$}
\begin{center}
\textcolor{red}{$P=\seq{a,b,d,e}$}
\end{center}
\begin{columns}[c]
\column{.5\textwidth}
\begin{block}{Rotula��o $\epsilon$}
\small
$\epsilon(a)=1$ \\
Se $u \in P$ ent�o $\epsilon(u)=\epsilon(\pred(u))+1$ \\
Se $u \notin P$ ent�o $\epsilon(u)=\epsilon(\pred(u))$
\end{block}
\tikzstyle{node}=[circle,fill=black!25,minimum size=10pt,inner sep=0pt]
\tikzstyle{root} = [node, fill=red!24]
\tikzstyle{leaf} = [node, fill=red!24]
\tikzstyle{onPath} = [node, fill=red!24]
\tikzstyle{edge} = [draw,thick,-]
\tikzstyle{weight} = [font=\tiny]
\tikzstyle{selected edge} = [draw,thick,-,red!50]
%\tikzstyle{LabelStyle}=[fill=white,sloped]
\begin{tikzpicture}[scale=1.0]
% Set the overall layout of the tree
\tikzstyle{level 1}=[level distance=1.5cm, sibling distance=2.5cm]
\tikzstyle{level 2}=[level distance=1.5cm, sibling distance=2.5cm]
  	\node[root,label=right:{\footnotesize $\epsilon=1$}] (a) {a} [grow=down,-] 
  	    child {
				node[onPath,label=right:{\footnotesize $2$}] (ab) {b}
				child[grow=south west]{
						node[onPath,label=right:{\footnotesize $3$}](abd){d} 
						child[grow=south west]{
							node[onPath,label=left:{\footnotesize $4$}] (abde) {e}
							edge from parent
							node[left,weight] {}
						}
						child[grow=south east] {
							node[node,label=right:{\footnotesize $3$}] {c}
							edge from parent
							node[right,weight] {}
	 					}
						edge from parent
						node[left,weight] {}
				}
				child[grow=south east]{
					node[node,label=right:{\footnotesize $2$}] {f} 
					edge from parent
					node[left,weight] {}
				}				
				edge from parent
				node[left,weight] {}
	 };
  	\path[selected edge] (a) -- node[weight] {} (ab);
   \path[selected edge] (ab) -- node[weight] {} (abd);
   \path[selected edge] (abd) -- node[weight] {} (abde);   
	\end{tikzpicture}
\column{.5\textwidth}
\begin{block}{Rotula��o $\zeta$}
\small
$\zeta(e)=|P|$ \\
Se $u \in P$ ent�o $\zeta(u)=\zeta(\pred(u))-1$ \\
Se $u \notin P$ ent�o $\zeta(u)=\zeta(\pred(u))$
\end{block}
\tikzstyle{node}=[circle,fill=black!25,minimum size=10pt,inner sep=0pt]
\tikzstyle{root} = [node, fill=red!24]
\tikzstyle{leaf} = [node, fill=red!24]
\tikzstyle{onPath} = [node, fill=red!24]
\tikzstyle{edge} = [draw,thick,-]
\tikzstyle{weight} = [font=\tiny]
\tikzstyle{selected edge} = [draw,thick,-,red!50]
%\tikzstyle{LabelStyle}=[fill=white,sloped]
\begin{tikzpicture}[scale=1.0]
% Set the overall layout of the tree
\tikzstyle{level 1}=[level distance=1.5cm, sibling distance=2.5cm]
\tikzstyle{level 2}=[level distance=1.5cm, sibling distance=2.5cm]
  	\node[root,label=right:{\footnotesize $\zeta=4$}] (e) {e} [grow=up,-] 
  	    child {
				node[onPath,label=right:{\footnotesize $3$}]  (ed) {d} 
				child[grow=north west]{
						node[onPath,label=left:{\footnotesize $2$}] (edb) {b}  
						child{
							node[onPath,label=left:{\footnotesize $1$}] (edba) {a} 
							edge from parent
							node[left,weight] {}
						}
						edge from parent
						node[left,weight] {}
				}
				child {
							node[node,label=right:{\footnotesize $3$}] {c}
							edge from parent
							node[left,weight] {}
	 					}
				child[grow=north east] {
							node[node,label=right:{\footnotesize $3$}] {f}
							edge from parent
							node[left,weight] {}
	 					}
				edge from parent
				node[left,weight] {}
	 }
	     ;
 	\path[selected edge] (e) -- node[weight] {} (ed);
   \path[selected edge] (ed) -- node[weight] {} (edb);
   \path[selected edge] (edb) -- node[weight] {} (edba);
	\end{tikzpicture}
\end{columns}
\end{frame}


\begin{frame}{Desvio de custo m�nimo com rotula��es}
%%%%%%%%%%%%%%%%%%%%%%%%%%%%%%%%%%%%%%%%%%%%%%%%%%%%%%%%%%%%%%%%%%%%%%%%%%%%%%%%%%%
%%%%%%%%%%%%%%%%%%%%%%%%%%%%%%%%%%grafo%%%%%%%%%%%%%%%%%%%%%%%%%%%%%%%%%%%%%%%%%%%5
%%%%%%%%%%%%%%%%%%%%%%%%%%%%%%%%%%%%%%%%%%%%%%%%%%%%%%%%%%%%%%%%%%%%%%%%%%%%%%%%%%%
\begin{columns}
\column{.6\textwidth}
\tikzstyle{vertex}=[circle,fill=black!25,minimum size=15pt,inner sep=0pt]
\tikzstyle{start} = [vertex, fill=red!24]
\tikzstyle{end} = [vertex, fill=red!24]
\tikzstyle{edge} = [draw,thick,-]
\tikzstyle{weight} = [font=\tiny]
\tikzstyle{selected edge} = [draw,line width=2pt,-,red!50]
\tikzstyle{not used} = [draw,line width=2pt,-,blue!50]
\only<1>{
	\begin{tikzpicture}[scale=1.0, auto,swap,node distance=2cm]
        \node[start] (a) {s};
        \node[vertex] (b) [right of=a] {b};
        \node[vertex] (c) [above right of=b] {c};
        \node[vertex] (d) [right of=b] {d};
        \node[vertex] (f) [below right of=b] {f};
        \node[end] (e) [right of=d] {t};
          
         \path[selected edge] (a) -- node[weight] {$1$} (b);
         \path[selected edge] (b) -- node[weight] {$1$} (d);
         \path[selected edge] (d) -- node[weight] {$1$} (e);
         \path[not used] (a) -- node[weight,above] {$10$} (c);
         \path[not used] (b) -- node[weight] {$10$} (c);
         \path[edge] (c) -- node[weight] {$1$} (d);
         \path[edge] (b) -- node[weight] {$1$} (f);
         \path[edge] (f) -- node[weight] {$1$} (d);
         
   \end{tikzpicture}
}
\only<2>{
	\tikzstyle{novo} = [draw,line width=2pt,-,green!50]
	\begin{tikzpicture}[scale=1.0, auto,swap,node distance=2cm]
        \node[start] (a) {s};
        \node[vertex] (b) [right of=a] {b};
        \node[vertex] (c) [above right of=b] {c};
        \node[vertex] (d) [right of=b] {d};
        \node[vertex] (f) [below right of=b] {f};
        \node[end] (e) [right of=d] {t};
          
         \path[novo] (a) -- node[weight] {$1$} (b);
         \path[novo] (b) -- node[weight] {$1$} (f);
         \path[novo] (f) -- node[weight] {$1$} (d);
         \path[novo] (d) -- node[weight] {$1$} (e);
         \path[selected edge] (b) -- node[weight] {$1$} (d);
         \path[not used] (a) -- node[weight,above] {$10$} (c);
         \path[not used] (b) -- node[weight] {$10$} (c);
         \path[edge] (c) -- node[weight] {$1$} (d);
   \end{tikzpicture}
}
\only<3>{
	\tikzstyle{novo} = [draw,line width=2pt,-,green!50]
	\begin{tikzpicture}[scale=1.0, auto,swap,node distance=2cm]
        \node[start] (a) {s};
        \node[vertex] (b) [right of=a] {b};
        \node[vertex] (c) [above right of=b] {c};
        \node[vertex] (d) [right of=b] {d};
        \node[vertex] (f) [below right of=b] {f};
        \node[end] (e) [right of=d] {t};
          
         \path[selected edge] (a) -- node[weight] {$1$} (b);
         \path[selected edge] (b) -- node[weight] {$1$} (d);
         \path[novo] (d) -- node[weight] {$1$} (e);
         \path[novo] (a) -- node[weight,above] {$10$} (c);
         \path[not used] (b) -- node[weight] {$10$} (c);
         \path[novo] (c) -- node[weight] {$1$} (d);
         \path[edge] (b) -- node[weight] {$1$} (f);
         \path[edge] (f) -- node[weight] {$1$} (d);
   \end{tikzpicture}
}
\only<4>{
	\tikzstyle{novo} = [draw,line width=2pt,-,green!50]
	\begin{tikzpicture}[scale=1.0, auto,swap,node distance=2cm]
        \node[start] (a) {s};
        \node[vertex] (b) [right of=a] {b};
        \node[vertex] (c) [above right of=b] {c};
        \node[vertex] (d) [right of=b] {d};
        \node[vertex] (f) [below right of=b] {f};
        \node[end] (e) [right of=d] {t};
          
         \path[novo] (a) -- node[weight] {$1$} (b);
         \path[selected edge] (b) -- node[weight] {$1$} (d);
         \path[novo] (d) -- node[weight] {$1$} (e);
         \path[not used] (a) -- node[weight,above] {$10$} (c);
         \path[novo] (b) -- node[weight] {$10$} (c);
         \path[novo] (c) -- node[weight] {$1$} (d);
         \path[edge] (b) -- node[weight] {$1$} (f);
         \path[edge] (f) -- node[weight] {$1$} (d);
   \end{tikzpicture}
}

   \begin{columns}[t]
   \column{.5\textwidth}
	
	\begin{block}{Tipo I}
	\begin{itemize}   
		\only<2>{\item \small \textcolor{green}{$\seq{s,b,f,d,t}$}}
		\onslide<3->{ \item \small $\seq{s,b,f,d,t}$ }
   \end{itemize}
	\end{block}   

   \column{.5\textwidth}

  	\begin{block}{Tipo II}
	\begin{itemize}   
		\only<3>{\item \small \textcolor{green}{$\seq{s,c,d,t}$}}
		\onslide<4->{ \item \small \textcolor{green}{$\seq{s,b,c,d,t}$} }
   \end{itemize}
	\end{block} 
   \end{columns}
%	\begin{itemize}
%  	\only<3>{\item \small \textcolor{green}{$\seq{a,c,d,e}$}}
%  	\only<4>{\item \small \textcolor{green}{$\seq{a,b,c,d,e}$}}
%   \end{itemize}
   
%%%%%%%%%%%%%%%%%%%%%%%%%%%%%%%%%%%%%%%%%%%%%%%%%%%%%%%%%%%%%%%%%%%%%%%%%%%%%
\column{.4\textwidth}
\tikzstyle{node}=[circle,fill=black!25,minimum size=10pt,inner sep=0pt]
\tikzstyle{root} = [node, fill=red!24]
\tikzstyle{leaf} = [node, fill=red!24]
\tikzstyle{onPath} = [node, fill=red!24]
\tikzstyle{edge} = [draw,thick,-]
\tikzstyle{weight} = [font=\tiny]
\tikzstyle{selected edge} = [draw,thick,-,red!50]
%\tikzstyle{LabelStyle}=[fill=white,sloped]
\begin{tikzpicture}[remember picture=true,scale=0.9]
% Set the overall layout of the tree
\tikzstyle{level 1}=[level distance=1.5cm, sibling distance=2.5cm]
\tikzstyle{level 2}=[level distance=1.5cm, sibling distance=2.5cm]
  	\node[root,label=right:{\footnotesize $\epsilon=1$}] (a) {s} [grow=down,-] 
  	    child {
				node[onPath,label=right:{\footnotesize $2$}] (ab) {b}
				child[grow=south west]{
						node[onPath,label=right:{\footnotesize $3$}](abd){d} 
						child[grow=south west]{
							node[onPath,label=left:{\footnotesize $4$}] (abde) {t}
							edge from parent
							node[left,weight] {1}
						}
						child[grow=south east] {
							node[node,label=right:\textcolor{red}{\footnotesize $3$}] {c}
							edge from parent
							node[right,weight] {1}
	 					}
						edge from parent
						node[left,weight] {1}
				}
				child[grow=south east]{
					node[node,label=right:\textcolor{red}{\footnotesize $2$}] (abf) {f} 
					edge from parent
					node[left,weight] {1}
				}				
				edge from parent
				node[left,weight] {1}
	 };
  	\path[selected edge] (a) -- node[weight] {} (ab);
   \path[selected edge] (ab) -- node[weight] {} (abd);
   \path[selected edge] (abd) -- node[weight] {} (abde);   
	\end{tikzpicture}
\tikzstyle{node}=[circle,fill=black!25,minimum size=10pt,inner sep=0pt]
\tikzstyle{root} = [node, fill=red!24]
\tikzstyle{leaf} = [node, fill=red!24]
\tikzstyle{onPath} = [node, fill=red!24]
\tikzstyle{edge} = [draw,thick,-]
\tikzstyle{weight} = [font=\tiny]
\tikzstyle{selected edge} = [draw,thick,-,red!50]
%\tikzstyle{LabelStyle}=[fill=white,sloped]
\begin{tikzpicture}[remember picture=true,scale=0.9]
% Set the overall layout of the tree
\tikzstyle{level 1}=[level distance=1.5cm, sibling distance=2.5cm]
\tikzstyle{level 2}=[level distance=1.5cm, sibling distance=2.5cm]
  	\node[root,label=right:{\footnotesize $\zeta=4$}] (e) {t} [grow=up,-] 
  	    child {
				node[onPath,label=right:{\footnotesize $3$}]  (ed) {d} 
				child[grow=north west]{
						node[onPath,label=left:{\footnotesize $2$}] (edb) {b}  
						child{
							node[onPath,label=left:{\footnotesize $1$}] (edba) {s} 
							edge from parent
							node[left,weight] {1}
						}
						edge from parent
						node[left,weight] {1}
				}
				child {
							node[node,label=right:\textcolor{red}{\footnotesize $3$}] (tdc) {c}
							edge from parent
							node[left,weight] {1}
	 					}
				child[grow=north east] {
							node[node,label=right:\textcolor{red}{\footnotesize $3$}] (tdf) {f}
							edge from parent
							node[left,weight] {1}
	 					}
				edge from parent
				node[left,weight] {1}
	 }
	     ;
 	\path[selected edge] (e) -- node[weight] {} (ed);
   \path[selected edge] (ed) -- node[weight] {} (edb);
   \path[selected edge] (edb) -- node[weight] {} (edba);
	\end{tikzpicture}
		\only<2>{
	\begin{tikzpicture}[remember picture=true,overlay,>=latex]
	 	\path[dashed,green] (abf) edge (tdf);
	\end{tikzpicture}
	}
	\only<3>{
	\begin{tikzpicture}[remember picture=true,overlay,>=latex]
	 	\path[dashed,green,bend left] (a) edge (tdc);
	\end{tikzpicture}
	}
		\only<4>{
	\begin{tikzpicture}[remember picture=true,overlay,>=latex]
	 	\path[dashed,green,bend left] (ab) edge (tdc);
	\end{tikzpicture}
	}

\end{columns}
\end{frame}

\begin{frame}{Disposi��o esquem�tica das parti��es}
	  \begin{tikzpicture}[scale=0.70]
    \tikzstyle{node}=%
    [%
      inner sep=0pt,%
      outer sep=0pt,%
      ball color=example text.fg,%
      circle%
    ]
        \node[node,minimum size=8pt,label=left:{s}] at (0,3) (S)  {};
        \node[node] at (1,3) (A)  {};
        \node[node] at (2,3) (B)  {};
        \node[node] at (3,3) (C)  {};
        \node[node] at (4,1) (D)  {};
        \node[node,minimum size=8pt,label=right:{t}] at (11,3) (T) {};
       
        \node[color=red] at (6,3.2) (P1) {P1};

        \path [-,thick,red]  
                  (S) edge[] (T)
                  (S) edge (A)
                  (A) edge (B)
                  (B) edge (C);
       \onslide<1->{
        \path [-,thick,blue]  
                  (B) edge[bend right] (D)
                  (D) edge[bend right] (T);
         \node[color=blue] at (6,1) (P2) {P2};

        }
        \onslide<1->{
        \path [-,thick,black,dashed]  
                  (A.north) edge[in=90] (T.north);
                 \node[color=black] at (6,5.8) (Pc) {Pc};
}
        \onslide<1->{
        \path [-,thick,black,dashed]  
                  (C.north) edge[bend left] (T.north);
                 \node[color=black] at (6,4.5) (Pb) {Pb};
}
       \onslide<1->{
        \path [-,thick,black,dashed]  
                  (D) edge[in=-90,out=-90] (T);
                 \node[color=black] at (6,0) (Pa) {Pa};
}
        \end{tikzpicture}

\end{frame}

%% \begin{frame}{MINDMAP}
%% \tikz[mindmap,concept color=blue!80]
%% \node [concept] {\Yen}
%% child[concept color=red,grow=30] {node[concept] {\KIM}}
%% child[concept color=orange,grow=0] {node[concept] {\HMS}};
%% \end{frame}




\section{Implementa��o}
\subsection*{Implementa��o}

\begin{frame}{Implementa��o}
\begin{itemize}
\item<1-> Codifica��o em Java
\item<2-> Biblioteca open source JUNG: 1.7 e 2.0
\item<3-> Diversos algoritmos j� implementados, incluindo o \textcolor{red}{Dijkstra} 
\item<4-> Visualiza��o de grafos
\end{itemize}
\end{frame}

\subsection*{Experimentos}

\begin{frame}
\onslide<1->{
\begin{block}{Vari�veis}
\begin{itemize}
\item $n$: n�mero de v�rtices
\item $m$: n�mero de arestas
\item $d$: densidade=$m/\binom{n}{2}$
\item $k$: quantidade de menores caminhos
\end{itemize}
\end{block}
}
\onslide<2->{
\begin{block}{Metodologia}
\begin{itemize}
\item Grafos sim�tricos conexos com custos positivos 
\item 5 pontas iniciais e finais
\item Um grafo por densidade
\end{itemize}
\end{block}
}
\onslide<3->{
\begin{block}{Consumo de tempo}
\begin{itemize}
\item \KIM: $\Theta(kT(n,m))=\Theta(km\log n)$ usando-se \Dijkstra
\item \KIM: $\Theta(km\log n)=\Theta(kdn^2\log n)$
\end{itemize}
\end{block}
}
\end{frame}

\begin{frame}{Tempo em fun��o do n�mero de caminhos}
\only<1>{\alert{Consumo de tempo \KIM{}: $\Theta(kT(n,m))$}}
\only<2>{
\begin{figure}
\includegraphics[scale=0.8]{graficos_pdf/tempo_x_k_fit_no_residuals.pdf}
\end{figure}
}
\end{frame}


\begin{frame}{Tempo em fun��o do n�mero de v�rtices}
\only<1>{\alert{Consumo de tempo \KIM{} usando \Dijkstra{} "min-heap": $\Theta(kdn^2\log n)$}}
\only<2>{
\begin{figure}
\includegraphics[scale=0.8]{graficos_pdf/tempo_x_n_d_01_fit_no_residuals.pdf}
\end{figure}
}
\only<3>{
\begin{figure}
\includegraphics[scale=0.8]{graficos_pdf/tempo_x_n_d_05_fit_no_residuals.pdf}
\end{figure}
}
\only<4>{
\begin{figure}
\includegraphics[scale=0.8]{graficos_pdf/tempo_x_n_d_10_fit_no_residuals.pdf}
\end{figure}
}
\end{frame}

\begin{frame}{Principais rotinas}
\begin{block}{Rotinas}
\begin{itemize}
\item Problema do desvio m�nimo, rotina \SEP 
\item Problema \PCM, algoritmo Dijkstra
\end{itemize}
\end{block}

\begin{block}{Consumo de tempo}
\begin{itemize}
\item \SEP : $\Theta(m+n)$ 
\item \Dijkstra{} com "min-heap": $\Theta(m\log n)$
\end{itemize}
\end{block}
\end{frame}

\begin{frame}{Fatia de tempo utilizada pelas principais rotinas.}
\begin{figure}
\centering
\includegraphics[scale=0.5]{graficos_pdf/comparativo_percentual_sintetizado_by_densidade.pdf}
\end{figure}
\end{frame}

\begin{frame}{Fatia de tempo utilizada pelas execu��es do Dijkstra.}
\only<1>{
\begin{figure}
\includegraphics[scale=0.5]{graficos_pdf/proporcao_cm_sintetizado_by_n_d_01.pdf}
\end{figure}
}
\only<2>{
\begin{figure}
\includegraphics[scale=0.5]{graficos_pdf/proporcao_cm_sintetizado_by_n_d_05.pdf}
\end{figure}
}
\only<3>{
\begin{figure}
\includegraphics[scale=0.5]{graficos_pdf/proporcao_cm_sintetizado_by_n_d_10.pdf}
\end{figure}
}
\end{frame}

\begin{frame}{Fatia de tempo utilizada pelas execu��es do algoritmo de desvio m�nimo.}
\only<1>{
\begin{figure}
\includegraphics[scale=0.5]{graficos_pdf/proporcao_sep_sintetizado_by_n_d_01.pdf}
\end{figure}
}
\only<2>{
\begin{figure}
\includegraphics[scale=0.5]{graficos_pdf/proporcao_sep_sintetizado_by_n_d_05.pdf}
\end{figure}
}
\only<3>{
\begin{figure}
\includegraphics[scale=0.5]{graficos_pdf/proporcao_sep_sintetizado_by_n_d_10.pdf}
\end{figure}
}
\end{frame}

\section{Conclus�es}
\begin{frame}{Conclus�es}
No algoritmo \KIM{} a rotina para o problema \PCM{} consome a maior fatia do tempo de execu��o do algoritmo.
\end{frame}

\begin{frame}{Trabalhos futuros}
\begin{itemize}
\item Representar caminhos usando TRIE
\item Analisar id�ia de reconstru��o de �rvores
\item Implementar algoritmo de Hershberger
\end{itemize}
\end{frame}

\begin{frame}{FIM}
\begin{quote}
"N�o basta ensinar ao homem uma especialidade, 
porque se tornar� assim uma m�quina utiliz�vel e n�o uma personalidade.\\
� necess�rio que adquira um sentimento, senso pr�tico daquilo que vale a pena ser empreendido, 
daquilo que � belo, do que � moralmente correto."
\end{quote}

\textbf{Albert Einstein}
\end{frame}



\appendix
\section{\appendixname}
\frame{\tableofcontents}

\subsection{KIM}

\begin{frame}{Custo zero nas arestas}
%%%%%%%%%%%%%%%%%%%%%%%%%%%%%%%%%%%%%%%%%%%%%%%%%%%%%%%%%%%%%%%%%%%%%%%%%%%%%%%%%%%
%%%%%%%%%%%%%%%%%%%%%%%%%%%%%%%%%%grafo%%%%%%%%%%%%%%%%%%%%%%%%%%%%%%%%%%%%%%%%%%%5
%%%%%%%%%%%%%%%%%%%%%%%%%%%%%%%%%%%%%%%%%%%%%%%%%%%%%%%%%%%%%%%%%%%%%%%%%%%%%%%%%%%
\begin{figure}
\tikzstyle{vertex}=[circle,fill=black!25,minimum size=15pt,inner sep=0pt]
\tikzstyle{start} = [vertex, fill=red!24]
\tikzstyle{end} = [vertex, fill=red!24]
\tikzstyle{edge} = [draw,thick,-]
\tikzstyle{weight} = [font=\tiny]
\tikzstyle{selected edge} = [draw,line width=2pt,-,red!50]
	\begin{tikzpicture}[scale=1.8, auto,swap,node distance=2cm]
        \node[start] (a) {a};
        \node[vertex] (b) [right of=a] {b};
        \node[vertex] (c) [below right of=b] {c};
        \node[vertex] (d) [right of=b] {d};
        \node[end] (e) [right of=d] {e};
          
         \path[selected edge] (a) -- node[weight] {$1.0$} (b);
         \path[selected edge] (b) -- node[weight] {$0.0$} (d);
         \path[selected edge] (d) -- node[weight] {$1.0$} (e);
         \path[edge] (b) -- node[weight] {$1.0$} (c);
         \path[edge] (c) -- node[weight] {$1.0$} (d);
   \end{tikzpicture}
\end{figure}
%%%%%%%%%%%%%%%%%%%%%%%%%%%%%%%%%%%%%%%%%%%%%%%%%%%%%%%%%%%%%%%%%%%%%%%%%%%%%
\begin{columns}[b]
 \column{.3\textwidth}
   \onslide<2->
	    {%
\begin{figure}
\tikzstyle{node}=[circle,fill=black!25,minimum size=10pt,inner sep=0pt]
\tikzstyle{root} = [node, fill=red!24]
\tikzstyle{leaf} = [node, fill=red!24]
\tikzstyle{onPath} = [node, fill=red!24]
\tikzstyle{edge} = [draw,thick,->]
\tikzstyle{weight} = [font=\tiny]
\tikzstyle{selected edge} = [draw,thick,->,red!50]
%\tikzstyle{LabelStyle}=[fill=white,sloped]
\begin{tikzpicture}[scale=0.8]
% Set the overall layout of the tree
\tikzstyle{level 1}=[level distance=1.5cm, sibling distance=2.5cm]
\tikzstyle{level 2}=[level distance=1.5cm, sibling distance=2.5cm]
  	\node[root,label=right:{\footnotesize $\epsilon=1$}] (a) {a} [grow=down,->] 
  	    child {
				node[onPath,label=right:{\footnotesize $\epsilon=2$}] (ab) {b}
				child{
						node[onPath,label=right:{\footnotesize $\epsilon=3$}](abd){d} 
						child{
							node[onPath,label=left:{\footnotesize $\epsilon=4$}] (abde) {e}
							edge from parent
							node[left,weight] {1.0}
						}
						child {
							node[node,label=right:\textcolor{red}{\footnotesize $\epsilon=3$}] {c}
							edge from parent
							node[right,weight] {1.0}
	 					}
						edge from parent
						node[left,weight] {0.0}
				}
				edge from parent
				node[left,weight] {1.0}
	 }
	     ;
  	\path[selected edge] (a) -- node[weight] {} (ab);
   \path[selected edge] (ab) -- node[weight] {} (abd);
   \path[selected edge] (abd) -- node[weight] {} (abde);   
	\end{tikzpicture}
	
	\end{figure}
}
 \column{.3\textwidth}
   \onslide<2->
	    {%
\begin{figure}
\tikzstyle{node}=[circle,fill=black!25,minimum size=10pt,inner sep=0pt]
\tikzstyle{root} = [node, fill=red!24]
\tikzstyle{leaf} = [node, fill=red!24]
\tikzstyle{onPath} = [node, fill=red!24]
\tikzstyle{edge} = [draw,thick,->]
\tikzstyle{weight} = [font=\tiny]
\tikzstyle{selected edge} = [draw,thick,->,red!50]
%\tikzstyle{LabelStyle}=[fill=white,sloped]
\begin{tikzpicture}[scale=0.8]
% Set the overall layout of the tree
\tikzstyle{level 1}=[level distance=1.5cm, sibling distance=2.5cm]
\tikzstyle{level 2}=[level distance=1.5cm, sibling distance=2.5cm]
  	\node[root,label=right:{\footnotesize $\zeta=4$}] (e) {e} [grow=up,->] 
  	    child {
				node[onPath,label=right:{\footnotesize $\zeta=3$}] (ed) {d} 
				child{
						node[onPath,label=right:{\footnotesize $\zeta=2$}] (edb) {b}  
						child{
							node[onPath,label=right:{\footnotesize $\zeta=1$}] (edba) {a} 
							edge from parent
							node[right,weight] {1.0}
						}
						child {
							node[node,label=left:\textcolor{red}{\footnotesize $\zeta=2$}] {c}
							edge from parent
							node[left,weight] {1.0}
	 					}
						edge from parent
						node[left,weight] {0.0}
				}
				edge from parent
				node[left,weight] {1.0}
	 }
	     ;
 	\path[selected edge] (e) -- node[weight] {} (ed);
   \path[selected edge] (ed) -- node[weight] {} (edb);
   \path[selected edge] (edb) -- node[weight] {} (edba);
	\end{tikzpicture}
	
	\end{figure}
}
	\column{.4\textwidth}
   \onslide<3->
	    {%
		\begin{center}
		\begin{block}{Problema}
		\begin{itemize}
			\item {\small $\epsilon(c)>\zeta(c)$}
			\item {\small N�o consegue gerar o caminho $\seq{a,b,c,d,e}$} 
		\end{itemize}
		\end{block}

		\end{center}
	}
\end{columns}
\end{frame}

\begin{frame}{$P_a$}
\only<1>{
Na figura temos os caminhos $P_j$ e $P_k$, $1<=j<k$ onde $P_j$ � pai de $P_k$ e que compartilham o prefixo $\seq{s,\ldots,u}$
diferenciando-se a partir de $u$.
}
\only<2>{
Esquema dos caminhos na parti��o $P_a$.
}
\begin{figure}[c]
\tikzstyle{node}=[circle,fill=black!25,minimum size=10pt,inner sep=0pt]
\tikzstyle{root} = [node, fill=red!24]
\tikzstyle{leaf} = [node, fill=red!24]
\tikzstyle{deviation} = [node, fill=blue]
\tikzstyle{onPath} = [node, dashed,fill=red!24]
\tikzstyle{edge} = [draw,thick,->]
\tikzstyle{unknown} = [dashed,thick,->]
\tikzstyle{novos} = [node, fill=red!50]
\tikzstyle{weight} = [font=\tiny]
\tikzstyle{selected edge} = [draw,thick,->,red!50]
%\tikzstyle{LabelStyle}=[fill=white,sloped]
\only<1>{
\begin{tikzpicture}[scale=1]
% Set the overall layout of the tree
\tikzstyle{level 1}=[level distance=1.5cm, sibling distance=2.5cm]
\tikzstyle{level 2}=[level distance=1.5cm, sibling distance=2.5cm]
\node[root,label=right:{s}] {}
child {
				node[deviation,label=right:{u}] {}
				child[->,solid] {
						node[node] {a}
						child[->,dashed]{
							node[leaf] {$t_j$}
						}
				}
				child[->,solid] {
					node[node] {b}
						child[grow=down,dashed]{
							node[leaf] {$t_{k}$} 
						}
				}
				edge from parent[dashed]
};
\end{tikzpicture}
}
\only<2> 
{
\begin{tikzpicture}[scale=1]
% Set the overall layout of the tree
\tikzstyle{level 1}=[level distance=1.5cm, sibling distance=2.5cm]
\tikzstyle{level 2}=[level distance=1.5cm, sibling distance=2.5cm]
\node[root,label=right:{s}] (s) {}
child {
				node[deviation,label=right:{u}] {}
				child[->,solid] {
						node[node] {a}
						child[->,dashed]{
							node[leaf] {$t_j$}
						}
				}
				child[->,solid] {
					node[node] {b}
						child[sibling distance=10mm,level distance=1cm,dashed,->,line width=2pt]{node[novos]{}}
						child[sibling distance=10mm,level distance=1cm,dashed,->]{
							child[dashed,line width=2pt]{node[novos]{}}
							child[line width=2pt]{node[novos]{}}
							child[level distance=1.5cm,grow=down]{
								node[leaf] {$t_{k}$} 
							}
							child[line width=2pt]{node[novos]{}}
							child[line width=2pt]{node[novos]{}}
						}
						child[sibling distance=10mm,level distance=1cm,dashed,->,line width=2pt]{node[novos]{}}
					}
				edge from parent[dashed]
};
\end{tikzpicture}

}

\end{figure}
\end{frame}

\begin{frame}{$P_b$}
\begin{figure}
\tikzstyle{node}=[circle,fill=black!25,minimum size=10pt,inner sep=0pt]
\tikzstyle{root} = [node, fill=red!24]
\tikzstyle{leaf} = [node, fill=red!24]
\tikzstyle{novos} = [node, fill=red!70]
\tikzstyle{deviation} = [node, fill=blue]
\tikzstyle{onPath} = [node, dashed,fill=red!24]
\tikzstyle{edge} = [draw,thick,->]
\tikzstyle{unknown} = [dashed,thick,->]
\tikzstyle{filhos} = [node,fill=green!50]
\tikzstyle{weight} = [font=\tiny]
\tikzstyle{selected edge} = [draw,thick,->,red!50]
%\tikzstyle{LabelStyle}=[fill=white,sloped]
\only<1>{
\begin{tikzpicture}[scale=1]
% Set the overall layout of the tree
\tikzstyle{level 1}=[level distance=1.5cm, sibling distance=2.5cm]
\tikzstyle{level 2}=[level distance=1cm, sibling distance=2cm]
\node[root,label=right:{s}] {}
child {
				node[deviation,label=right:{u}] {}
				child[->,dashed]{node[filhos]{}}
				child[->,dashed] {
						child[->,dashed]{node[filhos]{}}
						child[->,dashed,grow=down]{
							child[->,dashed]{
													node[filhos]{}
							}
							child[->,dashed,grow=down]{
													child[->,dashed]{
																			node[filhos]{}
													}
													child[->,dashed,grow=down]{
																			node[leaf] {$t_j$}
													}
							}
						}
				}
				child[->,solid] {
					node[node] {b}
						child[grow=down,dashed]{
							node[leaf] {$t_{k}$} 
						}
				}
				edge from parent[dashed]
};
\end{tikzpicture}
}
\only<2>{
\begin{tikzpicture}[scale=1]
% Set the overall layout of the tree
\tikzstyle{level 1}=[level distance=1.5cm, sibling distance=2.5cm]
\tikzstyle{level 2}=[level distance=1cm, sibling distance=2cm]
\node[root,label=right:{s}] {}
child {
				node[deviation,label=right:{u}] {}
				child[->,dashed]{node[filhos]{}}
				child[->,dashed] {
						node[deviation,label=right:{v}] {}
						child[->,dashed]{node[filhos]{$i_l$}}
						child[->,dashed,grow=down]{
							child[->,dashed]{
													node[filhos]{}
							}
							child[->,dashed,grow=down]{
													child[->,dashed]{
																			node[filhos]{}
													}
													child[->,dashed,grow=down]{
																			node[leaf] {$i_j$}
													}
							}
						}
				}
				child[->,solid] {
					node[node] {b}
						child[grow=down,dashed]{
							node[leaf] {$i_{k}$} 
						}
				}
				edge from parent[dashed]
};
\end{tikzpicture}
}
\only<3>{
\begin{tikzpicture}[scale=1]
% Set the overall layout of the tree
\tikzstyle{level 1}=[level distance=1.5cm, sibling distance=2.5cm]
\tikzstyle{level 2}=[level distance=1cm, sibling distance=2cm]
\node[root,label=right:{s}] {}
child {
				node[deviation,label=right:{u}] {}
				child[->,dashed,line width=2pt]{node[novos]{}}
				child[->,dashed]{node[filhos]{}}
				child[grow=down]{
					child[->,dashed,line width=2pt]{node[novos]{}}
					child[grow=down,level distance=5mm]{
						child[level distance=10mm,grow=south east,->,dashed,line width=2pt]{node[novos]{}}
						child[level distance=15mm,->,dashed,grow=down] {
								node[deviation,label=right:{v}] {}
								child[level distance=10mm,->,dashed]{node[leaf]{$t_l$}}
								child[level distance=10mm,->,dashed,grow=down]{
										node[leaf] {$t_j$}
								}
						}
					}
				}
				child[->,solid] {
					node[node] {b}
						child[grow=down,dashed]{
							node[leaf] {$t_k$} 
						}
				}
				edge from parent[dashed]
};
\end{tikzpicture}
}
\end{figure}
\end{frame}

\begin{frame}{$P_c$}
\begin{figure}
\tikzstyle{node}=[circle,fill=black!25,minimum size=10pt,inner sep=0pt]
\tikzstyle{root} = [node, fill=red!24]
\tikzstyle{leaf} = [node, fill=red!24]
\tikzstyle{novos} = [node, fill=red!70]
\tikzstyle{deviation} = [node, fill=blue]
\tikzstyle{onPath} = [node, dashed,fill=red!24]
\tikzstyle{edge} = [draw,thick,->]
\tikzstyle{unknown} = [dashed,thick,->]
\tikzstyle{filhos} = [node,fill=green!50]
\tikzstyle{weight} = [font=\tiny]
\tikzstyle{selected edge} = [draw,thick,->,red!50]
%\tikzstyle{LabelStyle}=[fill=white,sloped]
\only<1>{
\begin{tikzpicture}[scale=1]
% Set the overall layout of the tree
\tikzstyle{level 1}=[level distance=5mm, sibling distance=2cm]
\tikzstyle{level 2}=[level distance=5mm, sibling distance=2cm]
\node[root,label=right:{s}] {}
child {
	child[->,dashed]{node[filhos]{}}
		child[-,dashed,grow=down]{
			child[-,dashed]{
				node{$\ldots$} edge from parent[draw=none]
			}
			child[->,dashed,grow=down]{
				child[->,dashed]{
					node[filhos]{}
				}
				child[level distance=1.5cm,->,dashed,grow=down]{	
					node[deviation,label=right:{u}] {}
					child[grow=down,dashed]{
						node[leaf] {$t_{k}$} 
					}
					child{
						node[leaf] {$t_j$}
					}
				}
			}
		}
	edge from parent[dashed]
};
\end{tikzpicture}
}
\only<2>{
\begin{tikzpicture}[scale=1]
% Set the overall layout of the tree
\tikzstyle{level 1}=[level distance=5mm, sibling distance=2cm]
\tikzstyle{level 2}=[level distance=5mm, sibling distance=2cm]
\node[root,label=right:{s}] {}
child {
	child[->,dashed]{node[leaf]{}}
		child[-,dashed,grow=down]{
			child[-,dashed]{
				node{$\ldots$} edge from parent[draw=none]
			}
			child[->,dashed,grow=down]{
				node[deviation,label=right:{v}]{}
				child[->,dashed]{
					node[filhos]{$t_l$}
				}
				child[level distance=1.5cm,->,dashed,grow=down]{	
					node[deviation,label=right:{u}] {}
					child[grow=down,dashed]{
						node[leaf] {$t_{k}$} 
					}
					child{
						node[leaf] {$t_j$}
					}
				}
			}
		}
	edge from parent[dashed]
};
\end{tikzpicture}
}
\only<3>{
\begin{tikzpicture}[scale=1]
% Set the overall layout of the tree
\tikzstyle{level 1}=[level distance=5mm, sibling distance=2cm]
\tikzstyle{level 2}=[level distance=5mm, sibling distance=2cm]
\node[root,label=right:{s}] {}
child {
		child[-,dashed,grow=down]{
			child[->,dashed,grow=down]{
				node[deviation,label=right:{v}]{}
				child[->,dashed,line width=2pt]{
					node[novos]{}
				}
				child[-,grow=down]{
					child[->,line width=2pt]{node[novos]{}}
					child[-,grow=down]{
						child{node{$\ldots$} edge from parent[draw=none]}
						child[-,grow=down]{
							child[->,line width=2pt]{node[novos]{}}
							child[level distance=1.5cm,->,dashed,grow=down]{	
								node[deviation,label=right:{u}] {}
								child[grow=down,dashed]{
									node[leaf] {$t_{k}$} 
								}
								child{
									node[leaf] {$t_j$}
								}
							}
						}
					}
				}
				child[->,dashed]{
					node[filhos]{$t_l$}
				}
			}
		}
	edge from parent[dashed]
};
\end{tikzpicture}
}
\end{figure}
\end{frame}

\subsection{�rvore dos prefixos}
\begin{frame}{Defini��o}
\framesubtitle{\textbf{$\Qcal,(N,E),f,V(\Qcal),A(\Qcal)$}}
\begin{itemize}
	\only<1>{
        \item $\Qcal$: uma cole��o de caminhos
        \begin{columns}
          \column{.3\textwidth}
        \begin{block}{Caminhos}
            {$\seq{s,a,i,t}$} \\
            {$\seq{s,b,f,l,t}$} \\
            {$\seq{s,b,g,l,t}$} \\
            {$\seq{s,b,h,l,t}$} \\
            {$\seq{s,b,h,i,t}$} \\
            {$\seq{s,a,i,h,l,t}$}
        \end{block}
        \end{columns}
        \mbox{} \\
        \mbox{} \\
        \mbox{} \\
     }
	\only<2>{
        \item $(N,E)$: uma arboresc�ncia
        \begin{center}
        \begin{tikzpicture}[auto,thick]
          \tikzstyle{node}=%
          [%
            minimum size=10pt,%
            inner sep=0pt,%
            outer sep=0pt,%
            ball color=example text.fg,%
            circle%
          ]

          \node [node] {} [->]
            child {node [node] {} edge from parent node[swap]{a}}
            child {node [node] {} edge from parent node{e}}
            child {node [node] {} 
                child {node [node] {} edge from parent node{c}}
                edge from parent node{b}
            };
        \end{tikzpicture}
        \end{center}
        \begin{itemize}
        \item Grafo ac�clico $(N,E)$ 
        \item $|N| = |E| + 1$ 
        \item \defi{raiz}
        \end{itemize}
        Todo v�rtice, exceto a \defi{raiz}, � ponta final de exatamente um arco.
        }
	\only<3>{
        \item $f$: uma \defi{fun��o r�tulo} 
  \begin{block}{}
	\textbf{Associa n�s e arestas do grafo aos n�s e arestas da �rvore de prefixos.}
  \end{block}
  Se \begin{eqnarray*}
  R=\seq{u_{0}, e_{1}, u_{1}, \ldots, e_{t}, u_{t}}
  \end{eqnarray*}
  for um caminho em  $(N,E)$, ent�o
  \begin{eqnarray*}
  f(R):=\seq{f(u_{0}), f(e_{1}), f(u_{1}), \ldots, f(e_{t}), f(u_{t})}
  \end{eqnarray*}
  ser� uma seq��ncia de v�rtices e arcos dos caminhos em $\Qcal$.
        \mbox{} \\
        \mbox{} \\
        \mbox{} \\
        \mbox{} \\
        \mbox{} \\
        }
        \only<1>{
	\item $V(\Qcal)$: conjunto de v�rtices
	\item $A(\Qcal)$: conjunto de arcos
        }
\end{itemize}
\end{frame}

%Seja $(N,E)$ uma arboresc�ncia e  
%$f$ uma \defi{fun��o r�tulo} que associa a cada n� em 
%$N$ um v�rtice em $V(\Qcal)$ e a
%cada arco em $E$ um arco em $A(\Qcal)$. 

%\begin{frame}{Defini��o}
%$(N,E,f)$ � \defi{�rvore dos prefixos} de $\Qcal$ se 
%\begin{itemize}
%\item para cada caminho $R$ em $(N,E)$ com in�cio na
%      raiz, $f(R)$ for prefixo de algum caminho em $\Qcal$; e
%\item para cada prefixo $Q$ de algum caminho em $\Qcal$
%      existir um caminho $R$ em $(N,E)$ com in�cio na
%      raiz tal que $f(R)=Q$; e
%\item o caminho $R$ do item anterior for �nico. 
%\end{itemize}
%\end{frame}



\end{document}


\section{Proposta}
\begin{frame}{Melhoria do desempenho do KIM}
\onslide<1->
{%
\begin{block}{Considera��es sobre o Algorimo KIM}
	\begin{itemize}
		\item Principal subrotina: �rvores de caminhos m�nimos.
		\item Grafos intermedi�rios semelhantes.	
		\item Custos discretos
	\end{itemize}
\end{block}
}
\onslide<2->
{%
\begin{block}{Id�ias}
	\begin{itemize}
		\item Reconstru��o de �rvores de caminhos m�nimos
		\item Mais caminhos candidatos por itera��o
	\end{itemize}
\end{block}
}
\end{frame}

\begin{frame}{Plano}
\begin{enumerate}
\item Estudo de YEN
\item Estudo de KIM
\item Estudo de Reconstru��o de �rvores
\item Gera��o de mais caminhos candidatos
\item Comparar implementa��es: 
	\begin{itemize}
		\item YEN
		\item KIM
		\item KIM com reconstru��o
		\item KIM com mais caminhos candidatos
	\end{itemize}
\end{enumerate}
\end{frame}

%\end{document}

\begin{frame}[t]{General formalization of haplotyping.}
  \begin{block}{Inputs}
    \begin{itemize}
    \item A \alert{genotype matrix} $G$.
    \item The \alert{rows} of the matrix are \alert{taxa / individuals}.
    \item The \alert{columns} of the matrix are \alert{SNP sites /
        characters}. 
    \end{itemize}
  \end{block}
  \begin{block}{Outputs}
    \begin{itemize}
    \item A \alert{haplotype matrix} $H$.
    \item Pairs of rows in $H$ \alert{explain} the rows of $G$.
    \item The haplotypes in $H$ are \alert{biologically plausible}. 
    \end{itemize}
  \end{block}
\end{frame}


\begin{frame}[t]{Our formalization of haplotyping.}
  \begin{block}{Inputs}
    \begin{itemize}
    \item A genotype matrix $G$.
    \item The rows of the matrix are individuals / taxa.
    \item The columns of the matrix are SNP sites / characters.
    \item<alert@1->
      The problem is directed: one haplotype is known.
    \item<alert@1->
      The input is biallelic: there are only two homozygous
      states (0 and 1) and one heterozygous state (2).
    \end{itemize}
  \end{block}
  \begin{block}{Outputs}
    \begin{itemize}
    \item A haplotype matrix $H$.
    \item Pairs of rows in $H$ explain the rows of $G$.
    \item<alert@1> The haplotypes in $H$ form a perfect phylogeny.
    \end{itemize}
  \end{block}
\end{frame}


\begin{frame}{We can do perfect phylogeny haplotyping efficiently, but
    \dots}
  \begin{enumerate}
  \item \alert{Data may be missing.}
    \begin{itemize}
    \item This makes the problem NP-complete \dots
    \item \dots even for very restricted cases.
    \end{itemize}
    \textcolor{green!50!black}{Solutions:}
    \begin{itemize}
    \item Additional assumption like the rich data hypothesis. 
    \end{itemize}
  \item \alert{No perfect phylogeny is possible.}
    \begin{itemize}
    \item This can be caused by chromosomal crossing-over effects.
    \item This can be caused by incorrect data.
    \item This can be caused by multiple mutations at the same sites.
    \end{itemize}
    \textcolor{green!50!black}{Solutions:}
    \begin{itemize}
    \item Look for phylogenetic networks.
    \item Correct data.
    \item<alert@1->
       Find blocks where a perfect phylogeny is possible.
    \end{itemize}
  \end{enumerate}
\end{frame}


\subsection{The Integrated Approach}

\begin{frame}{How blocks help in perfect phylogeny haplotyping.}
  \begin{enumerate}
  \item Partition the site set into overlapping contiguous blocks.
  \item Compute a perfect phylogeny for each block and combine them.
  \item Use dynamic programming for finding the partition.
  \end{enumerate}

  \begin{tikzpicture}
    \useasboundingbox (0,-1) rectangle (10,2);
    
    \draw[line width=2mm,dash pattern=on 1mm off 1mm]
      (0,1) -- (9.99,1) node[midway,above] {Genotype matrix}
      (0,0.6666) -- (9.99,0.6666)
      (0,0.3333) -- (9.99,0.3333)
      (0,0) -- (9.99,0) node[midway,below] {\only<1>{no perfect phylogeny}};

    \begin{scope}[xshift=-.5mm]
      \only<2->
      {
        \draw[red,block]            (0,.5)   -- (3,.5)
          node[midway,below] {perfect phylogeny};
      }
        
      \only<3->
      {
        \draw[green!50!black,block] (2.5,.5)   -- (7,.5)
          node[pos=0.6,below] {perfect phylogeny};
      }

      \only<4->
      {
        \draw[blue,block]           (6.5,.5) -- (10,.5)
          node[pos=0.6,below] {perfect phylogeny};
      }
    \end{scope}
  \end{tikzpicture}
\end{frame}

\begin{frame}{Objective of the integrated approach.}
  \begin{enumerate}
  \item Partition the site set into \alert{noncontiguous} blocks. 
  \item Compute a perfect phylogeny for each block and combine them. 
  \item<alert@1-> Compute partition while computing perfect
    phylogenies. 
  \end{enumerate}

  \begin{tikzpicture}
    \useasboundingbox (0,-1) rectangle (10,2);

    \draw[line width=2mm,dash pattern=on 1mm off 1mm]
      (0,1) -- (9.99,1) node[midway,above] {Genotype matrix}
      (0,0.6666) -- (9.99,0.6666)
      (0,0.3333) -- (9.99,0.3333)
      (0,0) -- (9.99,0) node[midway,below] {\only<1>{no perfect phylogeny}};

    \only<2->
    {
      \begin{scope}[xshift=-0.5mm]
        \draw[red,block] (0,.5)   -- (3,.5) 
          node[midway,below] {perfect phylogeny}
                         (8,.5) -- (9,.5);

        \draw[green!50!black,block]
          (3,.5)   -- (6,.5)
            node[pos=0.6,below] {perfect phylogeny}
          (6.4,.5)   -- (8,.5)
          (9,.5) -- (10,.5);

        \draw[blue,block] (6,.5) -- (6.4,.5)
          node[midway,below=5mm] {perfect phylogeny};
      \end{scope}
    }
  \end{tikzpicture}
\end{frame}


\begin{frame}{The formal computational problem.}
  We are interested in the computational complexity of \\
  \alert{the function \alert{$\chi_{\operatorname{PP}}$}}:
  \begin{itemize}
  \item It gets genotype matrices as input.
  \item It maps them to a number $k$.
  \item This number is minimal such that the sites can be
    covered by $k$ sets, each admitting a perfect phylogeny.
    \\
    (We call this a \alert{pp-partition}.)
  \end{itemize}
\end{frame}


\section{Bad News: Hardness Results}

\subsection{Hardness of PP-Partitioning of Haplotype Matrices}

\begin{frame}{Finding pp-partitions of haplotype matrices.}
  We start with a special case:
  \begin{itemize}
  \item The inputs $M$ are \alert{already haplotype matrices}.
  \item The inputs $M$ \alert{do not allow a perfect phylogeny}.
  \item What is $\chi_{\operatorname{PP}}(M)$?
  \end{itemize}
  \begin{example}
    \begin{columns}
      \column{.3\textwidth}
      $M\colon$
      \footnotesize
      \begin{tabular}{cccc}
        0 & 0 & 0 & 1 \\
        0 & 1 & 0 & 0 \\
        1 & 0 & 0 & 0 \\
        0 & 1 & 0 & 0 \\
        1 & 0 & 0 & 0 \\
        0 & 1 & 0 & 1 \\
        1 & 1 & 0 & 0 \\
        0 & 0 & 1 & 0 \\
        1 & 0 & 1 & 0
      \end{tabular}%
      \only<2>
      {%
        \begin{tikzpicture}
          \useasboundingbox (2.9,0);

          \draw [red, opacity=0.7,line width=1cm] (1.7 ,1.9) -- (1.7 ,-1.7);
          \draw [blue,opacity=0.7,line width=5mm] (0.85,1.9) -- (0.85,-1.7)
                                                  (2.55,1.9) -- (2.55,-1.7);
        \end{tikzpicture}
      }
      \column{.6\textwidth}
      \begin{overprint}
        \onslide<1>
        No perfect phylogeny is possible.
        
        \onslide<2>
        \textcolor{blue!70!bg}{Perfect phylogeny}
        
        \textcolor{red!70!bg}{Perfect phylogeny}
        
        $\chi_{\operatorname{PP}}(M) = 2$.
        
      \end{overprint}
    \end{columns}
  \end{example}
\end{frame}

\begin{frame}{Bad news about pp-partitions of haplotype matrices.}
  \begin{theorem}
    Finding \alert{optimal pp-partition of haplotype matrices}\\
    is equivalent to finding \alert{optimal graph colorings}.
  \end{theorem}

  \begin{proof}[Proof sketch for first direction]
    \begin{enumerate}
    \item Let $G$ be a graph.
    \item Build a matrix with a column for each vertex of $G$.
    \item For each edge of $G$ add four rows inducing\\the
      submatrix $\left(
        \begin{smallmatrix}
          0 & 0 \\
          0 & 1 \\
          1 & 0 \\
          1 & 1
        \end{smallmatrix}\right)$.
    \item The submatrix enforces that the columns lie in different
      perfect phylogenies. \qedhere  
    \end{enumerate}
  \end{proof}
\end{frame}

\begin{frame}{Implications for pp-partitions of haplotype matrices.}
  \begin{corollary}
    If $\chi_{\operatorname{PP}}(M) = 2$ for a haplotype matrix $M$,
    we can find an optimal pp-partition in polynomial time. 
  \end{corollary}

  \begin{corollary}
    Computing $\chi_{\operatorname{PP}}$ for haplotype matrices is
    \begin{itemize}
    \item $\operatorname{NP}$-hard,
    \item not fixed-parameter tractable, unless
      $\operatorname{P}=\operatorname{NP}$, 
    \item very hard to approximate.
    \end{itemize}
  \end{corollary}
\end{frame}


\subsection{Hardness of PP-Partitioning of Genotype Matrices}


\begin{frame}{Finding pp-partitions of genotype matrices.}
  Now comes the general case:
  \begin{itemize}
  \item The inputs $M$ are \alert{genotype matrices}.
  \item The inputs $M$ \alert{do not allow a perfect phylogeny}.
  \item What is $\chi_{\operatorname{PP}}(M)$?
  \end{itemize}
  \begin{example}
    \begin{columns}
      \column{.3\textwidth}
      $M\colon$
      \footnotesize
      \begin{tabular}{cccc}
        2 & 2 & 2 & 2 \\
        1 & 0 & 0 & 0 \\
        0 & 0 & 0 & 1 \\
        0 & 0 & 1 & 0 \\
        0 & 2 & 2 & 0 \\
        1 & 1 & 0 & 0 
      \end{tabular}%
      \only<2>
      {%
        \begin{tikzpicture}
          \useasboundingbox (2.9,0);
          
          \draw [red, opacity=0.7,line width=1cm] (1.7 ,1.3) -- (1.7 ,-1.1);
          \draw [blue,opacity=0.7,line width=5mm] (0.85,1.3) -- (0.85,-1.1)
                                                  (2.55,1.3) -- (2.55,-1.1);
        \end{tikzpicture}
      }
      \column{.6\textwidth}
      \begin{overprint}
        \onslide<1>
        No perfect phylogeny is possible.
        
        \onslide<2>
        \textcolor{blue!70!bg}{Perfect phylogeny}
        
        \textcolor{red!70!bg}{Perfect phylogeny}
        
        $\chi_{\operatorname{PP}}(M) = 2$.
        
      \end{overprint}
    \end{columns}
  \end{example}
\end{frame}


\begin{frame}{Bad news about pp-partitions of haplotype matrices.}
  \begin{theorem}
    Finding \alert{optimal pp-partition of genotype matrices}
    is at least as hard as finding \alert{optimal colorings of
      3-uniform hypergraphs}. 
  \end{theorem}

  \begin{proof}[Proof sketch]
    \begin{enumerate}
    \item Let $G$ be a 3-uniform hypergraph.
    \item Build a matrix with a column for each vertex of $G$.
    \item For each hyperedge of $G$ add four rows inducing\\ the submatrix
      $\left(
        \begin{smallmatrix}
          2 & 2 & 2 \\
          1 & 0 & 0 \\
          0 & 1 & 0 \\
          0 & 0 & 1
        \end{smallmatrix}\right)
      $.
    \item The submatrix enforces that the three columns do not all lie
      in the same perfect phylogeny. \qedhere
    \end{enumerate}
  \end{proof}
\end{frame}

\begin{frame}{Implications for pp-partitions of genotype matrices.}
  \begin{corollary}
    Even if we know $\chi_{\operatorname{PP}}(M) = 2$ for a genotype matrix $M$,\\
    finding a pp-partition of any fixed size is still
    \begin{itemize}
    \item $\operatorname{NP}$-hard,
    \item not fixed-parameter tractable, unless
      $\operatorname{P}=\operatorname{NP}$, 
    \item very hard to approximate.
    \end{itemize}
  \end{corollary}
\end{frame}


\section{Good News: Tractability Results}

\subsection{Perfect Path Phylogenies}

\begin{frame}{Automatic optimal pp-partitioning is hopeless, but\dots}
  \begin{itemize}
  \item The hardness results are \alert{worst-case} results for\\
    \alert{highly artificial inputs}.
  \item \alert{Real biological data} might have special properties
    that make the problem \alert{tractable}.
  \item One such property is that perfect phylogenies are often
    perfect \alert{path} phylogenies:

    In HapMap data, in 70\% of the blocks where a perfect phylogeny
    is possible a perfect path phylogeny is also possible.
  \end{itemize}  
\end{frame}


\begin{frame}{Example of a perfect path phylogeny.}
  \begin{columns}[t]
    \column{.3\textwidth}
    \begin{exampleblock}{Genotype matrix}
      $G\colon$
      \begin{tabular}{ccc}
        A & B & C \\\hline
        2 & 2 & 2 \\
        0 & 2 & 0 \\
        2 & 0 & 0 \\
        0 & 2 & 2 
      \end{tabular}
    \end{exampleblock}

    \column{.3\textwidth}
    \begin{exampleblock}{Haplotype matrix}
      $H\colon$
      \begin{tabular}{ccc}
        A & B & C \\\hline
        1 & 0 & 0 \\
        0 & 1 & 1 \\
        0 & 0 & 0 \\
        0 & 1 & 0 \\
        0 & 0 & 0 \\
        1 & 0 & 0 \\
        0 & 0 & 0 \\
        0 & 1 & 1 
      \end{tabular}
    \end{exampleblock}

    \column{.4\textwidth}
    \begin{exampleblock}{Perfect path phylogeny}
      \begin{center}
        \begin{tikzpicture}[auto,thick]
          \tikzstyle{node}=%
          [%
            minimum size=10pt,%
            inner sep=0pt,%
            outer sep=0pt,%
            ball color=example text.fg,%
            circle%
          ]
        
          \node [node] {} [->]
            child {node [node] {} edge from parent node[swap]{A}}
            child {node [node] {}
              child {node [node] {} edge from parent node{C}}
              edge from parent node{B}
            };
        \end{tikzpicture}
      \end{center}
    \end{exampleblock}
  \end{columns}
\end{frame}


\begin{frame}{The modified formal computational problem.}
  We are interested in the computational complexity of \\
  the function $\chi_{\alert{\operatorname{PPP}}}$:
  \begin{itemize}
  \item It gets genotype matrices as input.
  \item It maps them to a number $k$.
  \item This number is minimal such that the sites can be
    covered by $k$ sets, each admitting a perfect \alert{path} phylogeny.
    \\
    (We call this a ppp-partition.)
  \end{itemize}
\end{frame}



\subsection{Tractability of PPP-Partitioning of Genotype Matrices}

\begin{frame}{Good news about ppp-partitions of genotype matrices.}
  \begin{theorem}
    \alert{Optimal ppp-partitions of genotype matrices} can be
    computed in \alert{polynomial time}. 
  \end{theorem}
  \begin{block}{Algorithm}
    \begin{enumerate}
    \item Build the following partial order:
      \begin{itemize}
      \item Can one column be above the other in a phylogeny?
      \item Can the columns be the two children of the root of a
        perfect path phylogeny?
      \end{itemize}
    \item Cover the partial order with as few compatible chain pairs 
      as possible. 

      For this, a maximal matching in a special graph needs to be
      computed.
    \end{enumerate}
  \end{block}
  \hyperlink{algorithm<1>}{\beamergotobutton{The algorithm in action}}
  \hypertarget{return}{}
\end{frame}

\section*{Summary}

\begin{frame}
  \frametitle<presentation>{Summary}

  \begin{itemize}
  \item
    Finding optimal pp-partitions is \alert{intractable}. 
  \item
    It is even intractable to find a pp-partition when \alert{just two 
      noncontiguous  blocks are known to suffice}.
  \item
    For perfect \alert{path} phylogenies, optimal partitions can be
    computed \alert{in polynomial time}.
  \end{itemize}
\end{frame}


\appendix

\section*{Appendix}

\begin{frame}[label=algorithm]{The algorithm in action.}{Computation of
    the partial order.}
  \begin{columns}[t]
    \column{.4\textwidth}
    \begin{exampleblock}{Genotype matrix}
      $G\colon$
      \begin{tabular}{ccccc}
        A & B & C & D & E \\\hline
        2 & 2 & 2 & 2 & 2 \\
        0 & 1 & 2 & 1 & 0 \\
        1 & 0 & 0 & 1 & 2 \\
        0 & 2 & 2 & 0 & 0
      \end{tabular}
    \end{exampleblock}
    \column{.6\textwidth}
    \begin{exampleblock}{Partial order}
      \begin{tikzpicture}[node distance=15mm]
        \tikzstyle{every node}=
        [%
          fill=green!50!black!20,%
          draw=green!50!black,%
          minimum size=7mm,%
          circle,%
          thick%
        ]

        \node (A) {A};
        \node (B) [right of=A] {B};
        \node (C) [below of=B] {C};
        \node (D) [above of=A] {D};
        \node (E) [below of=A] {E};

        \path [thick,shorten >=1pt,-stealth'] (A) edge (E)
                         (B) edge (C)
                         (D) edge (A)
                             edge[bend right] (E);

        \uncover<2>{
        \path [-,blue,thick](A) edge (B)
                                edge (C)  
                            (B) edge (E)
                            (C) edge (E);}
      \end{tikzpicture}

      Partial order: \tikz[baseline] \draw[thick,-stealth'] (0pt,.5ex)
      -- (5mm,.5ex); 

      \uncover<2>{\textcolor{blue}{Compatible as children of root:
          \tikz[baseline] \draw[thick] (0pt,.5ex) -- (5mm,.5ex);}} 
    \end{exampleblock}
  \end{columns}  
\end{frame}

\begin{frame}{The algorithm in action.}{The matching in the special graph.}
  \begin{columns}[t]
    \column{.3\textwidth}
    \begin{exampleblock}{Partial order}
      \begin{tikzpicture}[node distance=15mm]
        \tikzstyle{every node}=%
        [%
          fill=green!50!black!20,%
          draw=green!50!black,%
          minimum size=8mm,%
          circle,%
          thick%
        ]

        \node (A)              {$A$};
        \node (B) [right of=A] {$B$};
        \node (C) [below of=B] {$C$};
        \node (D) [above of=A] {$D$};
        \node (E) [below of=A] {$E$};

        \path [thick,shorten >=1pt,-stealth'] (A) edge (E)
                         (B) edge (C)
                         (D) edge (A)
                             edge[bend right] (E);

        \path [-,blue,thick](A) edge (B)
                                edge (C)  
                            (B) edge (E)
                            (C) edge (E);

        \only<3->
        {
          \path[very thick,shorten >=1pt,-stealth',red] (D) edge (A) (B) edge (C);
          \path [-,red,very thick](E) edge (B);
        }
      \end{tikzpicture}
    \end{exampleblock}
    \column{.7\textwidth}
    \begin{exampleblock}{Matching graph}
      \begin{tikzpicture}[node distance=15mm]
        \tikzstyle{every node}=%
        [%
          fill=green!50!black!20,%
          draw=green!50!black,%
          minimum size=8mm,%
          circle,%
          thick,%
          inner sep=0pt%
        ]

        \node (A)              {$A$};
        \node (B) [right of=A] {$B$};
        \node (C) [below of=B] {$C$};
        \node (D) [above of=A] {$D$};
        \node (E) [below of=A] {$E$};

        \begin{scope}[xshift=4.75cm]
          \node (A')               {$A'$};
          \node (B') [right of=A'] {$B'$};
          \node (C') [below of=B'] {$C'$};
          \node (D') [above of=A'] {$D'$};
          \node (E') [below of=A'] {$E'$};
        \end{scope}
        
        \path [thick]    (A) edge (E')
                         (B) edge (C')
                         (D) edge (A')
                             edge (E');

        \path [blue,thick](A') edge (B')
                               edge (C')  
                          (B') edge (E')
                          (C') edge (E');

        \only<2->
        {
          \path[very thick,red] (D) edge (A')
                           (B) edge (C')
                           (B') edge (E');
        }
      \end{tikzpicture}
    \end{exampleblock}
  \end{columns}

  \medskip
  \uncover<2->{A \alert{maximal matching} in the matching graph
    \uncover<3>{induces\\ \alert{perfect path phylogenies}.}}

  \hfill\hyperlink{return}{\beamerreturnbutton{Return}}
\end{frame}

\end{document}


