\section{Introdu��o}

Um problema bastante conhecido � o de escolher uma rota para se fazer 
uma viagem, tal que a rota minimize a dist�ncia 
do percurso. Nesta forma b�sica, esse problema � o problema de caminho 
m�nimo em grafos onde as arestas s�o poss�veis trechos, valorados por 
seu comprimento. Algumas vezes um caminho m�nimo desta forma � bom, 
outras vezes n�o. Existem ocasi�es, onde tal caminho possui propriedades
indesej�veis. Por exemplo, alguns trechos podem ter tr�fego denso e nos 
fazer perder muito tempo na travessia, ou existem muitos ped�gios com taxas
que, acumuladas pelo caminho, v�o exceder o dinheiro que temos dispon�vel.
Isso nos leva a considerar um ou mais par�metros adicionais para a escolha
do caminho. Os casos mais comuns de par�metros a considerar 
envolvem o consumo de recursos
em um or�amento que limita a quantidade dispon�vel desses
recursos. Um caminho m�nimo com essas limita��es adicionais � chamado de
{\bf caminho m�nimo com recursos limitados } 
({\it resource constrained shortest path - RCSP}) \cite{BC89}. 

\begin{problema}{\rcsp($G, s, t, k, r, l, c$)} 
Como par�metros do problema s�o dados:
\begin{itemize}
\item\ um grafo dirigido $G=(V,A)$,
\item\ um v�rtice origem $s \in V$ e um v�rtice destino $t \in V$, $s \neq t$,
\item\ um n�mero $k \in \mathbb{N}$ de recursos dispon�veis $\{1,\dots,k\}$,
\item\ o consumo de recursos $r^i_a \in \mathbb{N}_0$ de cada arco 
de $G$ sobre os  $k$ recursos dispon�veis, $i = 1, \dots, k$, $a \in A$,
\item\ o limite $l^i \in \mathbb{N}_0$ que dispomos de cada recurso, 
$i = 1, \dots, k$,
\item\ o custo $c_a \in \mathbb{N}_0$, para cada arco, $a \in A$.
\end{itemize}

O consumo de um recurso $i$, $i = 1, \dots, k$ em um $st$-caminho $P$ � 
$r^i(P) = \sum_{a \in P}{r^i_a}$. Um $st$-caminho $P$ � 
limitado pelos recursos $1, \dots, k$ se este consome n�o mais que o limite
dispon�vel de cada recurso, ou seja, se $ r^i(P) \leq l^i$, $i = 1, \dots, k$.
O custo de um $st$-caminho $P$ � $c(P) = \sum_{a \in P}{c_a}$.
O problema \rcsp\ consiste em encontrar o caminho limitado pelos recursos de
menor custo.

\end{problema}

Usaremos no decorrer deste trabalho $n = |V|$ e $m = |A|$. Quando
estivermos tratando de um contexto onde existe apenas um recurso, ou seja,
$k = 1$, usaremos apenas $l$ para representar $l^1$ e apenas $r_a$ para
representar $r^1_a$. 
