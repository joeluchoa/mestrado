\section{Hist�rico do Problema}

Os primeiros a apresentarem resultados sobre o problema foram 
Witzgall e Goldman 1965 \cite{WG65}. Joksch 1966 \cite{JO66}, 
trabalhando de forma independente, apresentou um resultado similar 
ao resultado de Witzgall e Goldman 1965 \cite{WG65} (veja tamb�m Lawler 1976 \cite{LA76}). 
Todos esses autores apresentaram algoritmos baseados em programa��o 
din�mica com complexidade pseudopolinomial, com recorr�ncias similares. 
A partir de ent�o,
o problema tem recebido grande aten��o de diversos pesquisadores.

Temos v�rios algoritmos exatos propostos ao longo do tempo para o \rcsp.
Dentre eles, cronologicamente, podemos citar Joskch 1966 \cite{JO66}, 
Lawler 1976 \cite{LA76}, Handler e Zang 1980 \cite{HZ80}, Aneja, Aggarwal 
e Nair 1983 \cite{AAN83}, Henig 1985 \cite{HE85}, Beasley e 
Christofides 1989 \cite{BC89}, Hassin 1992 \cite{RH92}. 
Os algoritmos descritos em \cite{HZ80} e \cite{BC89} usam uma relaxa��o 
lagrangeana da formula��o mais usual do \rcsp\ por programa��o linear. 
Vale ressaltar ainda que \cite{BC89} usa o m�todo de subgradiente
para resolver, aproximadamente, a relaxa��o lagrangeana.
Temos ainda Mehlhorn e Ziegelmann 2000 \cite{KM00}, que 
apresenta a abordagem do envolt�rio ({\it hull approach}), um
algoritmo combinat�rio para resolver uma relaxa��o de uma outra representa��o 
do \rcsp\ como um problema de programa��o linear.

Al�m dos algoritmos exatos, tamb�m surgiram algoritmos de aproxima��o.
Warburton 1987 \cite{WA87} desenvolveu um \fptas\ ({\it fully polynomial-time approximation scheme}) 
para o problema. Hassin 1992 \cite{RH92}, al�m de apresentar um algoritmo pseudo-polinomial, 
apresentou melhoras ao resultado de Warburton e tamb�m prop�s
um outro \fptas\ para o problema. Temos ainda mais um \fptas\ proposto por
Phillips 1993 \cite{PH93}.


