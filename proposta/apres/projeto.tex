
\begin{frame}
\frametitle{Projeto}

Nesse projeto, pretendemos 

\begin{itemize}
    \item desenvolver um estudo detalhado sobre o
            problema \rcsp. 
    \item expor os resultados mais relevantes,
            mantendo uma nota��o padronizada,
            tomando o cuidado de fazer provas detalhadas sobre corretude, 
            complexidade de espa�o e tempo de algoritmos. 
    \item fazer implementa��es dos algoritmos e 
            analisar seus comportamentos na pr�tica.
\end{itemize}

\end{frame}

\subsection{Estrutura da disserta��o}

\begin{frame}
\frametitle{Projeto}


Inicialmente pretendemos que a disserta��o siga uma estrutura semelhante
a deste projeto, apresentando os seguintes cap�tulos.

\begin{itemize}

\item Introdu��o: defini��o, hist�rico e aplica��es para o problema

\item Preliminares: nota��o, terminologias e conceitos b�sicos

\item Complexidade

\item Algoritmos exatos

\item Algoritmos de aproxima��o

\item Implementa��o e Resultados Computacionais

\end{itemize}


\end{frame}

\subsection{Planejamento}


\begin{frame}
\frametitle{Projeto}

{\scriptsize

Dentre as atividades que devem ser desempenhadas para o desenvolvimento desse
projeto, as principais s�o as seguintes.
\begin{enumerate}
\item Leitura de material sobre o problema e sobre conceitos afins,
necess�rios para o entendimento de todo o conte�do.
\item Implementa��o dos principais algoritmos e testes de tais implementa��es.
\item Realiza��o de testes e experimentos pr�ticos com inst�ncias geradas 
para o problema e an�lise de seus resultados. 
\item Escrita, revis�o e corre��o do texto da tese.
\item Defesa da disserta��o.
\end{enumerate}

\begin{table}[htb] % [htb]-> here, top, bottom
   \centering   % tabela centralizada
    % tamanho da fonte 
   \setlength{\arrayrulewidth}{2\arrayrulewidth}  % espessura da  linha
   \setlength{\belowcaptionskip}{10pt}  % espa�o entre caption e tabela
   \caption{\it Distribui��o do tempo por atividade}
   \begin{tabular}{|c|c|c|c|c|c|c|c|} % c=center, l=left, r=right 
      \hline
      Atividade & Set& Out & Nov & Dez & Jan & Fev & Mar\\
      \hline \hline
      [1] & $\surd$ & $\surd$ &  &  &  &  &  \\
      \hline
      [2] &  $\surd$ &  $\surd$ & $\surd$ &  &  &  & \\
      \hline
      [3] &  &  $\surd$ & $\surd$ &  $\surd$ &  &  & \\
      \hline
      [4] & & & $\surd$ & $\surd$ & $\surd$ & $\surd$ & \\
      \hline
      [5] &  &  &  &  & &  & $\surd$ \\
      \hline
   \end{tabular}
   \label{tab:Referencia_desejada}
\end{table}

}

\end{frame}
